%-------------------------------------------------------------------------
% INFORMACIÓN DEL ARTÍCULO
\thispagestyle{portadapage}
\setcounter{section}{0}
\setcounter{subsection}{0}
\setcounter{subsubsection}{0}
\setcounter{actividad}{0}
\setcounter{actividad_previa}{0}
\setcounter{actividad_entre}{0}
\renewcommand{\articulotipo}{Taller}
\renewcommand{\articulotitulo}{Preparación de estudiantes para las Olimpiadas Matemáticas: Técnicas de preparación}
\phantomsection
\stepcounter{section}
\addcontentsline{toc}{section}{\protect\numberline{\thesection} \articulotitulo}
\desctotoc{del Valle Vides, C. N.; Chocobar, E. F.}

\begin{center}
	\setstretch{1.5}
	{\Huge \scshape
		\articulotitulo
	}
\end{center}

\noindent\rule{\linewidth}{2pt}

\vspace{0.25cm}

\begin{flushright}
	{\Large \scshape
		Antonio Noé Sángari
	}\\
	{\large \itshape
		Universidad Nacional de Salta
	}\\
	{\ttfamily \small
		diamantecinthia@gmail.com
	}\\
	\vspace{1em}
	{\Large \scshape
		Veronica Flores Rocha
	}\\
	{\large \itshape
		Universidad Nacional de Salta
	}\\
	\vspace{1em}
	{\Large \scshape
		Nadia Noel Ghiglia
	}\\
	{\large \itshape
		Universidad Nacional de Salta
	}\\
	\vspace{1em}
	{\Large \scshape
		Silvia Ester Coria
	}\\
	{\large \itshape
		Universidad Nacional de Salta
	}
\end{flushright}

\vspace{0.5cm}

\begin{center}
	\begin{minipage}{0.75\linewidth} \small
		\textsc{Resumen}. ~
		Este curso-taller está diseñado para capacitar a docentes de matemáticas de secundaria en técnicas de demostración matemática, con el objetivo de preparar a sus estudiantes para participar en las Olimpiadas Matemáticas. Se enfocará en el desarrollo de habilidades de razonamiento lógico y deductivo, la enseñanza de estrategias didácticas específicas y la aplicación de estos conocimientos en la práctica. Los participantes tendrán acceso a materiales teóricos, actividades prácticas y herramientas de evaluación para asegurar un aprendizaje integral.
	\end{minipage}
	
	\vspace{0.5em}
	
	\begin{minipage}{0.75\linewidth} \small
		\textsc{Palabras clave} --- Curso-taller, Matemáticas, Olimpiadas Matemáticas, Razonamiento lógico, Razonamiento deductivo.
	\end{minipage}
\end{center}
%-------------------------------------------------------------------------

\subsection{Introducción}

\subsubsection{Sobre el taller}

El curso-taller «Preparación de Estudiantes para las Olimpiadas Matemáticas: Técnicas de Demostración» está diseñado específicamente para docentes de matemáticas de nivel secundario. Este programa tiene como objetivo principal dotar a los profesores de las herramientas y estrategias necesarias para enseñar a sus estudiantes a construir demostraciones matemáticas rigurosas. La habilidad de realizar demostraciones precisas es un aspecto crucial en las Olimpiadas Matemáticas, con un énfasis particular en la geometría básica.

En el contexto de las Olimpiadas Matemáticas, tanto a nivel nacional como internacional, la capacidad de los estudiantes para presentar demostraciones claras y lógicas es fundamental para su éxito. Por esta razón, el curso-taller pone un énfasis especial en proporcionar a los docentes un sólido soporte bibliográfico a través de la Red Olímpica. Esta red ofrece una guía valiosa para las demostraciones esperadas en las competiciones, ayudando a los estudiantes a prepararse adecuadamente y a competir con confianza en los diferentes niveles de las Olimpiadas.

Con este curso-taller, se espera que los docentes adquieran un mayor dominio en la enseñanza de técnicas de demostración y, a su vez, transmitan estos conocimientos a sus alumnos, potenciando así sus habilidades matemáticas y su desempeño en las competiciones.

\subsubsection{Importancia de la enseñanza de la demostración}

Destacamos que en el contexto de la Olimpiada Matemática Argentina, la demostración ocupa un lugar central y fundamental en la resolución de problemas matemáticos. A diferencia de las respuestas simples o los cálculos rápidos, una demostración matemática exige un razonamiento riguroso y lógico.

La demostración es esencial en la Olimpiada Matemática porque asegura la validez de las soluciones, desarrolla habilidades críticas y analíticas, promueve una comunicación efectiva, profundiza la comprensión matemática y permite una evaluación justa y completa de las capacidades
de los estudiantes.

\subsection{Contenidos}

\subsubsection{Módulo 1: Definición y tipos de demostraciones.}
\begin{itemize}
	\item Elementos de una demostración rigurosa.
	\item Ejemplos históricos y su importancia.
	\item Ejemplos de certámenes nacionales e internacionales de la Olimpiada Matemática Argentina, tanto en soporte bibliográfico como en publicación en linea.
	\item Estrategias didácticas para enseñar estos conceptos.
\end{itemize}

\subsubsection{Módulo 2: Técnicas Básicas de Demostración}
\begin{itemize}
	\item Demostraciones directas.
	\item Demostraciones por contradicción.
	\item Demostraciones por contraposición.
	\item Ejercicios prácticos y resolución de problemas.
	\item Métodos para enseñar estas técnicas a los estudiantes.
\end{itemize}

\subsubsection{Módulo 3: Geometría}
\begin{itemize}
	\item Postulados y teoremas fundamentales.
	\item Construcción de demostraciones geométricas.
	\item Aplicación de teoremas en problemas de Olimpiadas en certámenes regionales, nacionales e internacionales.
	\item Actividades prácticas.
\end{itemize}

\subsection{Requisitos previos}

Docentes de matemáticas de nivel secundario y estudiantes avanzados de profesorados de matemática, interesados en preparar a sus estudiantes para participar en Olimpiadas Matemáticas.

\subsection{Objetivos}
\begin{enumerate}
	\item Capacitar a los docentes en técnicas de demostración matemática.
	\item Fomentar habilidades de razonamiento lógico y deductivo en sus estudiantes.
	\item Proveer estrategias didácticas para enseñar demostraciones.
	\item Preparar a los docentes para guiar a sus estudiantes en competiciones matemáticas regionales, nacionales e internacionales.
\end{enumerate}

\subsection{Actividades}

\subsubsection{Actividades previas}

Antes del inicio formal del curso-taller, los participantes tendrán acceso a un curso en la plataforma Moodle de la Facultad de Ciencias Exactas. En esta plataforma, se adjuntarán extractos esenciales de elementos de una prueba en matemática, especialmente del texto de \textcite{margaris1968} sobre lógica formal como soporte básico y del clásico de \textcite{polya1945} sobre resolución de problemas y organización de demostraciones. Además, se proporcionarán videos explicativos que cubren demostraciones básicas para familiarizar a los docentes con los conceptos fundamentales. El programa y el cronograma detallado de actividades del curso estarán disponibles en la plataforma, permitiendo a los participantes planificar su tiempo de estudio. Para asegurar que los docentes asimilen los conceptos preliminares, se implementará un cuestionario de autoevaluación, el cual los participantes deberán completar antes de la primera sesión sincrónica.

\subsubsection{Primeras hora y media sincrónicas}

La primera sesión sincrónica comenzará con una breve presentación del taller, estableciendo los objetivos y la estructura del curso. Se introducirán las ideas básicas de los elementos de una demostración, proporcionando una visión general de los conceptos fundamentales que serán abordados en profundidad más adelante. Se solicitará la participación activa de los cursantes, quienes serán invitados a compartir sus perspectivas y experiencias sobre la enseñanza de demostraciones matemáticas. Esta discusión inicial ayudará a identificar los conocimientos previos de los docentes y a ajustar el enfoque del taller según sus necesidades.

\subsubsection{Primeras tres horas entre clases}

Estas horas estarán
dedicadas a la preparación de las próximas clases sincrónicas centradas en el módulo 2, que aborda diversos métodos de demostración y técnicas para enseñarlos. Se pondrán a disposición textos extraídos del libro de \textcite{larson1983} sobre resolución de problemas para estructurar los métodos de demostración. Adicionalmente, se seleccionarán problemas relevantes de los libros de \textcite{araujo2020a, araujo2020b, araujo2020c, fauring2000, fauring2023}, los cuales serán utilizados como ejemplos prácticos durante las sesiones sincrónicas. Los participantes deberán revisar estos materiales y reflexionar sobre su aplicación en el aula, preparando preguntas y comentarios para la próxima sesión.

\subsubsection{Segundas hora y media sincrónicas}

En esta sesión, se abordará el módulo 2 siguiendo una metodología similar a la de las primeras horas sincrónicas. El enfoque estará en las técnicas básicas de demostración y en métodos pedagógicos para enseñar estas técnicas a los estudiantes. Los cursantes serán invitados a explicar el material que se les asignó para lectura durante las primeras tres horas entre clases, fomentando así una discusión profunda y colaborativa. Se debatirán posibles modificaciones a los ejercicios presentados para obtener respuestas más generales y abarcativas, permitiendo a los docentes adaptar los problemas a diferentes niveles de dificultad y contextos de enseñanza.

\subsubsection{Segundas tres horas entre clases}

Durante estas horas, los participantes prepararán las próximas clases sincrónicas que continuarán con el módulo 3. Este módulo se centrará en la justificación de las hipótesis y en el camino lógico necesario para obtener los teoremas más comunes en la geometría elemental. Se proporcionarán materiales adicionales que exploran estos conceptos en profundidad, incluyendo ejemplos prácticos y ejercicios para resolver. Los docentes deberán revisar estos materiales y preparar sus propias explicaciones y preguntas para la próxima sesión sincrónica, asegurando así una comprensión sólida de los temas tratados.

\subsubsection{Terceras hora y media sincrónicas}

La tercera sesión sincrónica continuará con el módulo 3, siguiendo el mismo enfoque interactivo y participativo de las sesiones anteriores. Se profundizará en las técnicas de demostración y resolución de problemas, con un énfasis especial en la búsqueda de primeros principios para fundamentar los resultados obtenidos. Se dedicará la parte final de esta sesión a explicar la evaluación final del curso-taller, aclarando cualquier duda que los participantes puedan tener. Esto asegurará que todos los cursantes estén bien preparados para demostrar su comprensión y aplicación de las técnicas de demostración matemática enseñadas a lo largo del taller.

\subsubsection{Evaluación final}

\begin{itemize}
	\item Desarrollo de recursos educativos para el entrenamiento de estudiantes, centrados en el razonamiento lógico de deducciones geométricas.
	\item Cuestionario final que aborde los principales temas del taller.
	\item Reflexión escrita sobre la experiencia del taller y el aprendizaje obtenido.
\end{itemize}

\subsection{Bibliografía}

\nocite{*}
\printbibliography[keyword={01}]