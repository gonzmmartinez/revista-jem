\section*{\centering \sffamily Prólogo}

Las VII Jornadas de Enseñanza de la Matemática híbridas, realizadas en su parte presencial en las instalaciones de la Universidad Nacional de Salta del 24 de julio al 2 de agosto de 2023, representaron un hito en el continuo esfuerzo por mejorar la calidad de la educación matemática en nuestra región. Durante esos días, docentes, investigadores y estudiantes se congregaron con el objetivo de reflexionar sobre sus prácticas docentes, compartiendo experiencias y conocimientos que, sin duda, contribuirán al enriquecimiento de la comunidad educativa.

Las Jornadas ofrecieron un variado programa de actividades que incluyó conferencias magistrales, talleres prácticos y comunicaciones breves. Entre las actividades destacadas, se encuentran las conferencias de Valeria Borsani, que abordó \emph{El pasaje de la aritmética al álgebra en los primeros años de la escuela media}, la de Andrés Rieznik, que abordó \emph{Discalculia, un capítulo olvidado de la neuropsicología}, y el de Daniela Reyes, \emph{Empoderamiento docente, ¿por qué pensar en ello?} y cinco talleres sobre enseñanza de la matemática. La participación activa de los asistentes reflejó el interés y el compromiso con la mejora continua de la enseñanza de la matemática.

Queremos expresar nuestro más sincero agradecimiento a todos los participantes, ponentes y talleristas que, con su dedicación y entusiasmo, hicieron posible el éxito de este evento. Asimismo, extendemos nuestro reconocimiento a la Universidad Nacional de Salta y a todas las instituciones que nos brindaron su apoyo.

Las presentes Memorias recogen las ideas, debates y conclusiones surgidas durante las Jornadas, sirviendo como un valioso recurso para todos aquellos interesados en la enseñanza de la matemática. Esperamos que este compendio inspire a los lectores a continuar explorando y aplicando nuevas estrategias pedagógicas en sus aulas.

Con la mirada puesta en el futuro, confiamos en que las próximas ediciones de las Jornadas de Enseñanza de la Matemática seguirán siendo un espacio de encuentro, reflexión e innovación para nuestra comunidad.

\vspace{5cm}

\begin{center}
	\large Prof. Antonio Noé Sángari
	
	\normalfont Coordinador de las VII Jornadas de Enseñanza de la Matemática
\end{center}