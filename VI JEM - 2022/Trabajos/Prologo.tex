\section*{\centering \sffamily Prólogo}

Las dificultades en el proceso de enseñanza y aprendizaje de la Matemática son cuestiones de análisis de diversas investigaciones en educación matemática y tratados en comunidades científicas y por docentes de esta disciplina. 

La Matemática se enseña a lo largo de toda la escolaridad obligatoria. Esto significa que durante muchos años los estudiantes están en contacto con esta disciplina, por lo que se espera que no tengan mayores dificultades al momento de iniciar los estudios superiores. La realidad muestra que no es así y que es necesaria la capacitación de los docentes y la puesta en común de las distintas problemáticas que ellos afrontan en cada nivel. 

La experiencia de las primeras cinco ediciones de estas Jornadas (JEM I, II, III, IV y V) llevadas a cabo en esta Universidad, con gran participación e interés de docentes de todos los niveles educativos y de estudiantes, no solo de nuestra provincia sino de otras, motiva y justifica la enorme tarea que demandó la realización de las Sextas Jornadas de Enseñanza de la Matemática. 

En las Quintas Jornadas de Enseñanza de la Matemática, se contó con la participación de aproximadamente 300 asistentes entre docentes de todos los niveles educativos y estudiantes en cada una. Ellos expresaron al momento de las conclusiones su interés por que actividades como las desarrolladas tengan continuidad en el tiempo.
\begin{itemize}
	\item En las encuestas realizadas a los asistentes, destacaron: 
	\item La calidad de las diferentes actividades desarrolladas. 
	\item El nivel académico y humano de los disertantes y talleristas. 
	\item El hecho de que las jornadas cubrieran un espacio que era requerido, deseado y sumamente necesario para los docentes de los diferentes niveles educativos. 
\end{itemize}

El Departamento de Matemática de la Facultad de Ciencias Exactas de la Universidad Nacional de Salta, a través de la función de extensión al medio, propuso la realización de las VI Jornadas de Enseñanza de la Matemática, en la que docentes de los diferentes niveles participaron de talleres y de comunicaciones breves que permitieron la reflexión y el intercambio de ideas y experiencias. Ello contribuyó a superar las dificultades existentes y a proponer posibles soluciones a la problemática planteada. En esta oportunidad, las VI Jornadas de la Enseñanza de la Matemática se centraron en la temática "La enseñanza de la matemática en diferentes modalidades". 
