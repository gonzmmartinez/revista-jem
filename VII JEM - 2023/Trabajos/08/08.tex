%-------------------------------------------------------------------------
% INFORMACIÓN DEL ARTÍCULO
\thispagestyle{portadapage}
\setcounter{subsection}{0}
\setcounter{subsubsection}{0}
\setcounter{actividad}{0}
\setcounter{actividad_previa}{0}
\setcounter{actividad_entre}{0}
\renewcommand{\articulotipo}{Comunicación breve}
\renewcommand{\articulotitulo}{Epistemología e historia de la Matemática: evaluación con monografía y ejemplo}
\phantomsection
\stepcounter{section}
\addcontentsline{toc}{section}{\protect\numberline{\thesection} \articulotitulo}
\desctotoc{Ferrero, M. M.; Nuñez, A. V.}

\begin{center}
	\setstretch{1.5}
	{\Huge \scshape 
		\articulotitulo
	}
\end{center}

\noindent\rule{\linewidth}{2pt}

\vspace{0.25cm}

\begin{flushright}
	{\Large \scshape
		María Martha Ferrero
	}\\
	{\large \itshape
		Universidad Nacional del Comahue
	}\\
	{\ttfamily \small
		marthaferrero@gmail.com
	}\\ \vspace{1em}
	{\Large \scshape
		Abraham Vargas Nuñez
	}\\
	{\large \itshape
		Universidad Nacional del Comahue
	}\\
\end{flushright}

\vspace{0.5cm}

\begin{center}
	\begin{minipage}{0.75\linewidth} \small
		\textsc{Resumen}. ~
		Numerosos planes de estudio de la carrera Profesorado en Matemática incluyen una materia de “Epistemología e Historia de la Matemática” con la finalidad de animar a que los futuros profesores reflexionen, analicen y conozcan el proceso de construcción del conocimiento matemático, así como también profundicen en los aspectos culturales y epistemológicos de la disciplina. En la sede Bariloche de la Universidad Nacional del Comahue intentamos lograr en los-las estudiantes una visión integradora de los temas de Matemática que a lo largo de la carrera han visto en distintas materias y desde distintos abordajes. La evaluación monográfica como cierre de la materia, ha resultado una herramienta valiosa como oportunidad para que los alumnos profundicen un tema de Matemática o de la historia de la Matemática que les interese especialmente, claramente delimitado y  desarrollado de forma lógica, recurriendo a la investigación bibliográfica. En esta comunicación, presentamos un resumen de la monografía “\textit{El problema isoperimétrico}”, añadiendo comentarios referidos a los contenidos trabajados en la cursada. En la elaboración del trabajo monográfico se presentan numerosas oportunidades de discutir entre pares y con la docente acerca de las temáticas generales de la Matemática y particulares de cada uno de los temas elegidos. Se trata de una producción andamiada y colaborativa, con perspectiva al futuro desempeño de los-las estudiantes como docentes.
	\end{minipage}
\end{center}
%-------------------------------------------------------------------------

\subsection{Introducción}

En la actualidad, encontramos que los planes de estudio de la carrera Profesorado en Matemática incluyen una materia de “Epistemología e Historia de la Matemática” con la finalidad de animar a que los futuros profesores reflexionen, analicen y conozcan el proceso de construcción del conocimiento matemático, así como también profundicen en los aspectos culturales y epistemológicos de la disciplina.

En la sede B. de la U. N. intentamos lograr en los/las estudiantes una visión integradora de los temas de Matemática que a lo largo de la carrera han visto en distintas materias y desde distintos abordajes, teniendo en cuenta que después de 4 años de formación tienen una mirada m´as cercana a la mirada experta por lo cual pueden resignificar sus aprendizajes y proyectarse en su futuro desempeño como profesores. Los/las estudiantes revisan entonces su propia experiencia y algunos conceptos e ideas matemáticas son contextualizadas desde un punto de vista histórico y cultural.

La evaluación monográfica como cierre de la materia, ha resultado una herramienta valiosa como oportunidad para que los alumnos profundicen un tema de Matemática o de la historia de la Matemática que les interese especialmente, claramente delimitado y desarrollado de forma lógica, recurriendo a la investigación bibliográfica.

En esta comunicación, presentamos un resumen de la monografía (producto final) “\textit{El problema isoperimétrico}” del estudiante A. V. N., añadiendo comentarios referidos a los contenidos trabajados en la cursada y a modo de cierre evaluamos la dinámica del proceso de su elaboración, mostrando el diálogo docente-alumno en cuanto a sugerencias bibliográficas, adecuación y retroalimentación, así como también algunas valoraciones sobre el aprendizaje que representa este modo de trabajo.

\subsection{Desarrollo}

En primer lugar, cabe mencionar que etimológicamente «isoperimetría» significa literalmente “con igual perímetro”. En Matemática, la isoperimetría es el estudio general de las figuras geométricas planas que tienen contornos iguales y su relación con el área que encierran

De manera explícita se prioriza desde la cátedra un acercamiento histórico y epistemológico a los temas, siendo que los problemas didácticos y cognitivos que la relación entre perímetro y área de figuras puede suscitar escapan al alcance de la materia (si bien forman parte de los procesos de discusión). Es así que, a quienes quieran profundizar en estos aspectos recomendamos el artículo de \textcite{damore2007}, “\textit{Relaciones entre área y perímetro: convicciones de maestros y de estudiantes}”.

A continuación se hace un relato “a dos voces” en que la primera corresponde a extractos de la monografía del estudiante y la segunda contiene comentarios relacionados con los contenidos históricos y epistemológicos trabajados durante la cursada.

\subsubsection{Introducción a la monografía}

\begin{quote}
	Todas las personas en la toma de decisiones, con las cuales pretenden obtener alguna ganancia, involucran procesos de optimización. Cuando se trata de situaciones sencillas, los procesos también lo son; y la puesta en juego de la intuición y experiencia personal, tal vez, juegan un rol importante. Sin embargo, a medida que se abandonan las situaciones más triviales y la mirada se centra en aquellas donde la solución óptima es fundamental, por diferentes motivos, resalta la necesidad de comprender e investigar las operaciones óptimas para la toma de decisiones; y por tanto, surge en la sociedad la tendencia a la complejización y sofisticación matemática para poder dar respuestas viables y más favorables a diferentes problemáticas (\textcite{hernandez2011,zavala2014}).
\end{quote}
\vspace{1em}

Las tendencias recientes en filosofía de las matemáticas reconocen un triple carácter en esta disciplina: las matemáticas como quehacer humano, comprometido con la resolución de cierta clase de situaciones problemáticas; las matemáticas como lenguaje simbólico y como un sistema conceptual lógicamente organizado y socialmente compartido, emergente de la actividad de matematización. (\textcite{godino2003}).

\vspace{1em}
\begin{quote}
	Sujeto a estos argumentos es que, cuando estaba cursando el primer año del Profesorado Universitario en Matemáticas y estaba preparándome para rendir el final de la materia Geometría Euclidiana, surgieron de entre los conceptos que iban y venían, y entre las múltiples hojas que escribía resolviendo diversos ejercicios, la siguiente cadena de interrogantes: ¿cuál es el triángulo que encierra el área máxima entre todos los triángulos de igual perímetro?, más aún, ¿cuál es el polígono de n lados que encierra área máxima entre todos los polígonos de $n$ lados cuyo perímetro es idéntico?; y finalmente, ¿cuál es la curva que encierra máxima área entre todas las curvas de igual longitud? Sin darme cuenta, me había sumergido en una problemática ya existente de un problema que fue discutido durante más de dos mil años para ser resuelto: el problema isoperimétrico.
\end{quote}
\vspace{1em}

Recorte acorde a los alcances de la monografía: \textit{Tratamiento en el marco de la Geometría Euclidiana}, materia en que ha surgido la inquietud del estudiante por explorar y profundizar en el tema elegido.

\subsubsection{El problema isoperimétrico}

\begin{quote}
	El problema isoperimétrico es un problema de optimización (llamaremos problema de optimización a todo aquel en el cual el objetivo fundamental es obtener un valor máximo o un valor mínimo de alguna variable). Claro que no fue entendido ni clasificado como tal, sino hasta las proximidades del siglo XIX.
\end{quote}
\vspace{1em}

Encuadre del tema en un área específica de la matemática: \textbf{optimización}. Dimensión matemática.

\paragraph{El problema isoperimétrico como un hito histórico}

\begin{quote}
	La historia muestra grandes hitos respecto al estudio de los problemas de optimización. Algunos de estos son:
	\begin{itemize}
		\item La obra de Pappus de Alejandría “Colección matemática” (320 a.C.) 
		\item Los problemas isoperimétricos (Zenodoro, Gergonne, Steiner).
		\item El desarrollo del cálculo diferencial en el siglo XVII, reconocida por el uso de derivadas para resolver problemas de máximos y mínimos. Distinguida aún más, por los aportes de Euler, quien propone y crea el cálculo de variaciones, considerando la obtención de funciones que optimizan funciones. Esto proporcionó valiosas herramientas matemáticas para afrontar problemas más avanzados.
		\item El desarrollo de la programación lineal en la primera mitad del siglo XX. Kantorovich y Koopmans recibieron el premio Nobel de Economía en 1975 (\textit{teoría de la asignación óptima de recursos}).
	\end{itemize}
\end{quote}
\vspace{1em}

Encuadre del tema como hito histórico en optimización junto con otros hitos en el área. Dimensión histórica

\paragraph{Origen del problema isoperimétrico}

\begin{quote}
	Existen varias fuentes que narran el surgimiento del problema isoperimétrico, sin duda alguna, todas sostienen que es una leyenda que hunde sus raíces en la mitología. La misma nació a continuación de la migración fenicia, cuando Dido, o Elisa, se estableció en el norte de África (en la región que actualmente se llama Túnez).
\end{quote}
\vspace{1em}

Elisa llegó a las costas de África, donde vivían los gétulos, una tribu de libios cuyo rey era Jarbas. Pidió hospitalidad y un trozo de tierra para instalarse en ella con su séquito. Jarbas le expuso que le daría tanta tierra como ella pudiera abarcar con una piel de buey. Elisa, a fin de que la piel abarcara la máxima tierra posible, la hizo cortar en finas tiras y así consiguió circunscribir un extenso perímetro. Tras esto hizo erigir una fortaleza llamada Birsa, que más tarde se convirtió en la ciudad de Carthago o Qart-Hadašh, sobre un promontorio existente entre el lago de Túnez y la laguna Sebkah er-Riana, que desembocaba en mar abierto. Instaurada como soberana de la ciudadela, recibió de los habitantes el nombre de Dido. Dido es sin duda una mujer excepcional. \textcite{arguedas2013}.

No es fortuito que las historietas y leyendas que ligan área y perímetro sean antiquísimas y se repitan en el tiempo, incluso a distancia de siglos (basta pensar en el mito sobre la fundación de Cartagine por parte de Didone y a la célebre adivinanza de Galileo). Esta es una señal, no más que una señal, por supuesto, de obstáculo epistemológico; por otra parte: cuando una idea matemática no entra inmediatamente a formar parte de esta disciplina y, por el contrario, es causa de discusiones, contestaciones, luchas; generalmente puede considerarse un obstáculo epistemológico en el sentido de Brousseau. \textcite{damore2007}.

Dimensión epistemológica.

\paragraph{El problema isoperimétrico “vivo” durante 2000 años}

\begin{quote}
	El problema isoperimétrico no quedó resuelto, en la época de Dido, desde un punto de vista matemático. Existieron muchos matemáticos que se esforzaron en resolver este problema. Partiendo desde el griego Zenodoro, quien hizo grandes contribuciones para la resolución de este aporte, y vivió en torno al 200 a.C.; y concluyendo en principios del siglo XIX, cuando el problema fue completamente resuelto por Weierstrass, el cual utilizó el cálculo de variaciones.
\end{quote}
\vspace{1em}

Las matemáticas constituyen, por tanto, una realidad cultural constituida por conceptos, proposiciones, teorías, etc. (los objetos matemáticos) y cuya significación personal e institucional está íntimamente ligada a los sistemas de prácticas realizadas para la resolución de las situaciones-problemas. \textcite{godino2003}.

\subsubsection{Zenodoro, el geómetra griego}

\begin{quote}
	Zenodoro, al parecer, vivió en Atenas, aproximadamente entre los años 200 y 140 a.C. Su trabajo sobre figuras isoperimétricas se conoce por medio de algunas referencias, como la que realiza Teón de Alejandría (335-405), matemático griego, en sus amplios comentarios al Almagesto de Ptolomeo. Pappus (290-350) también hace uso de las proposiciones de Zenodoro en el libro V de su \textit{Colección Matemática} (\textcite{herrero2011}).
	
	Zenodoro, tras abordar el problema isoperimétrico, intenta demostrar que el círculo tiene mayor área que cualquier polígono con el mismo perímetro.
	
	Una de las propiedades que demostró Zenodoro falla. Fue, probablemente Pappus, quien se percató de tal error y hace una nueva demostración tras recoger los escritos de Zenodoro. Años más tarde, Lhulier (1750-1840) y Steiner (1796-1863) resuelven el problema.
\end{quote}
\vspace{1em}

En este apartado el estudiante describe los lemas y teoremas enunciados por Zenodoro, muestra varias ilustraciones y realiza un análisis de los mismos. Se detiene en el examen de un argumento de Zenodoro fallido y, si bien no ha quedado registrado en la monografía, se buscaron contraejemplos con el software GeoGebra.

\subsubsection{Una nueva forma de abordar el problema isoperimétrico: Gergonne}

\begin{quote}
	Joseph Diaz Gergonne (1771-1859), es el primer autor que se plantea resolver el problema isoperimétrico sin recurrir a los polígonos, lo que viene a ser un aporte diferente a los que se venían registrando.

	Una figura que tenga perímetro fijo y área máxima, ha de ser convexa ya que, si no lo fuera bastará tomar su envoltura convexa (menor conjunto convexo que lo contiene), que tendrá perímetro menor y área mayor. Sólo habrá coincidencia en caso de que la figura de partida fuera convexa. \textcite[6]{herrero2011}.
\end{quote}
\vspace{1em}

El estudiante resalta el cambio de perspectiva en el abordaje del problema hacia la generalización (de polígonos a figuras). Aparecen nuevos conceptos y procedimientos matemáticos: convexidad, simetrización.

\subsubsection{El matemático cuyo nombre está ligado con más fuerza al problema isoperimétrico: Jakob Steiner (1796-1863)}

\begin{quote}
	El matemático Jakob Steiner (1796-1863) realizó varias demostraciones en el contexto de extensos e interesantes trabajos sobre diferentes aspectos de los máximos y mínimos de medidas asociadas a diferentes figuras (\textcite{herrero2011}). Las demostraciones de Steiner encierran construcciones y razonamientos puramente geométricos. Sin embargo, a Steiner se le reprocha, que en sus demostraciones da por supuesta la existencia de solución. Lo mismo ocurre con el razonamiento de Gergonne.
	
	En general, las demostraciones de Steiner utilizan el razonamiento por el absurdo, a pesar de que se trate de argumentos y construcciones diferentes. Esto es, suponer que existe una figura no circular con perímetro fijo y área máxima, y luego demostrar que se puede construir otra figura con el mismo perímetro que tiene mayor área. Lo cual es una contradicción con lo supuesto. Y finalmente, se concluye que la figura óptima debe ser un círculo ya que las nuevas figuras presentan propiedades que solo tiene este.
	
	Demostración del Teorema Principal de Steiner.
\end{quote}
\vspace{1em}

Acá se aborda otro elemento epistemológico a considerar: la demostración por absurdo y no por construcción, que ha generado numerosos debates al interior de la comunidad matemática.

\subsubsection{El problema isoperimétrico resuelto después de 2000 años}

\begin{quote}
	Fue el ilustre matemático K. Weierstrass (1815-1897) quien dio solución al problema isoperimétrico, y su resolución no vino de la mano de la geometría euclidiana. Posteriormente otros matemáticos como Hurwitz (1859-1919), Blaschke (1855-1962), Schmidt (1876-1959) y Santaló (1911-2001) ofrecieron otras soluciones al problema isoperimétrico abordado desde diferentes caminos. Los “famosos” problemas de optimización o de extremos ligados que se trabajan en carreras universitarias son generalmente abordados utilizando el Teorema de los Multiplicadores de Lagrange, propuesto por el matemático, físico y astrónomo Joseph-Louis Lagrange (1736- 1813). No obstante, Bonnesen ofrece una prueba que mejora la desigualdad isoperimétrica.
\end{quote}
\vspace{1em}

... disponemos de todo un sistema conceptual previo, herencia del trabajo anterior de las mentes matemáticas más capaces, que nos proporcionan la solución de un sinnúmero de problemas. Esta herencia quedaría desaprovechada si cada estudiante tuviese que redescubrir por sí mismo todos los conceptos que se le tratan de enseñar. La ciencia, y en particular las matemáticas, no se construye en el vacío, sino sobre los pilares de los conocimientos construidos por nuestros predecesores. (\textcite{godino2003}).

\subsection{Reflexiones finales}

\begin{quote}
	\begin{enumerate}[series=isoper]
		\item En la evolución del problema isoperimétrico hubo diferentes aportes de diversos matemáticos en diferentes tiempos y partes del mundo. Existieron también, errores sustanciales en algunas de las demostraciones propuestas como solución al problema isoperimétrico, dicho desde un punto de vista matemático; sin embargo, se observó una actitud constructiva por parte de los que retomaron la problemática. Pues, si bien, los que retomaban el problema señalaban el error en algunas de las demostraciones que fueron presentadas por otros matemáticos en el pasado (por ejemplo en las demostraciones Zenodoro, Gergonne, Steiner, etc), continuaban trabajando en búsqueda de una solución y retomando las buenas conclusiones de los matemáticos que les precedieron.
		
		Por otro lado, ha contribuido a la resolución del problema isoperimétrico un cambio de perspectiva. Esto resulta importante, pues una mirada reflexiva sobre la evolución histórica del problema isoperimétrico, nos invitaría a plantearnos que, si ya se ha probado por tanto tiempo resolver un problema por un determinado camino, ¿por qué no abordarlo desde otro lugar?
	\end{enumerate}
\end{quote}
\vspace{1em}

...si se considera que las matemáticas son una construcción humana que surge como consecuencia de la necesidad y curiosidad del hombre por resolver cierta clase de problemas o disposiciones del entorno; que, asimismo, en la invención de los objetos matemáticos tiene lugar un proceso de negociación social y que estos objetos son falibles y sujetos a evolución, entonces el aprendizaje y la enseñanza debe tener en cuenta estos procesos. (\textcite{godino2003}).

\vspace{1em}
\begin{quote}
	\begin{enumerate}[resume=isoper]
		\item El avance en los distintos campos de la matemática ha contribuido en la resolución de múltiples problemas, en particular, el problema isoperimétrico. Como se observó, con las herramientas del cálculo de variables, del cual no disponía Zenodoro, se resolvió el problema de una forma muy sencilla. Entonces, este es un punto de reflexión y, en alguna medida, de confianza para el futuro. Pues, muchos de los problemas matemáticos que actualmente no pueden ser resueltos es probable que más adelante se logren resolver.
	\end{enumerate}
\end{quote}
\vspace{1em}

Las herramientas materiales, conceptuales y tecnológicas influyen y condicionan la actividad matemática y por tanto a la Matemática misma.

\vspace{1em}
\begin{quote}
	\begin{enumerate}[resume=isoper]
		\item En mi propia experiencia, durante la lectura y comprensión de los teoremas explorados de los diferentes autores, noté que el lenguaje matemático ha ido evolucionando a nuestro favor en cuanto a la facilitación de la comunicación de ideas entre matemáticos. La universalización del lenguaje matemático, ayuda a agilizar la comprensión y el crecimiento de la ciencia matemática.
	\end{enumerate}
\end{quote}
\vspace{1em}

Las matemáticas son un lenguaje simbólico en el que se expresan las situaciones-problemas y las soluciones encontradas; ..., como todo lenguaje implica unas reglas de uso que hay que conocer y su aprendizaje ocasiona dificultades similares al aprendizaje de otro lenguaje no materno. (\textcite{godino2003})

\vspace{1em}
\begin{quote}
	\begin{enumerate}[resume=isoper]
		\item Otro aspecto a reflexionar sobre la actividad que manifiestan, en líneas generales, los que trabajan (o trabajaron) con matemática es el hecho de que a pesar de que un problema haya sido resuelto, si no se logró resolver por un determinado camino por el cual se había intentado, parece ser una invitación para muchos a querer hallar la solución por tal recorrido, o al menos querer entender por qué no se pudo ir por ese lado. ¿Cuáles serán los fundamentos de esta actitud, tantas veces vista a lo largo de la humanidad?
	\end{enumerate}
\end{quote}
\vspace{1em}

Para llevar a cabo este análisis se deben operar transformaciones geométricas sobre las figuras, pero sólo a finales del siglo XIX estas transformaciones, su potencia, su necesidad, se revelaron completamente a los ojos de los matemáticos; por milenios dominó la rigidez de los Elementos de Euclides; incluso este retardo en la introducción-aceptación es una obvia señal de obstáculo epistemológico. (\textcite{damore2007}).

Dimensión epistemológica

\vspace{1em}
\begin{quote}
	\begin{enumerate}[resume=isoper]
		\item Por último, las TIC son herramientas con alto potencial matemático que pueden ser usadas a nuestro favor. Estoy seguro de que lo que para Zenodoro representó una gran dificultad, y lo llevó a conclusiones equivocadas, no le hubiera pasado si hubiese sabido utilizar GeoGebra, por ejemplo. Y como se mencionó anteriormente, las intuiciones que se ponen en juego a la hora del trabajo con la geometría euclidiana son muy fuertes, pero a veces equivocadas. Es por eso que el buen uso de las TIC puede representar en nuestro trabajo exploratorio matemático una potente herramienta que complementa otros métodos de la matemática.
	\end{enumerate}
\end{quote}
\vspace{1em}

Lo que queremos resaltar es que las TIC no solo podrían ser usadas para ahorrar tiempo. ¡Hay mucha matemática valiosa que podría abordarse solo si contamos con tecnología! Sí, sólo si contamos con tecnología. (\textcite{rodriguez2017}).

\bigskip

A modo de cierre, siendo el objetivo general de la asignatura “Epistemología e Historia de la Matemática” realizar un análisis sobre aspectos relevantes de la epistemología de la matemática que se proyectan en la problemática didáctica con el fin de proporcionar a los-las estudiantes, futuros profesores de matemática, herramientas para la comprensión de los condicionamientos histórico-sociales del conocimiento científico y sus consecuencias en lo educativo, la elección de la modalidad “Monografía” no es caprichosa.

En la elaboración del trabajo monográfico se consulta bibliografía que debe justificarse pertinente, aparecen conceptos, demostraciones y procedimientos matemáticos que requieren fundamentación y se presentan numerosas oportunidades de discutir entre pares y con la docente acerca de las temáticas generales de la Matemática y particulares de cada uno de los temas elegidos. Se trata de una producción individual, aunque andamiada y colaborativa, con perspectiva al futuro desempeño como docentes.

Pero, así como en las materias de didáctica de la matemática el foco está puesto en la enseñanza, en esta materia el foco está puesto en la matemática pero desde una mirada humanizada y humanizante, que deje lugar a la actividad y a la creatividad. Como señalamos al principio, es fundamental que los estudiantes cuenten con la madurez necesaria para abordar cuestiones epistemológicas en que reflexionen, cuestionen y comprendan los temas que se abordan y entiendan la matemática como una ciencia activa en desarrollo que no está libre de polémicas al interior de la comunidad de matemáticos ya sean más históricas como las controversias entre las escuelas logicista, formalista y constructivista o las actuales como matemática pura o aplicada, el uso de tecnologías para la demostración, etc.

\subsection{Bibliografía de la monografía}

\nocite{*}
\printbibliography[keyword={08-monografia}]

\subsection{Bibliografía}

\nocite{*}
\printbibliography[keyword={08}]