%-------------------------------------------------------------------------
% INFORMACIÓN DEL ARTÍCULO
\thispagestyle{portadapage}
\setcounter{subsection}{0}
\setcounter{subsubsection}{0}
\setcounter{actividad}{0}
\setcounter{actividad_previa}{0}
\setcounter{actividad_entre}{0}
\renewcommand{\articulotipo}{Taller}
\renewcommand{\articulotitulo}{Autovalores, autovectores y transformaciones: una mirada geométrica}
\phantomsection
\stepcounter{section}
\addcontentsline{toc}{section}{\protect\numberline{\thesection} \articulotitulo}
\desctotoc{Chañi, M. D.; Romero, S. N.; Gutierrez, M. D.; Velasquez, N. A. M.}

\begin{center}
	\setstretch{1.5}
	{\Huge \scshape 
		\articulotitulo
	}
\end{center}

\noindent\rule{\linewidth}{2pt}

\vspace{0.25cm}

\begin{flushright}
	{\Large \scshape
		Marcos Darío Chañi
	}\\
	{\large \itshape
		Universidad Nacional de Salta
	}\\
	{\ttfamily \small
		mat.marcos2.0@gmail.com
	}\\
	\vspace{1em}
	{\Large \scshape
		Silvia Noemí Romero
	}\\
	{\large \itshape
		Universidad Nacional de Salta
	}\\
	\vspace{1em}
	{\Large \scshape
		Marcela Daniela Gutierrez
	}\\
	{\large \itshape
		Universidad Nacional de Salta
	}\\
	\vspace{1em}
	{\Large \scshape
		Noelia Adriana Melisa Velasquez
	}\\
	{\large \itshape
		Universidad Nacional de Salta
	}\\
\end{flushright}

\vspace{0.5cm}

\begin{center}
	\begin{minipage}{0.75\linewidth} \small
		\textsc{Resumen}. ~
		Dentro de la enseñanza del matemática, más precisamente el álgebra lineal uno de los tópicos fundamentales el concepto de «Transformación Lineal», y a veces por razones de tiempo no se puede profundizar en algunos temas complementarios. Es así que se propone en este taller resignificar los temas de «Transformaciones Lineales, autovalor y autovector» entre otros, para abordar un nuevo tema que son las «Transformaciones ortogonales», así mediante la ayuda del álgebra lineal se espera que los alumnos logren interpretar estos conceptos de manera gráfica y geométrica, para que no sean sólo un conjunto de conocimientos abstractos. Este trabajo ofrece una alternativa de enseñanza para los conceptos de diferentes transformaciones. El taller es un espacio para realizar gráficos dinámicos con GeoGebra, acompañado de situaciones de análisis y de debate, con la intención de realizar actividades matemáticas mediadas por un instrumento tecnológico donde el eje central es el concepto matemático, y de esta manera superar las dinámicas tradicionales. Así se propone una nueva forma de emprender las prácticas educativas contextualizas en los actuales paradigmas educativos.
	\end{minipage}\\
	
	\vspace{0.5em}
	
	\begin{minipage}{0.75\linewidth} \small
		\textsc{Palabras clave} --- GeoGebra, Álgebra, Trasformaciones Ortogonales, Matrices, Autovalor, Autovector.
	\end{minipage}
\end{center}
%-------------------------------------------------------------------------

\subsection{Introducción}

\subsubsection{Importancia del Taller}

Dentro de la enseñanza del matemática, más precisamente el álgebra lineal uno de los tópicos fundamentales el concepto de <<Transformación Lineal>>, y a veces por razones de tiempo no se puede profundizar en algunos temas complementarios. Es así que se propone en este taller resignificar los temas de «Transformaciones Lineales, autovalor y autovector» entre otros, para abordar un nuevo tema que son las <<Transformaciones ortogonales>>, así mediante la ayuda del álgebra lineal se espera que los alumnos logren interpretar estos conceptos de manera gráfica y geométrica, para que no sean sólo un conjunto de conocimientos abstractos. Este trabajo ofrece una alternativa de enseñanza para los conceptos de diferentes transformaciones. El taller es un espacio para realizar gráficos dinámicos con GeoGebra, acompañado de situaciones de análisis y de debate, con la intención de realizar actividades matemáticas mediadas por un instrumento tecnológico donde el eje central es el concepto matemático, y de esta manera superar las dinámicas tradicionales. Así se propone una nueva forma de emprender las prácticas educativas contextualizas en los actuales paradigmas educativos.

\subsubsection{Fundamentos}

En relación al marco pedagógico se va a trabajar a través del aprendizaje secuencial y colaborativo, mediante un “conocimiento espiralado”, es decir el alumno construye y/o afirma conceptos por medio de conceptos ya aprendidos, de manera que realiza un reordenamiento de sus ideas y los confirma. Es por ello que se propone un taller, ya que las actividades pueden realizarse en forma individual o colectiva, la evaluación de las actividades es realizada por los integrantes, de esta manera se ubica al alumno para que sea crítico tanto con su trabajo como el de sus pares. 

Se elaborará una guía para los 3 encuentros, en las cuales se proponen actividades que se realizarán en forma grupal o individual en forma alternada. Las actividades se llevarán a cabo en tres instancias las cuales habrá:
\begin{itemize}
	\item Un desarrollo matemático en función a lo que se pide en las guías.
	\item Un trabajo en la máquina mediante el uso del Software GeoGebra.
	\item Una puesta en común para debatir sobre lo realizado en las instancias anteriores.
\end{itemize}

Los nuevos escenarios educativos brindan la posibilidad de incorporar recursos vinculados con las TIC, así también cada día aumenta la
demanda de la formación continua en relación a las nuevas corrientes tecnológicas-matemáticas por parte de los docentes, esto conlleva
a generar espacios de aprendizaje en donde puedan elaborar y llevar a cabo propuestas educativas que involucren las TIC en la enseñanza de la matemática. 

El presente escrito describe un taller de actividades mediante el uso del software GeoGebra, con la intención de contribuir a la formación profesional continua en el campo de la matemática y su enseñanza. La elección del software GeoGebra, se debe no solo a sus potencialidades matemáticas y didácticas, sino también a que es de uso libre.

Hoy en día el desafío de enseñar matemática implica que los docentes no solo superen algunas tradiciones que provienen de nuestras biografías escolares y formativas, sino también el replantear el uso de las tecnologías en el proceso de enseñanza y aprendizaje. A partir de esto se realiza una propuesta para estudiantes avanzados y docentes de matemática, con la intención de realizar actividades matemáticas mediadas por un instrumento tecnológico con eje central el concepto matemático, y de esta manera superar las dinámicas tradicionales.

Uno de los aspectos del taller es la visualización, en relación a ella distintos investigadores \textcite{castro1997,cantoral2002} acuerdan que al realizar una actividad de visualización se requiere la utilización de nociones y conceptos matemáticos para poder ser interpretada de forma adecuada. Es decir, la capacidad de visualizar un concepto matemático o un problema requiere de la habilidad para interpretar y entender la información de tal, manipularlo mentalmente y así expresarlo en un soporte material. Así también cuando se usa la representación gráfica de un concepto matemático como herramienta para interpretar otros o resolver problemas, la visualización no es el fin en si mismo sino un medio para llegar a la compresión de propiedades y de relaciones entre distintos conceptos. Según \textcite[103]{castro1997}, dominar un concepto matemático consiste en conocer sus principales representaciones y el significado de cada una de ellas, así como operar con las reglas internas de cada sistema y en convertir o traducir unas representaciones en otras, detectando qué sistema es más ventajoso para trabajar cono determinadas propiedades. 

Debe remarcarse que GeoGebra permite otros modos de hacer en la clase ya que los participantes pueden enfocar su atención en procesos de análisis, que permiten la modelización, la toma de decisiones, el razonamiento y la resolución de situaciones, tal como lo plantean \textcite{arcavi2000} (citado en Santos Trigo, 2007, p. 39), los ambientes dinámicos no sólo permiten a los estudiantes construir figuras con ciertas propiedades y visualizarlas, sino que también les permite transformar esas construcciones en tiempo real. Este dinamismo puede contribuir en la formación de hábitos para transformar (mentalmente o por medio de una herramienta) una instancia particular, para estudiar variaciones, invariantes visuales y posiblemente proveer bases intuitivas para justificaciones formales de conjeturas y proposiciones.

\subsection{Contenidos}

Interpretación geométrica de transformaciones lineales, autovalor y autovector de una transformación lineal. Transformaciones ortogonales. 

\subsection{Requisitos Previos}

Conceptos básicos de un curso de álgebra lineal. Especialmente Transformaciones lineales. Autovalor y autovector.

\subsection{Objetivos}
\begin{itemize}
	\item Valorizar el concepto de transformación lineal, autovalor y autovector a través de su interpretación geométrica.
	\item Facilitar el aprendizaje de las diferentes transformaciones ortogonales, a través de la visualización gráfica. mediante el Software GeoGebra.
	\item Realizar representaciones gráficas dinámicas de los conceptos descritos anteriormente.
	\item Generar en los participantes nuevas ideas de cómo utilizar los recursos vistos, en su práctica docente.
\end{itemize}

\subsection{Actividades}

\subsubsection{Actividades previas}

Lectura del material bibliográfico sobre el tema <<Transformaciones lineales, autovalores y autovectores>> sumado a los ejercicios de identificar si una transformación es lineal o no, y sobre autovalores y autovectores. Se espera que mediante las actividades los participantes refresquen dichos contenidos que fueron impartidos en un curso de álgebra lineal.

\subsubsection{Primera hora y media presenciales}\label{subsec:Primeras-dos-sinc-03}

Los cursantes del taller a partir de la <<Definición de Transformación Ortogonal>>, mediante razonamientos lógicos y deductivos deberán encontrar las condiciones para ser una transformación ortogonal y de esta manera hallar las dos maneras de expresar una transformación ortogonal en forma matricial, a las que llamaremos de primera y segunda especie. Cuando los cursantes lo requieran contarán con la guía de los docentes a cargo para poder evacuar dudas o confirmar si sus razonamientos son correctos y poder validarlos.

\subsubsection{Primeras dos horas entre clases}\label{subsec:Primeras-dos-EC-03}

En este espacio se dará material tanto teórico como práctico, para el uso de GeoGebra. En relación al material teórico se propone un resumen de cómo introducir matrices en el software GeoGebra, para luego pasar a las actividades donde deberán introducir las diferentes matrices que se proponen. Los archivos se guardarán, para en forma posterior ser enviados mediante la plataforma para su control por parte de los responsables.

\subsubsection{Segundas dos horas presenciales}

En las segunda dos horas presenciales, los cursantes del taller deberán deducir la expresión matricial de diferentes transformaciones rígidas en $\mathbb{R}^{3}$ es decir en el plano (traslación, simetría central, simetría axial, etc.) con el sustento de un un marco gráfico, para ello deberán hacer uso del concepto de autovalor y autovector de una transformación lineal. Dichas actividades estarán guiadas mediante preguntas y/o observaciones de manera que los participantes del taller puedan razonar y debatir ya sea de manera individual o grupal. En todo el proceso de deducción y discusión los docentes responsables podrán guiar a los participantes para que logren hallar las expresiones matriciales de cada transformación rígida.

\subsubsection{Segundas dos horas entre clases}

Este espacio estará dedicado para que los participantes puedan validar mediante el Software GeoGebra que las transformaciones halladas son correctas o no. Para ello se les pedirá en GeoGebra en la ventada 2D:
\begin{itemize}
	\item Construir una polígono (por ejemplo un triángulo).
	\item Ingresar la matriz de la transformación correspondiente (se aclara que en las primeras horas entre clases, sección \ref{subsec:Primeras-dos-EC-03}, los participantes ya aprendieron a cargar matrices en GeoGebra).
	\item Encontrar mediante la matriz ingresada, los transformados de los vértices de la figura y construir en nuevo polígono con los puntos resultantes de aplicar dicha transformación.
	\item Comprobar, validar o refutar que dicha es la correcta.
\end{itemize}

\subsubsection{Terceras dos horas presenciales}

En las terceras y últimas dos horas presenciales, los cursantes nuevamente deberán deducir la expresión matricial de las transformaciones estudiadas en las dos horas presenciales anteriores, sección \ref{subsec:Primeras-dos-sinc-03}, pero en este caso para el espacio $\mathbb{R}^{3}$. Incentivando el poder de generalización de los participantes. Dichas actividades nuevamente estarán guiadas mediante preguntas y/o observaciones de manera que los participantes puedan generalizar el método encontrado en la clase anterior. Así también en todo el proceso de deducción y discusión los docentes responsables podrán sugerir a los participantes pensar en cómo generalizar las expresiones matriciales de cada transformación rígida.

\subsubsection{Evaluación final}

La evaluación será mediante la entrega de archivos de GeoGebra, para ello a participantes se les dará una transformación en $\mathbb{R}^{3}$ a realizar, la cual deberán realizar la matriz de transformación y luego ingresarla en el Software. En forma posterior se le pedirá que verifiquen que una figura (polígono o poliedro) se transforma de manera correcta mediante el uso de la matriz ingresada.

\subsection{Bibliografía}

\nocite{*}
\printbibliography[keyword={03}]