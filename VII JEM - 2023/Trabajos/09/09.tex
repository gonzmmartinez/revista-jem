%-------------------------------------------------------------------------
% INFORMACIÓN DEL ARTÍCULO
\thispagestyle{portadapage}
\setcounter{subsection}{0}
\setcounter{subsubsection}{0}
\setcounter{actividad}{0}
\setcounter{actividad_previa}{0}
\setcounter{actividad_entre}{0}
\renewcommand{\articulotipo}{Comunicación breve}
\renewcommand{\articulotitulo}{La teoría de conjuntos y su enseñanza}
\phantomsection
\stepcounter{section}
\addcontentsline{toc}{section}{\protect\numberline{\thesection} \articulotitulo}
\desctotoc{Sángari, A. N.; Coria Saravia, S. I.}

\begin{center}
	\setstretch{1.5}
	{\Huge \scshape 
		\articulotitulo
	}
\end{center}

\noindent\rule{\linewidth}{2pt}

\vspace{0.25cm}

\begin{flushright}
	{\Large \scshape
		Antonio Noé Sángari
	}\\
	{\large \itshape
		Universidad Nacional de Salta
	}\\
	{\ttfamily \small
		jem@exa.unsa.edu.ar
	}\\ \vspace{1em}
	{\Large \scshape
		Silvia Isabel Coria Saravia
	}\\
	{\large \itshape
		Universidad Nacional del Comahue
	}\\
\end{flushright}

\vspace{0.5cm}

\begin{center}
	\begin{minipage}{0.75\linewidth} \small
		\textsc{Resumen}. ~
		La teoría de conjuntos es un pilar fundamental de las matemáticas y su enseñanza requiere una comprensión de las bases matemáticas y estrategias didácticas adecuadas. Los primeros axiomas, como el de extensionalidad, el del conjunto vacío, el de especificación, el de par, el de unión y el del conjunto potencia, establecen las reglas fundamentales para la construcción y manipulación de conjuntos. Expresar estos axiomas en lenguaje simbólico proporciona claridad y precisión, y su traducción al lenguaje coloquial facilita la comprensión por parte de los estudiantes. A partir de estos axiomas, se pueden obtener deducciones y resultados importantes, como el concepto de función. La enseñanza de la teoría de conjuntos requiere considerar tanto los aspectos matemáticos como los didácticos, adaptando el contenido a las necesidades de los estudiantes. En el contexto del sistema educativo argentino, la teoría de conjuntos tiene una importancia destacada y se incluye en la formación docente de matemática. El objetivo de este trabajo final es desarrollar herramientas prácticas para facilitar la enseñanza y el aprendizaje de la teoría de conjuntos, contribuyendo así a mejorar la calidad de la educación matemática en el aula.
	\end{minipage}
\end{center}
%-------------------------------------------------------------------------

\subsection{Introducción}

\subsubsection{Generalidades}

a teoría de conjuntos es un pilar fundamental de las matemáticas, que proporciona un marco conceptual sólido para comprender y analizar las relaciones entre los objetos matemáticos. Desde su desarrollo inicial a principios del siglo XX, la teoría de conjuntos ha evolucionado y se ha convertido en una disciplina central en el currículo de matemáticas, tanto a nivel teórico como aplicado.

La enseñanza de la teoría de conjuntos no solo implica transmitir conocimientos sobre sus fundamentos y técnicas, sino también abordar aspectos didácticos que faciliten su comprensión y aplicación por parte de los estudiantes. Es esencial que los educadores comprendan tanto las bases matemáticas como las estrategias pedagógicas para presentar la teoría de conjuntos de manera accesible y significativa.

Desde una perspectiva matemática, la teoría de conjuntos se basa en un conjunto de axiomas que establecen las reglas fundamentales para la construcción y manipulación de conjuntos. Estos axiomas, como los formulados en la teoría de conjuntos de Zermelo-Fraenkel (ZFC), proporcionan una base coherente y consistente para desarrollar argumentos lógicos y razonamientos rigurosos en el contexto de conjuntos. Comprender estas bases matemáticas es crucial para los educadores, ya que les permite transmitir los conceptos esenciales de la teoría de conjuntos de manera precisa y clara.

Sin embargo, la enseñanza de la teoría de conjuntos va más allá de la mera exposición de los conceptos matemáticos. Es importante tener en cuenta las consideraciones didácticas para garantizar un aprendizaje efectivo. Esto implica seleccionar estrategias de enseñanza apropiadas, diseñar actividades y ejemplos que fomenten la comprensión activa y reflexiva, y adaptar el contenido a las necesidades y nivel de desarrollo de los estudiantes.

\subsubsection{Importancia en el sistema educativo}

En el contexto del sistema educativo argentino, la importancia de la teoría de conjuntos se ve resaltada, sobre todo en la formación docente de matemática. El currículo educativo en Argentina insiste en la inclusión de estos temas, reconociendo su relevancia para el desarrollo del pensamiento lógico y la comprensión de conceptos matemáticos fundamentales.

\subsubsection{Propósito de este trabajo}

Este artículo representa un avance de un traba jo final para completar la carrera de profesorado de matemática, donde se busca abordar el desafío de presentar el material específico de teoría de conjuntos de manera efectiva en el salón de clases. El objetivo principal de este proyecto es desarrollar un texto que faciliten la enseñanza y el aprendizaje de la teoría de conjuntos, adaptándola a las necesidades y nivel de desarrollo de los estudiantes. El objetivo final es contribuir al cuerpo de conocimientos y ofrecer herramientas prácticas a los educadores para abordar la enseñanza de la teoría de conjuntos de manera más efectiva y significativa.

\subsection{Requisitos previos}

Conocimientos básicos de matemáticas, incluyendo operaciones aritméticas,álgebra elemental, manipulación de expresiones simbólicas y resolución de ecuaciones simples. Entendimiento básico de la lógica proposicional y el razonamiento deductivo. Conceptos básicos de conjuntos, como pertenencia, inclusión, intersección, unión y diferencia. Conceptos de demostración matemática: la comprensión de conceptos como demostraciones directas, demostraciones por contradicción y demostraciones por casos.

\subsection{Desarrollo}

El trabajo final sobre la enseñanza de la teoría de conjuntos consta de tres partes principales que abordan diferentes aspectos de esta disciplina matemática. A continuación, se describen brevemente cada una de estas partes:

Historia de la teoría de conjuntos: En esta parte, se realiza un recorrido histórico de la teoría de conjuntos, destacando los hitos y desarrollos importantes a lo largo del tiempo. Se pueden abordar temas como los primeros conceptos de conjunto, las paradojas que surgieron en el siglo XIX, la formulación de los primeros axiomas por Ernst Zermelo y Abraham Fraenkel, y los avances posteriores en la teoría de conjuntos en el siglo XX y XXI. Explorar la historia de la teoría de conjuntos proporciona un contexto y una perspectiva más amplia para comprender su evolución y su importancia en las matemáticas.

Primeros axiomas de la teoría de conjuntos: Esta parte se enfoca en los primeros axiomas fundamentales de la teoría de conjuntos, como el axioma del conjunto vacío, el axioma de extensión, el axioma de especificación, el axioma de par y el axioma de unión. Se explican en detalle cada uno de estos axiomas, su significado y sus implicaciones. Además, se puede abordar la importancia de estos axiomas en la construcción y manipulación de conjuntos, así como su papel en la formulación de otros resultados y teoremas en la teoría de conjuntos.

Axiomas del infinito y de sustitución: En esta parte, se exploran los axiomas del infinito y de sustitución, que son dos axiomas adicionales que se añaden a los primeros axiomas básicos. El axioma del infinito establece la existencia de un conjunto infinito, mientras que el axioma de sustitución permite construir conjuntos a través de funciones aplicadas a conjuntos existentes. Estos axiomas amplían la capacidad de la teoría de conjuntos y permiten abordar conceptos y resultados más avanzados, como la construcción de números cardinales y ordinales, entre otros.

Cada una de estas partes del trabajo final aborda aspectos clave de la teoría de conjuntos y su enseñanza, proporcionando una comprensión más completa de los fundamentos teóricos y su aplicación pedagógica. Al explorar la historia, los primeros axiomas y los axiomas adicionales, se logra una visión panorámica de esta disciplina matemática y se sienta una base sólida para la enseñanza efectiva de la teoría de conjuntos en el aula.

A modo de ejemplo comentaremos brevemente dos de esas partes.

\subsubsection{Breve recorrido histórico}

Comenzaremos con un breve recorrido histórico de la teoría de conjuntos, destacando los problemas que tuvieron que resolver a lo largo de su desarrollo para concientizar a los lectores sobre la necesidad de la formalización. Ver por ejemplo \textcite{boyer1999historia}.

\begin{description}
	\item[Siglo XIX] El concepto intuitivo de conjunto ha estado presente en las matemáticas desde tiempos antiguos, pero fue en el siglo XIX cuando la teoría de conjuntos comenzó a tomar forma. Augustin-Louis Cauchy y Georg Cantor, entre otros matemáticos, realizaron contribuciones importantes al estudio de los conjuntos y las colecciones de objetos matemáticos.
	
	\item[Finales del siglo XIX] A medida que la teoría de conjuntos se desarrollaba, surgieron paradojas y contradicciones. El problema más famoso fue la Paradoja de Russell, formulada por Bertrand Russell en 1901. Esta paradoja señaló una aparente contradicción en el conjunto de todos los conjuntos que no se contienen a sí mismos. Estas paradojas plantearon la necesidad de establecer fundamentos lógicos más rigurosos para la teoría de conjuntos.
	
	\item[Primeras décadas del siglo XX] Ernst Zermelo y Abraham Fraenkel fueron dos matemáticos clave en la formulación de un sistema axiomático sólido para la teoría de conjuntos. En 1908, Zermelo propuso los primeros axiomas para establecer una base coherente para los conjuntos. Luego, en la década de 1920, Fraenkel y Zermelo desarrollaron conjuntamente el sistema de axiomas Zermelo-Fraenkel (ZFC), que se convirtió en el marco estándar de la teoría de conjuntos.
	
	\item[Mediados del siglo XX] Durante este período, la teoría de conjuntos se convirtió en una disciplina central en las matemáticas y se utilizaron sus conceptos y técnicas para desarrollar numerosas ramas de la disciplina. Sin embargo, también surgieron problemas adicionales. Uno de los desafíos más importantes fue el llamado “problema de la cardinalidad del continuo” propuesto por Cantor, que trata de determinar si hay algún conjunto que tenga una cardinalidad estrictamente mayor que los números naturales pero menor que los números reales.
	
	\item[Siglo XXI] La teoría de conjuntos sigue siendo un área activa de investigación matemática. Además de los problemas clásicos, han surgido nuevos desafíos, como los estudios sobre conjuntos grandes y estructuras axiomáticas alternativas, como la teoría de conjuntos constructivista y la teoría de conjuntos no estándar.
\end{description}

\subsubsection{Primeros axiomas}

En nuestro traba jo suponemos que los lectores están familiarizados con la lógica matemática de primer orden pero si no es el caso tenemos un apéndice basado en el texto de \textcite{margaris1990first}. El texto base es \textcite{cignoli2016teoriaconj}

Expresar los axiomas en lenguaje simbólico tiene varias ventajas y proporciona claridad, precisión y generalidad en el estudio de la teoría de conjuntos. A continuación, se explica la importancia de expresar los axiomas en lenguaje simbólico, su traducción al lenguaje coloquial y la ejemplificación, así como las deducciones que se pueden obtener a partir de estos primeros axiomas:
\begin{description}
	\item[Claridad y precisión] El lenguaje simbólico utilizado en los axiomas permite una expresión precisa de las ideas matemáticas. Al utilizar símbolos y notaciones formales, se evita la ambigüedad y se logra una mayor claridad en las definiciones y enunciados matemáticos.
	
	\item[Traducción al lenguaje coloquial] Para facilitar la comprensión, los axiomas se pueden traducir al lenguaje coloquial, es decir, al lenguaje cotidiano utilizado en la comunicación. Esto ayuda a relacionar los conceptos matemáticos con situaciones más familiares, lo que puede facilitar su comprensión para aquellos que no están familiarizados con el lenguaje simbólico.
	
	\item [Ejemplificación] Para ilustrar los axiomas, se pueden proporcionar ejemplos concretos que muestren cómo se aplican los conceptos y las reglas establecidas por los axiomas. Estos ejemplos pueden ayudar a los estudiantes a visualizar y comprender mejor los conceptos abstractos presentados en los axiomas.
	
	\item[Generalidad y aplicabilidad] Al expresar los axiomas en lenguaje simbólico, se logra una mayor generalidad y aplicabilidad. Los axiomas no están limitados a situaciones específicas, sino que establecen principios generales que se pueden aplicar en diversos contextos matemáticos. Esto permite una mayor flexibilidad y extensión del conocimiento matemático.
	
	\item[Deducciones] A partir de los primeros axiomas, se pueden realizar deducciones y demostraciones para obtener nuevos resultados matemáticos. Utilizando reglas lógicas y los axiomas como premisas, se pueden construir argumentos que conducen a conclusiones válidas. Estas deducciones permiten desarrollar teoremas y resultados más complejos basados en los axiomas iniciales.
\end{description}

Listaremos  los  primeros  axiomas  dando  una  explicación  breve  de  su  sentido. 

\paragraph*{Axioma de extensionalidad}\vspace{-1em}
\begin{equation*}
	\forall x\forall y(\forall t(t\in x\leftrightarrow t\in y)\rightarrow x=y)
\end{equation*}

Este axioma es fundamental en la teoría de conjuntos, ya que garantiza que las propiedades de igualdad y equivalencia se mantengan en el contexto de conjuntos. Al afirmar que dos conjuntos son iguales si y solo si tienen los mismos elementos, el axioma de extensionalidad establece una noción de igualdad entre conjuntos basada en la identidad de sus elementos. 

\paragraph*{Axioma del vacío}\vspace{-1em}
\begin{equation*}
	\exists y\forall x\sim\left(x\in y\right)
\end{equation*}

El axioma del vacío es uno de los axiomas básicos en la teoría de conjuntos y establece la existencia de un conjunto que no contiene elementos. Este conjunto vacío juega un papel importante en el desarrollo de la teoría de conjuntos, ya que sirve como punto de partida para la construcción de otros conjuntos y operaciones con conjuntos. Además, el axioma del vacío proporciona una base sólida para la formulación de otros axiomas y principios en la teoría de conjuntos.

Con estos dos axiomas obtenemos algunas conclusiones en base a la deducción. En particular mostramos que \textbf{el vacío es único}. También agregamos ejemplos explicativos.

\paragraph*{Axioma (esquema) de especificacion}

Si $\varphi$ es una fórmula cualquiera de la teoría de conjuntos, con la variable libre $t$, entonces el siguiente enunciado es un axioma:
\begin{equation*}
	\forall x\exists y\left(\forall t\left(t\in x\land\varphi\left(t\right)\right)\leftrightarrow t\in y\right)
\end{equation*}

El axioma de especificación permite definir subconjuntos basados en una condición o propiedad determinada. Es esencial en la construcción de conjuntos más complejos y en la formulación de teoremas y resultados dentro de la teoría de conjuntos. Proporciona una herramienta poderosa para la construcción de conjuntos más especializados y restringidos a través de la especificación de condiciones específicas.

\paragraph*{Axioma de par}\vspace{-1em}
\begin{equation*}
	\forall x\forall y\exists z(\forall t\left(t=x\lor t=y\right)\rightarrow t\in z)
\end{equation*}

En otras palabras, el axioma de par asegura la existencia de un conjunto que contiene los elementos “$x$” e “$y$”.

Este axioma es fundamental en la teoría de conjuntos, ya que proporciona una base para la construcción de conjuntos más complejos y para el desarrollo de operaciones y estructuras más avanzadas en la teoría de conjuntos.

En este punto, ya nos es relativamente fácil mostrar que \textbf{existen} conjuntos con una \textbf{cantidad finita cualquiera de elementos.}

\paragraph*{Axioma de unión}\vspace{-1em}

\begin{equation*}
	\forall z\exists x\left(\forall t\left(\exists y\left(y\in z\land t\in y\right)\right)\rightarrow t\in x\right)
\end{equation*}

El axioma de unión permite construir un conjunto que contiene todos los elementos de los conjuntos que forman parte de otro conjunto. Este axioma es fundamental en la teoría de conjuntos, ya que facilita la formación de conjuntos más grandes y la combinación de elementos de conjuntos más pequeños. Además, el axioma de unión se utiliza en numerosas demostraciones y construcciones en el campo de la teoría de conjuntos.

\paragraph*{Axioma del conjunto potencia}\vspace{-1em}
\begin{equation*}
	\forall x\exists y\left(\forall t\left(t\subseteq x\rightarrow t\in y\right)\right)
\end{equation*}

El axioma del conjunto potencia es fundamental en la teoría de conjuntos, ya que proporciona una herramienta poderosa para construir conjuntos más complejos y analizar la estructura de los conjuntos. Además, el conjunto potencia tiene aplicaciones en diversos campos de las matemáticas, como la teoría de conjuntos, la teoría de gráficos, la topología y la teoría de la medida.

A partir del axioma del conjunto potencia, se pueden obtener varias deducciones y resultados. Por ejemplo, se puede demostrar que el conjunto vacío es un subconjunto de cualquier conjunto dado, y que el conjunto potencia de un conjunto finito con $n$ elementos tiene $2^{n}$ elementos. Además, el axioma del conjunto potencia se utiliza en la formulación de otros axiomas y principios en la teoría de conjuntos, permitiendo construir estructuras más complejas y desarrollar resultados más avanzados.

\bigskip

Con los primeros axiomas de la teoría de conjuntos, incluyendo el axioma del conjunto potencia, el axioma de unión, el axioma de par, el axioma de especificación y otros axiomas que se pueden derivar de ellos, es posible llegar a resultados muy importantes, incluyendo el concepto fundamental de función.

La noción de función es esencial en las matemáticas y juega un papel fundamental en diversas ramas de la disciplina. Una función es una relación que asigna a cada elemento de un conjunto, llamado dominio, un único elemento de otro conjunto, llamado codominio. Es una herramienta poderosa para describir y analizar las relaciones entre diferentes conjuntos.

Usando los axiomas mencionados, podemos construir y estudiar funciones de manera rigurosa. Por ejemplo, podemos definir una función f que toma un elemento x del dominio y lo asigna a un único elemento y del codominio. Podemos establecer propiedades como la inyectividad (cada elemento del dominio se asigna a un único elemento del codominio), la sobreyectividad (cada elemento del codominio tiene al menos un elemento del dominio que lo asigna) y la biyectividad (una función que es al mismo tiempo inyectiva y sobreyectiva).

Además, los axiomas permiten estudiar las propiedades de las funciones, como la composición de funciones, las funciones inversas y la imagen directa e inversa de conjuntos bajo una función. Estos conceptos son fundamentales en el análisis, el álgebra, la geometría y muchas otras áreas de las matemáticas.

Es importante destacar que los axiomas iniciales proporcionan la base para la construcción de conceptos más avanzados en la teoría de conjuntos y en matemáticas en general. A partir de estos axiomas, se pueden desarrollar teoremas y resultados más sofisticados, que permiten profundizar en el estudio de funciones y su aplicabilidad en diversos contextos matemáticos.

\subsection{Bibliografía}

% \nocite{*}
\printbibliography[keyword={09}]