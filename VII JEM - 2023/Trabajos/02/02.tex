%-------------------------------------------------------------------------
% INFORMACIÓN DEL ARTÍCULO
\thispagestyle{portadapage}
\setcounter{subsection}{0}
\setcounter{subsubsection}{0}
\setcounter{actividad}{0}
\setcounter{actividad_previa}{0}
\setcounter{actividad_entre}{0}
\renewcommand{\articulotipo}{Taller}
\renewcommand{\articulotitulo}{Datos educativos: producción y uso de herramientas de procesamiento de información con asistencia de la IA para la toma de decisiones}
\phantomsection
\stepcounter{section}
\addcontentsline{toc}{section}{\protect\numberline{\thesection} \articulotitulo}
\desctotoc{Villagra, C. E.; Miguez, I. H; Acosta, R. N.; Chorolque, E. M.}

\begin{center}
	\setstretch{1.5}
	{\Huge \scshape 
		\articulotitulo
	}
\end{center}

\noindent\rule{\linewidth}{2pt}

\vspace{0.25cm}

\begin{flushright}
	{\Large \scshape
		Fernando Jorge Bifano
	}\\
	{\large \itshape
		Facultad de Ciencias Exactas y Naturales — Universidad de Buenos Aires
	}\\
	{\ttfamily \small
		fjbifano@ccpems.exactas.uba.ar
	}\\
	\vspace{1em}
	{\Large \scshape
		Pablo Fabián Carranza
	}\\
	{\large \itshape
		Universidad Nacional de Río Negro
	}\\
	\vspace{1em}
\end{flushright}

\vspace{0.5cm}

\begin{center}
	\begin{minipage}{0.75\linewidth} \small
		\textsc{Resumen}. ~
		En general los estudiantes del nivel secundario y los ingresantes al nivel superior  muestran que tienen dificultades al resolver situaciones vinculadas a la proporcionalidad.  Para lograr que haya una buena comprensión conceptual de la proporcionalidad es fundamental desarrollar el pensamiento proporcional cualitativo y cuantitativo de los estudiantes desde el nivel primario de escolaridad. De esta manera podrán  desempeñarse con fluidez en su vida cotidiana, pero también podrán construir conceptos más complejos en niveles educativos superiores , como por ejemplo  las variaciones, la función lineal, las razones de cambios, las derivadas. Por eso en este taller se  generará un espacio de reflexión sobre la enseñanza y aprendizaje de la proporcionalidad, articulando el nivel primario y secundario, poniendo en relevancia los diferentes significados de la proporcionalidad: aritmético, proto-algebraico y algebraico-funcional, haciendo uso de los distintos registros semióticos de representación e incluyendo la TIC. El taller estará destinado  a docentes de 6to. y 7mo. año del nivel primario y ciclo básico del nivel secundario y promoverá el uso del software dinámico GeoGebra que posibilita la coordinación de los diferentes registros  y de esta manera se convierte en una potente herramienta para propiciar la significación de conceptos, en este caso de la proporcionalidad que funciona en diferentes registros de representación semiótica.
	\end{minipage}
	
	\vspace{0.5em}
	
	\begin{minipage}{0.75\linewidth} \small
		\textsc{Palabras clave} --- Datos educativos, IA, Python, Procesamiento de la información.
	\end{minipage}
\end{center}
%-------------------------------------------------------------------------

\subsection{Introducción}



\subsection{Bibliografía}

\nocite{*}
\printbibliography[keyword={02}]