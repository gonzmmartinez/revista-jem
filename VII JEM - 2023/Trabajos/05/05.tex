%-------------------------------------------------------------------------
% INFORMACIÓN DEL ARTÍCULO
\thispagestyle{portadapage}
\setcounter{subsection}{0}
\setcounter{subsubsection}{0}
\setcounter{actividad}{0}
\setcounter{actividad_previa}{0}
\setcounter{actividad_entre}{0}
\renewcommand{\articulotipo}{Taller}
\renewcommand{\articulotitulo}{Geometría: algunos conceptos relacionados con Olimpiada Matemática Argentina}
\phantomsection
\stepcounter{section}
\addcontentsline{toc}{section}{\protect\numberline{\thesection} \articulotitulo}
\desctotoc{Sángari, A. N.; Flores Rocha, V.; Coria, S. E.; Ghiglia, N.}

\begin{center}
	\setstretch{1.5}
	{\Huge \scshape 
		\articulotitulo
	}
\end{center}

\noindent\rule{\linewidth}{2pt}

\vspace{0.25cm}

\begin{flushright}
	{\Large \scshape
		Antonio Noé Sángari
	}\\
	{\large \itshape
		Universidad Nacional de Salta
	}\\
	{\ttfamily \small
		jem@exa.unsa.edu.ar
	}\\
	\vspace{1em}
	{\Large \scshape
		Veronica Flores Rocha
	}\\
	{\large \itshape
		Colegio Italo-argentino Bilingüe Bicultural “Dante Alighieri”
	}\\
	\vspace{1em}
	{\Large \scshape
		Silvia Ester Coria
	}\\
	{\large \itshape
		Instituto de Educación Media “Dr. Arturo Oñativia”
	}\\
	\vspace{1em}
	{\Large \scshape
		Nadia Ghiglia
	}\\
\end{flushright}

\vspace{0.5cm}

\begin{center}
	\begin{minipage}{0.75\linewidth} \small
		\textsc{Resumen}. ~
		Este trabajo se enfoca en la importancia de un taller para docentes de educación secundaria en Argentina que participan en la Olimpiada Matemática Argentina. El taller tiene como objetivo proporcionarles herramientas y enfoques novedosos para guiar a sus estudiantes en la resolución de problemas geométricos, fortaleciendo así los conceptos geométricos y mejorando la calidad educativa. Se basa en fundamentos matemáticos, didácticos y pedagógicos, que incluyen una comprensión profunda de la geometría, la resolución de problemas desafiantes y un enfoque pedagógico centrado en el estudiante. Los temas abordados en el taller incluyen figuras geométricas básicas, cuadriláteros convexos, la circunferencia, la semejanza y las razones trigonométricas. La metodología se basa en plantear situaciones problemáticas para promover el aprendizaje activo y el razonamiento. Las actividades previas incluyen recopilar material bibliográfico y realizar un cuadro comparativo de los conceptos y dificultad de los textos. Durante el taller, se presentan recursos multimediales y material bibliográfico relacionado con la Olimpiada Matemática. Los participantes comparten sus cuadros comparativos y se fomenta la discusión. En resumen, este taller busca fortalecer la capacidad de los docentes para guiar a sus estudiantes en la resolución de problemas geométricos y promover una educación de calidad en matemáticas.
	\end{minipage}\\
	
	\vspace{0.5em}
	
	\begin{minipage}{0.75\linewidth} \small
		\textsc{Palabras clave} --- Taller, Docentes, Geometría, Olimpiada Matemática, Resolución de problemas.
	\end{minipage}
\end{center}
%-------------------------------------------------------------------------

\subsection{Introducción}

\subsubsection{Importancia del taller}

Cada año, un gran número de estudiantes de diversas instituciones educativas de Argentina participan en la Olimpiada Matemática Argentina, la cual se lleva a cabo en todo el territorio del país. Esta competencia tiene como objetivo abordar contenidos matemáticos a través de problemas desafiantes, cuya resolución requiere el adecuado manejo de conceptos básicos entrelazados entre sí. 

Los docentes de educación secundaria desempeñan un papel fundamental en la construcción de estas perspectivas diferentes y diversas. Su participación en este taller les brindará herramientas y enfoques novedosos para colaborar con sus estudiantes en la resolución de problemas geométricos, con el fin de mejorar la calidad educativa y fortalecer los conceptos geométricos en sus alumnos.

Durante el taller, los docentes tendrán la oportunidad de adquirir miradas enriquecedoras que les permitirán abordar la enseñanza de la geometría desde diferentes ángulos. Además, se les proporcionarán estrategias de resolución de problemas, lo que les permitirá guiar a sus estudiantes de manera efectiva en el desarrollo de habilidades matemáticas y en la comprensión de conceptos geométricos más profundos.

Al promover la participación de los docentes en este taller, se busca fomentar una educación de calidad y un mayor afianzamiento de los conocimientos geométricos en los estudiantes. La resolución de problemas de geometría no solo implica la aplicación de conceptos teóricos, sino también el desarrollo de habilidades analíticas y estratégicas, lo cual contribuye al crecimiento integral de los alumnos.

En resumen, la participación de los docentes de secundaria en este taller les proporcionará herramientas valiosas y diferentes perspectivas para colaborar con sus estudiantes en la resolución de problemas geométricos. Esto no solo mejorará la calidad educativa, sino que también fortalecerá los conceptos geométricos en el alumnado, brindándoles una base sólida para su desarrollo académico y personal.

\subsubsection{Fundamentos}

Las Olimpíadas Matemáticas, incluyendo la Olimpíada Matemática Argentina, se basan en fundamentos matemáticos, didácticos y pedagógicos para promover el aprendizaje y la excelencia en matemáticas, especialmente en el campo de la geometría. Estas bases son las siguientes:

\paragraph{Fundamentos Matemáticos}

Las Olimpíadas Matemáticas, incluyendo la geometría, se basan en una sólida comprensión de los conceptos matemáticos. Los participantes deben tener un conocimiento profundo de la geometría euclidiana, incluyendo teoremas, axiomas, propiedades y demostraciones. También es importante tener habilidades en cálculo y álgebra, ya que la geometría a menudo se relaciona con estos campos.

\paragraph{Fundamentos Didácticos}

Las Olimpíadas Matemáticas se centran en la resolución de problemas como medio para promover el aprendizaje matemático. La resolución de problemas desafiantes estimula el razonamiento lógico, la creatividad y el pensamiento crítico. Los problemas geométricos presentados en las olimpíadas suelen requerir una comprensión profunda de los conceptos geométricos y la capacidad de aplicarlos en situaciones no triviales. Además, estos problemas suelen fomentar la visualización, la manipulación de figuras y la deducción lógica.

\paragraph{Fundamentos Pedagógicos}

Las Olimpíadas Matemáticas promueven un enfoque pedagógico centrado en el estudiante, en el cual se fomenta el descubrimiento, la exploración y la construcción activa del conocimiento. Se busca desarrollar habilidades y competencias matemáticas más allá del currículo tradicional, fomentando el interés y la pasión por las matemáticas. Los problemas geométricos de las olimpíadas proporcionan a los estudiantes desafíos estimulantes que les permiten profundizar en su comprensión de la geometría y desarrollar habilidades de resolución de problemas.

Además de estos fundamentos, las Olimpíadas Matemáticas también promueven la colaboración y el intercambio de conocimientos entre estudiantes y docentes. Se organizan actividades de preparación, capacitaciones y talleres que brindan a los docentes herramientas y estrategias para guiar a los estudiantes en la resolución de problemas geométricos y fomentar su participación en las olimpíadas.

En resumen, las bases matemáticas, didácticas y pedagógicas de las Olimpíadas Matemáticas, especialmente en el campo de la geometría, se centran en una comprensión sólida de los conceptos matemáticos, la resolución de problemas desafiantes como medio de aprendizaje, y un enfoque pedagógico centrado en el estudiante. Estas bases buscan promover el pensamiento crítico, la creatividad y la pasión por las matemáticas, así como el desarrollo de habilidades geométricas y competencias más allá del currículo tradicional.

\subsection{Contenidos}

El taller se centrará en los siguientes temas: 
\begin{itemize}
	\item Figuras geométricas básicas: Lugares geométricos relacionados con la equidistancia. Puntos, rectas y circunferencias relacionados con triángulos. Construcciones de triángulos a partir de condiciones geométricas.
	\item Cuadriláteros convexos: Propiedades de los lados, diagonales y ángulos interiores.
	\item La circunferencia. Cuadriláteros inscritos en una circunferencia 
	\item La semejanza. Criterio de semejanza de triángulos. Aplicaciones: El teorema de Pitágoras y el área de polígonos. 
	\item Razones trigonométricas. Teoremas del seno. Consecuencias interesantes.
\end{itemize}

\subsection{Requisitos}

Para participar en este taller, se requiere tener conocimientos básicos de geometría y familiaridad con el software GeoGebra 5.0. \url{https://www.geogebra.org/}

\subsection{Objetivos}

Los objetivos principales del taller son: 
\begin{itemize}
	\item Proponer, anticipar y registrar situaciones problemáticas que puedan resolverse utilizando diferentes estrategias, dentro del marco de los conocimientos disponibles.
	\item Identificar y señalar las diferencias y similitudes entre diversas formas de resolución, fomentando la participación activa de los estudiantes y promoviendo la discusión respetuosa de las intervenciones de todos los estudiantes.
	\item Identificar los saberes previos y emergentes de los estudiantes en relación con la situación problemática planteada, para adaptar y personalizar la enseñanza de acuerdo a las necesidades individuales.
	\item Colaborar con los estudiantes para lograr la modelización matemática de las situaciones analizadas, es decir, representarlas y expresarlas en términos de conceptos y relaciones matemáticas adecuadas.
	\item Estimular a los estudiantes a construir generalizaciones y fórmulas matemáticas que permitan abordar y resolver problemas de manera más eficiente y efectiva.
	\item Presentar a los estudiantes tareas que involucren el uso de diferentes software y herramientas tecnológicas, considerando los límites y alcances de cada uno de ellos.
\end{itemize}

\subsection{Metodología}

La metodología utilizada en el taller se basará en el planteo de situaciones problemáticas como enfoque principal para abordar los siguientes temas:

\begin{itemize}
	\item Figuras geométricas básicas: Se plantearán situaciones problemáticas relacionadas con lugares geométricos que involucran la equidistancia. Los participantes resolverán problemas que requieran la identificación de puntos, rectas y circunferencias relacionados con triángulos. Además, se explorarán construcciones de triángulos a partir de condiciones geométricas específicas. \textcite{coxeter1997}
	
	\item Cuadriláteros convexos: Se presentarán situaciones problemáticas que permitan investigar y comprender las propiedades de los lados, diagonales y ángulos interiores de cuadriláteros convexos. Los participantes analizarán y resolverán problemas relacionados con estas propiedades. \textcite{coxeter1997,eves1963}
	
	\item La circunferencia: Se abordarán situaciones problemáticas que involucren la circunferencia y su relación con otros elementos geométricos. Se explorarán cuadriláteros inscritos en una circunferencia y se resolverán problemas relacionados con estas configuraciones. \textcite{coxeter1997}
	
	\item La semejanza: Se plantearán situaciones problemáticas que permitan explorar el concepto de semejanza entre triángulos. Los participantes aplicarán el criterio de semejanza de triángulos para resolver problemas específicos. Además, se investigarán aplicaciones de la semejanza, como la utilización del teorema de Pitágoras y el cálculo del área de polígonos. \textcite{fuxman2019}
	
	\item Razones trigonométricas: Se presentarán situaciones problemáticas que requieran el uso de las razones trigonométricas, en particular el teorema del seno. Los participantes resolverán problemas que involucren el cálculo de medidas de ángulos y longitudes de segmentos utilizando estas razones trigonométricas. Además, se explorarán las consecuencias interesantes derivadas del teorema del seno. \textcite{durell2003}
\end{itemize}

En resumen, la metodología del taller se basará en el planteo de situaciones problemáticas para que los participantes puedan investigar, analizar y resolver problemas relacionados con figuras geométricas, cuadriláteros, la circunferencia, la semejanza y las razones trigonométricas. Este enfoque permitirá un aprendizaje activo, promoviendo el razonamiento y la aplicación de los conceptos geométricos en contextos desafiantes y significativos. \textcite{oma2023}

\subsection{Actividades}

A continuación se detallan las principales actividades a realizar
durante el taller:

\subsubsection{Actividades previas}
\begin{itemize}
	\item Recopilación de material bibliográfico de secundaria en relación al eje Geometría y Medida del Ciclo Básico y Ciclo Orientado: \textcite{meccts2011}
	\item Solicitar a los participantes que recopilen material bibliográfico (libros, manuales, guías) relacionado con el tema de geometría y medida en el nivel de educación secundaria. Los participantes deben revisar los contenidos y enfoques de los materiales seleccionados para tener una visión general de los recursos disponibles. Elaboración de cuadro comparativo de los diferentes conceptos abordados por diferentes editoriales.
	\item Solicitar a los participantes que elaboren un cuadro comparativo que incluya las diferentes editoriales y los conceptos de geometría y medida abordados en sus materiales. Los participantes deben identificar los conceptos específicos y hacer una comparación de cómo se presentan y desarrollan en cada recurso. Elaboración de cuadro comparativo sobre la dificultad que muestran dichos textos en relación a la presentación de problemas de geometría.
	\item Solicitar a los participantes que elaboren un cuadro comparativo que evalúe la dificultad de los textos en relación a la presentación de problemas de geometría. Los participantes deben considerar la claridad de las explicaciones, la profundidad de los problemas y la progresión de dificultad en los textos analizados. 
	\item Los participantes deben explorar y familiarizarse con los recursos disponibles, como actividades interactivas, ejercicios y problemas de geometría.
\end{itemize}

Una guía para la búsqueda y la exploración se dejará en un curso de la plataforma Moodle.

\subsubsection{Primera hora y media presenciales}

\begin{itemize}
	\item Presentación de la Olimpiada Matemática Argentina y el Torneo de Geometría e Imaginación a través de recursos multimediales: Se utilizarán recursos multimediales, como videos, presentaciones o páginas web, para introducir a los participantes en la Olimpiada Matemática Argentina y el Torneo de Geometría e Imaginación. Estos recursos proporcionarán información sobre el propósito, las características y los desafíos de estas competencias matemáticas.
	\item Reconocimiento de páginas oficiales y el material de sustento proporcionado por cada página: Los participantes explorarán las páginas web oficiales de la Olimpiada Matemática Argentina y el Torneo de Geometría e Imaginación. Se les guiará para que identifiquen y examinen el material de sustento proporcionado en estas páginas, como guías de estudio, problemas anteriores y recursos didácticos. \textcite{oma2023}
	\item Presentación de material bibliográfico específico de la Olimpiada Matemática: Se compartirá con los participantes material bibliográfico específico relacionado con la Olimpiada Matemática Argentina y el Torneo de Geometría e Imaginación. Esto puede incluir libros, artículos o documentos que contengan ejercicios y problemas de geometría enfocados en las competencias.
	\item Presentación por parte de los docentes participantes de los cuadros comparativos elaborados: Los docentes participantes compartirán los cuadros comparativos que elaboraron previamente, donde se analizan y comparan diferentes recursos y materiales relacionados con la geometría y las olimpiadas matemáticas. Esto permitirá a los demás docentes conocer diferentes enfoques y contenidos presentes en el material disponible.
	\item Discusión en grupos pequeños sobre las características principales del material y la elección del material a utilizar durante el taller: Se formarán grupos pequeños de discusión, donde los participantes intercambiarán ideas y opiniones sobre las características principales del material presentado y su pertinencia para el desarrollo del taller. Se buscará consenso y reflexión sobre qué material utilizar durante el taller, teniendo en cuenta las necesidades y objetivos de los participantes.
	\item Estas actividades proporcionarán a los participantes una visión general de la Olimpiada Matemática Argentina y el Torneo de Geometría e Imaginación, les permitirán familiarizarse con el material de apoyo y promoverán la reflexión y la elección adecuada del material para el desarrollo del taller. Además, se fomentará la colaboración y el intercambio de ideas entre los participantes para enriquecer la experiencia de aprendizaje.
\end{itemize}

\subsubsection{Primeras dos horas entre clases}

Durante el taller, se les pedirá a los cursantes que resuelvan tres problemas de Torneo de Geometría e Imaginación, los cuales estarán adaptados a diferentes ciclos educativos. Estos problemas se resolverán de manera asíncrona, lo que significa que cada participante podrá trabajar en ellos a su propio ritmo y en el momento que le resulte más conveniente.

Se proporcionarán los problemas a los participantes, junto con los recursos y materiales necesarios para abordarlos. Cada problema estará diseñado para desafiar el pensamiento geométrico y la creatividad de los participantes, alentándolos a aplicar los conceptos y estrategias aprendidas durante el taller.

En resumen, los cursantes del taller deberán resolver tres problemas de Torneo de Geometría e Imaginación de forma asíncrona, y prepararse para la segunda sesión presencial para compartir y discutir sus soluciones en un ambiente de aprendizaje colaborativo. \textcite{oma2023}

\subsubsection{Segundas dos horas presenciales}

Una vez que los cursantes hayan resuelto los problemas de forma individual, se llevará a cabo una puesta en común. Durante esta sesión, los participantes compartirán sus soluciones y estrategias utilizadas para resolver los problemas. Se fomentará la discusión y el intercambio de ideas entre los participantes, en un entorno de respeto y colaboración.

La puesta en común permitirá que los cursantes aprendan de las diferentes aproximaciones y enfoques utilizados por sus compañeros, y también brindará la oportunidad de recibir retroalimentación por parte de los encargados de taller y del resto del grupo. Esta actividad enriquecerá el aprendizaje colectivo y contribuirá a fortalecer las habilidades y el razonamiento geométrico de los participantes.

\subsubsection{Segundas dos horas entre clases}

Durante las siguientes horas asincrónicas del taller, se seguirá utilizando la misma metodología de trabajo que se ha empleado anteriormente, pero con un enfoque ligeramente diferente. En lugar de resolver problemas de geometría de las olimpiadas seleccionados previamente, en estas horas se les pedirá a los cursantes del taller que intenten generar problemas de aplicación relacionados con los conceptos geométricos estudiados.

La tarea principal de los participantes será utilizar los conocimientos adquiridos hasta el momento y su creatividad para diseñar situaciones problemáticas que requieran la aplicación de los conceptos y habilidades geométricas abordadas en el taller. Estos problemas de aplicación deben ser desafiantes y significativos, presentando un contexto real o imaginario donde los conceptos geométricos sean relevantes.

Los participantes tendrán la libertad de elegir diferentes temas o áreas de aplicación, como la arquitectura, la ingeniería, el diseño, la navegación, entre otros. Se les animará a considerar situaciones problemáticas que les resulten interesantes y relevantes para su contexto educativo o profesional.

Durante estas horas asincrónicas, los cursantes del taller trabajarán de manera individual o en grupos pequeños para generar los problemas de aplicación. Se les proporcionará una guía o plantilla que les ayude a estructurar los problemas de manera clara, incluyendo la descripción del contexto, los datos relevantes, las preguntas a responder y las indicaciones para la resolución.

Esta actividad de generación de problemas de aplicación fomentará el pensamiento crítico, la creatividad y la capacidad de transferir los conocimientos geométricos a situaciones reales. Además, permitirá a los participantes desarrollar habilidades de diseño de problemas y enriquecerá el banco de recursos didácticos del taller con nuevas situaciones problemáticas para futuras actividades.

\subsubsection{Terceras dos horas presenciales}

Una vez generados los problemas de aplicación, los participantes tendrán la oportunidad de compartir y discutir sus creaciones en esta sesión presencial del taller. Durante esta sesión, se promoverá el intercambio de problemas entre los participantes, brindando retroalimentación constructiva y sugiriendo posibles mejoras o modificaciones.

\subsubsection{Evaluación final}

La evaluación final de este taller se llevará a cabo mediante la presentación de posibles soluciones a un problema de Geometría propuesto a nivel nacional de la Olimpiada Matemática Argentina y la resolución de un problema de Torneo de Geometría y Medida utilizando modalidades diferentes.

Para la evaluación del problema de Geometría propuesto de nivel nacional, los participantes deberán analizar en detalle el enunciado del problema, identificar los conceptos y técnicas geométricas relevantes y proponer una solución rigurosa y fundamentada. Se espera que demuestren un sólido dominio de los conceptos geométricos abordados en el taller, así como habilidades para plantear estrategias de resolución y justificar sus razonamientos. 

Por otro lado, la resolución del problema de Torneo de Geometría y Medida se llevará a cabo utilizando modalidades diferentes. Esto implica que los participantes deberán abordar el problema desde diferentes perspectivas o enfoques, empleando diversas estrategias de resolución. Se espera que muestren flexibilidad y creatividad en la búsqueda de soluciones, demostrando un pensamiento geométrico sólido y habilidades para argumentar y justificar sus procesos de resolución. 

En ambos temas, se valorará tanto la precisión y corrección de las soluciones propuestas como la claridad y coherencia en la presentación de los razonamientos. Se espera que los participantes apliquen los conocimientos adquiridos durante el taller de manera efectiva, demostrando un buen nivel de comprensión de los conceptos y habilidades geométricas abordadas. También se evaluará su capacidad para comunicar de manera clara y concisa sus ideas y resultados.

La evaluación final a través de la presentación de soluciones a un problema de Geometría de nivel nacional y la resolución de un problema de Torneo de Geometría y Medida con modalidades diferentes permitirá evaluar de manera integral el desempeño y los logros de los participantes en el taller, proporcionando una retroalimentación valiosa sobre su dominio de los contenidos y habilidades geométricas, así como su capacidad para aplicarlos en diferentes contextos y modalidades de resolución.

\subsection{Bibliografía}

\nocite{*}
\printbibliography[keyword={05}]