%% LyX 2.3.6 created this file.  For more info, see http://www.lyx.org/.
%% Do not edit unless you really know what you are doing.
\documentclass[11pt,oneside,spanish,thmsa]{amsart}
\usepackage[latin9]{inputenc}
\usepackage[a4paper]{geometry}
\geometry{verbose,tmargin=1cm,bmargin=2cm,lmargin=1cm,rmargin=1cm}
\usepackage{enumitem}
\usepackage{amsthm}
\usepackage{amssymb}

\makeatletter
%%%%%%%%%%%%%%%%%%%%%%%%%%%%%% Textclass specific LaTeX commands.
\numberwithin{equation}{section}
\numberwithin{figure}{section}
\newlength{\lyxlabelwidth}      % auxiliary length 

\makeatother

\usepackage{babel}
\addto\shorthandsspanish{\spanishdeactivate{~<>}}

\begin{document}
\title{Autoridades y Pr�logo}
\maketitle

\section*{Departamento de Matem�tica}
\begin{itemize}
\item Director Prof. Julio Pojasi
\item Vicedirector Dr. Jorge Yazlle
\item Secretario CPN. Cristian Pinto
\item Pro Secretario Prof. Ivone Patagua
\end{itemize}

\section*{Comit� Organizador}
\begin{itemize}
\item Prof. Celia Villagra
\item Prof. Silvia Baspi�eiro
\item Prof. Ivone Patagua
\item Prof. Blanca Azucena Formeliano
\item Prof. Antonio No� S�ngari
\end{itemize}

\section*{Pr�logo}

Las VII Jornadas de Ense�anza de la Matem�tica h�bridas, realizadas
en su parte presencial en las instalaciones de la Universidad Nacional
de Salta del 24 de julio al 2 de agosto, representaron un hito en
el continuo esfuerzo por mejorar la calidad de la educaci�n matem�tica
en nuestra regi�n. Durante esos d�as, docentes, investigadores y estudiantes
se congregaron con el objetivo de reflexionar sobre sus pr�cticas
docentes, compartiendo experiencias y conocimientos que, sin duda,
contribuir�n al enriquecimiento de la comunidad educativa.

Las Jornadas ofrecieron un variado programa de actividades que incluy�
conferencias magistrales, talleres pr�cticos y comunicaciones breves.
Entre las actividades destacadas, se encuentran las conferencias de
Valeria Borsani, que abord� \emph{El pasaje de la aritm�tica al �lgebra
en los primeros a�os de la escuela media}, la de Andr�s Rieznik, que
abord� \emph{Discalculia, un cap�tulo olvidado de la neuropsicolog�a,
}y el de Daniela Reyes, \emph{Empoderamiento docente, �por qu� pensar
en ello?} y cinco talleres sobre ense�anza de la matem�tica. La participaci�n
activa de los asistentes reflej� el inter�s y el compromiso con la
mejora continua de la ense�anza de la matem�tica.

Queremos expresar nuestro m�s sincero agradecimiento a todos los participantes,
ponentes y talleristas que, con su dedicaci�n y entusiasmo, hicieron
posible el �xito de este evento. Asimismo, extendemos nuestro reconocimiento
a la Universidad Nacional de Salta y a todas las instituciones que
nos brindaron su apoyo.

Las presentes Memorias recogen las ideas, debates y conclusiones surgidas
durante las Jornadas, sirviendo como un valioso recurso para todos
aquellos interesados en la ense�anza de la matem�tica. Esperamos que
este compendio inspire a los lectores a continuar explorando y aplicando
nuevas estrategias pedag�gicas en sus aulas.

Con la mirada puesta en el futuro, confiamos en que las pr�ximas ediciones
de las Jornadas de Ense�anza de la Matem�tica seguir�n siendo un espacio
de encuentro, reflexi�n e innovaci�n para nuestra comunidad.

Antonio S�ngari

Coordinador de las VII Jornadas de Ense�anza de la Matem�tica
\end{document}
