%-------------------------------------------------------------------------
% INFORMACIÓN DEL ARTÍCULO
\thispagestyle{portadapage}
\setcounter{subsection}{0}
\setcounter{subsubsection}{0}
\setcounter{actividad}{0}
\setcounter{actividad_previa}{0}
\setcounter{actividad_entre}{0}
\renewcommand{\articulotipo}{Comunicación breve}
\renewcommand{\articulotitulo}{Matemática, uno de los ejes principales en la Feria de Ciencias STEAM}
\renewcommand{\articulotitulocorto}{Matemática, uno de los ejes principales en la Feria de Ciencias STEAM}
\section{\articulotitulo}
\desctotoc{Céliz, Z. F.; Shneider, E.}

\noindent\rule{\linewidth}{2pt}

\vspace{0.25cm}

\begin{flushright}
	\addautor[zulmaceliz@gmail.com]{Zulma Fabiana Céliz}{-}
	\vspace{1em}
	\addautor[]{Ernesto Shneider}{-}
\end{flushright}

\vspace{0.5cm}

\begin{center}
	\begin{minipage}{0.75\linewidth} \small
		\textsc{Resumen}. ~
		Las Ferias de Ciencias, estrategia pedagógica implementadas en comunidades educativas en los distintos niveles y modalidades, públicas y privadas, se instalan en ellas para la mejora de los aprendizajes y de optimización de la enseñanza.
		
		En este año se espera que los proyectos feriales se fundamenten y desarrollen desde la perspectiva STEAM situada, adaptada al Nivel y Modalidad Educativa y al contexto de la escuela/comunidad donde se genera y desarrolla.
		
		La metodología STEAM, modelo educativo que promueve la integración de las áreas científicas, tecnológicas, matemática y artísticas en un único marco interdisciplinar, busca el desarrollo de habilidades en los alumnos de las distintas áreas, fomentar el pensamiento crítico y el razonamiento lógico. Se destacará la importancia de la Matemática como uno de los ejes principales de ésta metodología y que servirá para vertebrar los proyectos integrados y la construcción del aprendizaje autónomo y crítico.
	\end{minipage}
\end{center}
%-------------------------------------------------------------------------

\subsection{Introducción}

El enfoque STEAM se presenta como un conjunto de didácticas (singulares, innovadoras) que prioriza la actividad de los estudiantes de modo interdisciplinario y contextualizado por medio de la resolución de problemas, basados en el planteo y desarrollo de múltiples y variados proyectos de aula. Basado en los cambios de patrones o estándares que surgen de las necesidades emergentes de la sociedad y de la baja tendencia de sus indicadores económicos, ya que busca no solo ofrecer conocimientos específicos de cada disciplina, sino también promover habilidades y capacidades en un permanente ecosistema de aprendizaje.

Propone entre otros: mejorar la calidad de las experiencias de aprendizaje de los alumnos (Rogers, 2005); aumentar el interés de los estudiantes, mejorar los saberes en las prácticas de alfabetización y demostrar la utilidad de las matemáticas y las ciencias (Gattie y Wicklein, 2007); mejorar la alfabetización tecnológica (Rogers, 2005).

La feria de ciencias brinda la oportunidad para la construcción, implementación y evaluación del enfoque STEAM. En la actualidad las ferias de ciencias incluyen proyectos de todas las áreas curriculares y se extienden a todos los niveles y modalidades del sistema educativo nacional.

Cada feria propicia que el foco de todos los proyectos y actividades que en ella se presenten se encuentren en los contenidos de los diseños curriculares correspondientes a cada una de las jurisdicciones, en los Núcleos de Aprendizajes Prioritarios (NAP) y aquellos documentos que regulen la enseñanza en cada contexto educativo.

Las ferias forman parte de la planificación escolar y pueden considerarse un enfoque educativo con objetivos didácticos asociados al cotidiano de la escuela, a la enseñanza y, fundamentalmente, a la integración de aprendizajes. Resultan un enfoque que apunta a su mejora, es decir, al aumento y la promoción en la calidad de habilidades y capacidades de quienes forman parte del proyecto educativo. Reflejan la integración en la construcción y reconstrucción del conocimiento escolar. Asimismo, el proyecto se retroalimenta dando relevancia a la evaluación formativa, optimizan el proceso en constante enriquecimiento del proyecto final.

Los eventos de feria de ciencias se orientan por las normas escolares, la convivencia escolar, los diseños curriculares jurisdiccionales y los documentos federales y nacionales. Se integran ferias de ciencias semejantes de otros países. Se constituyen como un evento lejano a una competencia de equipos o una contienda de logros individuales. En las ferias no se rinde examen, no hay pruebas a superar, sino saberes por construir y reconstruir en un proceso educativo integrado. Al exhibir la producción se genera un reconocimiento auténtico del proyecto de ferias, ya que se comparte con otros actores de la escuela, de la localidad, de la jurisdicción y, al final de su recorrido, del país y posiblemente también del extranjero.

Para el Programa Nacional de Ferias de Ciencias y Tecnología la base epistémica de una propuesta STEAM es el “aprender a aprender”, ya que reconoce que el proceso de aprender no se limita simplemente a adquirir conocimientos y habilidades específicas en áreas particulares, sino que implica también desarrollar habilidades metacognitivas, capacidades de resolución de problemas y disposiciones para el aprendizaje continuo y adaptativo. Para esto recurrimos a dos planteos teóricos: el aprendizaje significativo y el aprendizaje crítico.

Parafraseando la obra de Paulo Freire concluimos que en una feria de ciencias es evidente que quienes enseñan, aprenden al enseñar; y quienes aprenden, enseñan al aprender. El conocimiento no es transmitido, sino que se construye o se produce, y tanto el educando como el educador deben percibirse y asumirse como sujetos activos en el proceso de construcción. Por lo que las ferias de ciencias proponen un escenario propicio para la proyección de una estrategia “STEAM situada”, dado que las características de sus trabajos y su evaluación proporcionan experiencias concretas atentos a los retos que nuestro sistema educativo debe afrontar.

Unos de los ejes principales de la metodología STEAM, MATEMATICA, que podrá tenerse en cuenta como vertebrador de éstos proyectos para los trabajos de Feria integrándose con otros campos curriculares de los otros ejes, para los cuales se considera: cómo se construye y reconstruye el conocimiento escolar, cómo se elaboran y reelaboran los saberes desde el aula, la valoración realizada, apropiación y desarrollo creativo del trabajo, con relación a ámbitos naturales o culturales, la realización o promoción de aportes a los procesos de enseñanza y aprendizaje, a la vida institucional de los establecimientos educativos de pertenencia de la zona y la vinculación del proyecto con el contexto social. Los trabajos que tengan como punto de partida el eje MATEMATICO deben ser formulados sobre temas curriculares vinculados con Aritmética, Álgebra, Cálculo, Geometría, Estadística, Probabilidades, pero también sobre temas que articulan con Topología, y aplicaciones matemáticas en otras áreas que a partir del abordaje de una problemática compleja articulen aspectos matemáticos para su planteo o resolución. También Historia de la Matemática podrá se considerarse vinculada con el área de matemática. Se diferenciarán tres tipos de proyectos matemáticos: los relacionados con el uso de la Matemática en otras áreas de conocimiento, aquellos relacionados con problemas matemáticos y los vinculados con la historia de la matemática. Los cuales deberán cumplir con ciertos criterios específicos: significatividad del problema elegido y pertinencia del análisis realizado, delimitación del problema de otra área a cuya comprensión aporta la matemática mediante el uso de modelos matemáticos, relevancia del problema elegido, utilización pertinente de diferentes modelos matemáticos al resolver el problema y uso adecuado de representaciones; claridad en la comunicación de los procedimientos utilizados y las nociones matemáticas involucradas; explicitación de manera clara y completa de las formas de resolución y de las nociones y propiedades involucradas, utilizando el lenguaje en forma adecuada; presentación, detalle, dibujos y gráficos; ordenamiento y sistematización; utilización pertinente de diferentes modelos matemáticos al resolver el problema; indagación sobre una noción en distintos momentos históricos, en el marco de las ideas de su tiempo.\footnote{Tomado de los documentos 1, 2, y 3 del Material del Programa Nacional de Feria de Ciencias}.

\subsection{Contenidos}

Feria de Educación, Ciencia, Arte, y Tecnología de la Provincia. Feria de Ciencia STEAM Situada. Eje Matemático y foco principal en el desarrollo de Proyecto de Feria de Ciencias.

\subsection{Objetivos}
\begin{itemize}
	\item Destacar la importancia de la Matemática como eje vertebrador de un proyecto de Feria de Ciencias STEAM situada, en un proceso en el cual se busca la mejora de la enseñanza y la calidad de los aprendizajes adquiridos.
	\item Visibilizar el trabajo de Feria de Ciencias desarrollada por docentes y alumnos, adoptando y apropiándose de formas efectivas y eficaces de estrategias del proceso de enseñanza/aprendizaje; y que proponen formas superadoras a implementar en la enseñanza de la Matemática.
\end{itemize}

\subsection{Desarrollo de la comunicación}

\begin{itemize}
	\item A cargo de la Profesora Zulma Céliz, exposición sobre Feria de Ciencia, exposición sobre características principales e historia.
	\item Presentación del Profesor Referente Nacional de Evaluación Feria de Ciencia: Ernesto Scheiner
	\item Exposición sobre: “El eje de Matemática en una Feria de Ciencias STEAM situada”.
	\item Recursos: Conexión virtual vía internet. Material de apoyo: diapositivas
\end{itemize}

\subsection{Bibliografía}

\nocite{*}
\printbibliography[keyword={14}]