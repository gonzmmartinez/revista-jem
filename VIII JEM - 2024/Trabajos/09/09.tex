%-------------------------------------------------------------------------
% INFORMACIÓN DEL ARTÍCULO
\thispagestyle{portadapage}
\setcounter{subsection}{0}
\setcounter{subsubsection}{0}
\setcounter{actividad}{0}
\setcounter{actividad_previa}{0}
\setcounter{actividad_entre}{0}
\renewcommand{\articulotipo}{Comunicación breve}
\renewcommand{\articulotitulo}{Herramientas de la Teoría de Juegos para fomentar la cooperación en el nivel inicial y  primario: Teoría de Juegos}
\renewcommand{\articulotitulocorto}{Herramientas de la Teoría de Juegos para fomentar la cooperación en el nivel inicial y  primario.: Teoría de Juegos}
\section{\articulotitulo}
\desctotoc{Villagra, C.; Carrasco, R.; Alvarez, D.; Miguez, I.}

\noindent\rule{\linewidth}{2pt}

\vspace{0.25cm}

\begin{flushright}
	\addautor[alifangi@gmail.com]{Alicia Fanny Gimenez}{Universidad Nacional de San Juan}
	\vspace{1em}
	\addautor[]{Eliana 	Perona}{Universidad Nacional de San Juan}
\end{flushright}

\vspace{0.5cm}

\begin{center}
	\begin{minipage}{0.75\linewidth} \small
		\textsc{Resumen}. ~
		Es esencial que la escuela de hoy genere condiciones básicas que aseguren tanto el desarrollo de las riquezas personales como las actitudes solidarias, pluralistas y democráticas.
		
		El objetivo de esta comunicación es concientizar sobre la importancia de desarrollar el pensamiento estratégico, la toma de decisiones y, por ende, incorporar contenidos que permitan desarrollar estas capacidades. La Teoría de Juegos facilita tal propósito.
		
		El juego es una estrategia pedagógica que promueve múltiples aprendizajes y le permite al niño conocer, investigar, experimentar, descubrir su contexto de una manera amigable y lúdica.
		
		Partiendo de esto, tomamos una rama de la Matemática como es la Teoría de Juegos para el desarrollo de la alfabetización social y fomento de las habilidades sociales que se van adquiriendo a través del aprendizaje.
		
		Estamos convencidos de que es necesario promover el trabajo cooperativo bajo la resolución de problemas. Para ello ofrecemos actividades para implementar en el aula, usando conceptos y fundamentos de la Teoría de Juegos.
	\end{minipage}\\
	
	\vspace{0.5em}
	
	\begin{minipage}{0.75\linewidth} \small
		\textsc{Palabras clave} --- Teoría de Juegos, Toma de decisiones, Pensamiento estratégico
	\end{minipage}
\end{center}
%-------------------------------------------------------------------------

\subsection{Introducción}

Partimos de la convicción de que es necesario ampliar la diversidad de saberes y promover el trabajo cooperativo bajo la resolución de problemas o conflictos. Esto trata de fomentar que dos o más personas intenten resolver un problema compartiendo significados y tomando en consideración el esfuerzo y las estrategias que utilizan. Esta perspectiva subraya las ventajas de la cooperación frente a la resolución individual.

Es esencial que la escuela de hoy genere condiciones básicas que aseguren tanto el desarrollo de las riquezas personales como las actitudes solidarias, cooperativas, pluralistas y democráticas.

Esta propuesta le otorga a la escuela primaria el desafío de ofrecer herramientas cognitivas y desarrollar competencias blandas que son rasgos de personalidad, habilidades socioemocionales, de comunicación, lenguaje y hábitos que moldean los vínculos que los individuos establecen con otras personas y, de esta manera, actuar de modo crítico, creativo, cooperativo y responsable frente a la información y sus usos para la construcción de conocimientos socialmente válidos. La nueva rama de la Matemática que facilita tal propósito es la \textbf{Teoría de Juegos}. La misma muestra cómo con sencillas estrategias, los alumnos pueden encontrar otras formas de resolución de conflictos teniendo a la cooperación, como una opción válida.

\subsection{Contenidos}

\begin{itemize}
	\item Pensamiento estratégico y toma de decisiones.
	\item Introducción a la Teoría de Juegos.
	\item Modelos: El Dilema del Prisionero. Tragedia de los comunes.
	\item Aplicación de los modelos en situaciones que involucran los contenidos de la Teoría de Juegos para el nivel Primario.
\end{itemize}

\subsection{Objetivos}

\begin{itemize}
	\item Abordar los conceptos básicos de Teoría de Juegos y las técnicas empleadas en esta teoría para el desarrollo del pensamiento estratégico y la toma de decisiones.
	\item Utilizar el juego como herramienta didáctica para fortalecer el proceso de enseñanza y aprendizaje.
	\item Promover y describir la relación que existe entre la implementación del juego cooperativo, como estrategia de enseñanza, y el fortalecimiento de la competencia matemática “resolución de problemas” en los estudiantes.
\end{itemize}

\subsection{Desarrollo de la comunicación}

Se comenzará señalando la importancia del desarrollo del pensamiento estratégico y la toma de decisiones. Esto dará lugar a una breve introducción sobre conceptos básicos de la Teoría de Juegos. A continuación, se desarrollarán algunos modelos de esta teoría como “el dilema del prisionero” y “la tragedia de los comunes”, y la manera de aplicarlos en la educación primaria.

Esta exposición se desarrollará mediante una presentación de diapositivas y uso de material multimedia como videos cortos ejemplificando algunos de los modelos antes mencionados.

\subsection{Bibliografía}

\nocite{*}
\printbibliography[keyword={09}]