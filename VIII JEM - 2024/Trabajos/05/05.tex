%-------------------------------------------------------------------------
% INFORMACIÓN DEL ARTÍCULO
\thispagestyle{portadapage}
\setcounter{subsection}{0}
\setcounter{subsubsection}{0}
\setcounter{actividad}{0}
\setcounter{actividad_previa}{0}
\setcounter{actividad_entre}{0}
\renewcommand{\articulotipo}{Taller}
\renewcommand{\articulotitulo}{Avanzando con las propiedades de los conjuntos numéricos: Encadenamientos y ausencias entre la escuela Primaria y la escuela Secundaria}
\renewcommand{\articulotitulocorto}{Avanzando con las propiedades de los conjuntos numéricos}
\section{\articulotitulo}
\desctotoc{Villagra, C.; Carrasco, R.; Alvarez, D.; Miguez, I.}

\noindent\rule{\linewidth}{2pt}

\vspace{0.25cm}

\begin{flushright}
	\addautor[susannedaher013@gmail.com]{Mariette Susanne Daher}{Universidad Nacional de Salta}
	\vspace{1em}
	\addautor[]{Josefina Lávaque Fuente}{Universidad Nacional de Salta}
	\vspace{1em}
	\addautor[]{Blanca Azucena Formeliano}{Universidad Nacional de Salta}
\end{flushright}

\vspace{0.5cm}

\begin{center}
	\begin{minipage}{0.75\linewidth} \small
		\textsc{Resumen}. ~
		El taller tiene como intención promover las problemáticas de prácticas de enseñanza sobre el estudio de las propiedades de los conjuntos numéricos naturales $\mathbb{N}$; enteros $\mathbb{Z}$ y racionales $\mathbb{Q}$.
		
		Asimismo, se brindará entradas para el objeto de estudio con distintos caminos; que el docente podrá analizar y reformular teniendo en cuenta las continuidades y rupturas entre la escuela primaria y la escuela secundaria sobre las propiedades de los conjuntos numéricos.
		
		Desde la mirada didáctica que sostienen los documentos curriculares tanto nacionales como jurisdiccionales, se pretende problematizar los recorridos de los contenidos y prácticas propios de la escuela primaria y de la escuela secundaria sobre los conjuntos numéricos.
		
		Durante el taller se propondrán actividades que permitan estudiar y rescatar propiedades de los conjuntos numéricos que se estudian en la escuela primaria como uno más que, uno menos que, entre números naturales, regularidades en la serie escrita del conjunto de números naturales y se continuará con el estudio de las propiedades de los números enteros y racionales para reflexionar acerca de la continuidad y provisoriedad de los conocimientos construidos sobre las propiedades de los números naturales al ampliar cada conjunto numérico.
	\end{minipage}\\
\end{center}
%-------------------------------------------------------------------------

\subsection{Introducción}

Este taller contribuirá a profundizar las propiedades que poseen los conjuntos numéricos; las que son propias y las que se van incrementando a medida que se avanza en el estudio de los conjuntos numéricos; como así también permitirán el estudio y la reflexión alrededor de los obstáculos y los errores que se producen. Desde un punto de vista matemático y didáctico, por medio de la resolución de problemas, se propone reflexionar a través de:
\begin{itemize}
	\item Presentación y resolución de problemas.
	\item Socialización de los procedimientos que hacen a la resolución de problemas
	\item Elaboración de conclusiones, mediante diagramas o tablas
\end{itemize}

\subsection{Contenidos}

\begin{itemize}
	\item \textbf{Módulo 1}: Números naturales. Tabla numérica. Propiedades
	\item \textbf{Módulo 2}: Números enteros. Tabla numérica. La recta numérica. Propiedades.
	\item \textbf{Módulo 3}: Números racionales. La recta numérica. Propiedades
\end{itemize}

\subsection{Requisitos previos}

Los docentes deberán tener conocimiento sobre los NAP. Diseños Curriculares de primaria y secundaria.

\subsection{Objetivos}

\begin{itemize}
	\item Analizar las propiedades de los conjuntos numéricos en la resolución de problemas.
	\item Establecer las variables didácticas que permiten poner juego las propiedades de los conjuntos numéricos.
	\item Identificar problemas y estrategias de resolución en relación con las propiedades de los conjuntos numéricos en la propia tarea y de la tarea con otros colegas.
	\item Reflexionar acerca de las continuidades y rupturas entre la escuela primaria y la escuela secundaria sobre las propiedades de los conjuntos numéricos desde perspectivas de enseñanza de la matemática sostenidas en documentos curriculares.
\end{itemize}

\subsection{Actividades}

\subsubsection{Actividades previas}

Lectura y análisis del siguiente texto para comentar en la primera clase sincrónica.

\bigskip

\begin{center}
	\begin{minipage}{0.8\linewidth}
		\begin{center}
			\bfseries
			Números Naturales y Enteros
		\end{center}
		
		Los números naturales, de símbolo $\mathbb{N}$, son todos los números enteros positivos, es decir, todas aquellas cifras sin decimales y mayores a 0. Algunos ejemplos de números naturales son 1, 6, 23, 147 y 30500.
		
		Dependiendo del área de ciencia y el convenio utilizado, los números naturales se representan en uno de los siguientes conjuntos:
		\begin{itemize}
			\item El conjunto de naturales sin el cero, que comienza con 1: $\mathbb{N} = \{ 1, 2, 3, 4, 5, 6, 7, 8, \dots \}$
			\item El conjunto de naturales con el cero, que empieza con dicha cifra: $\mathbb{N} = \{ 0, 1, 2, 3, 4, 5, 6, 7, 8, \dots \}$
		\end{itemize}
		
		No obstante, como el 0 no puede ser ni positivo ni negativo, es preferible no incluirlo dentro del conjunto de números naturales, pues solo aborda los números enteros positivos.
		
		Los números naturales fueron los primeros números que empleamos para cuantificar objetos. Con el tiempo, los hemos utilizado para ordenar valores, comparar cantidades diferentes y como base para todo tipo de operaciones matemáticas. De hecho, para obtener otros números, como los fraccionarios, nos servimos muchas veces de los números naturales.
	\end{minipage}
	
		\bigskip
		
		\begin{minipage}{0.8\linewidth}
			\begin{center}
				\bfseries
				Propiedades de los números naturales
			\end{center}
			
			\textbf{Los números naturales solo presentan números enteros positivos, es decir, del 1 en adelante}. Los números negativos quedan fuera del conjunto de los naturales.
			
			\begin{itemize}
				\item Los números fraccionarios o con cifras decimales tampoco encajan en el conjunto de números naturales.
				\item Todos los números naturales poseen un sucesor y siguen un orden específico. En otras palabras, para cada número natural existe uno mayor que viene justo después ($4\to5$, $19\to20$, $110\to111$, $3041\to3042$, etc.).
				\item Hay una cantidad infinita de números naturales, ya que siempre podemos hallar un número natural que sea mayor a otro.
				\item Entre dos números naturales hay un número finito de naturales. Por ejemplo, entre 5 y 12 solo hay seis números naturales: 6, 7, 8, 9, 10 y 11.
			\end{itemize}
		\end{minipage}
		
		\bigskip
		
		\begin{minipage}{0.8\linewidth}
		\begin{center}
			\bfseries
			Clasificación y ejemplos de números enteros
		\end{center}
		
		Los números enteros se agrupan en tres subconjuntos: el 0, los enteros positivos y los enteros negativos. El 0 tiene su propia categoría al ser un \textbf{valor neutro}, es decir, un número que \textbf{no puede ser ni positivo ni negativo}.
		
		A continuación, compartimos propiedades y características de los números enteros:
		\begin{itemize}
			\item Existe una \textbf{cantidad infinita de números enteros}, tanto positivos como negativos. La prueba es que siempre podemos hallar un número entero más pequeño o más grande que otro número entero.
			\item \textbf{Entre dos números enteros} hay una \textbf{cantidad finita de enteros}. Por ejemplo, entre -7 y 4 existen 10 números enteros, que son: -6, -5, -4, -3, -2, -1, 0, 1, 2 y 3.
			\item Para cada número entero siempre hay otro mayor denominado \textbf{sucesor}. Por ello, los números enteros siguen un orden específico que no cambia. Por ejemplo, al número 5 le sigue el número 6, después del 101 viene el 102, y el sucesor de -21 es -20.
			\item En la recta numérica, \textbf{los números enteros más pequeños se sitúan a la izquierda}, mientras que \textbf{los más grandes se sitúan a la derecha}.
			\item Cualquier suma, resta y multiplicación entre dos números enteros siempre devolverá otro número entero. No es así con las divisiones, ya que hay casos en que la división de dos números enteros devuelve un número fraccionario, es decir, un valor con cifras decimales.
			\item El valor absoluto de un número entero es siempre el mismo, independientemente de su signo, ya que simplemente mide la distancia del número respecto al cero. Por ejemplo, el valor absoluto de $|+11|$ y $|-11|$ es igual: 11.
		\end{itemize}
	\end{minipage}
\end{center}

\bigskip

\subsubsection{Presentación del taller}

\begin{itemize}
	\item Objetivos, Contenidos, forma de trabajo
	\item Criterios e indicadores de evaluación.
	\item Trabajo grupal del problema 1
	\item Puesta en común
	\item Contenidos: Números naturales. Importancia de la tabla numérica. Propiedades de los números naturales
\end{itemize}

\begin{itemize}
	\item \textbf{Tareas grupales para exponer:}
	\begin{enumerate}
		\item Resuelva los siguientes problemas.
		\item Enuncie otro problema más complejo.
		\item Identifique los conceptos involucrados como saberes previos.
		\item NAP o Diseño Curricular Jurisdiccional de su provincia ¿Cuáles son los contenidos de la educación obligatoria propuestos en los diseños curriculares, que se relacionan con los problemas?
		\item Escriba un problema que se corresponda con 6to, 7mo, 8vo, 9no año de escolaridad obligatoria
	\end{enumerate}
	
	\item \textbf{Recursos:} Tabla numérica de los 100 primeros números naturales. Recta numérica. Recta numérica con números del -10 al 10
	
	\begin{actividad}
		~
		\begin{enumerate}[a.]
			\item Escribir el siguiente y el anterior de 49.
			\item Escribir el anterior de 1.
			\item Escribir todos los números comprendidos entre 63 y 89.
			\item ¿Es posible encontrar números naturales entre 39 y 40?
			\item ¿Cuánto es 1 más 1000? ¿y uno más 2000?
			\item Elegir un número de la tabla y escribir: Todos los números que están en la misma columna y en la misma fila. ¿Qué observa de la secuencia de números escritos?
			\item ¿Qué propiedades están implícitas en las consignas anteriores? Identificarlas y escribirlas en forma coloquial y simbólica.
			\item ¿En qué se diferencia utilizar la tabla o la recta numérica?
			\item ¿Qué potencial se observa en la recta numérica para destacar las propiedades recién
			vistas?
		\end{enumerate}
	\end{actividad}
\end{itemize}

\subsubsection{Clase asincrónica}

\begin{itemize}
	\item \textbf{Tiempo:} 3 hs
\end{itemize}

\begin{actividad}
	~
	\begin{enumerate}[a.]
		\item Elaborar un relato de fortalezas y debilidades del primer encuentro sincrónico.
		\item Construir la tabla numérica de los 100 primeros números enteros negativos
	\end{enumerate}
\end{actividad}

\subsubsection{Clase sincrónica 2}

\begin{itemize}
	\item \textbf{Tiempo:} 1 $\nicefrac12$ hs
	
	\item \textbf{Contenidos:} Propiedades de los números enteros
	
	\item \textbf{Recursos:} Tabla numérica de los 100 primeros números naturales. Recta numérica. Tabla numérica de los 100 primeros números enteros negativos. Recta numérica con números del -10 al 10
	
	\item \textbf{Tareas grupales para exponer:}
	\begin{enumerate}[a.]
		\item Resuelva los siguientes problemas en grupo.
		\item Enuncie otro problema más complejo.
		\item Identifique los conceptos involucrados como saberes previos.
		\item NAP o Diseño Curricular Jurisdiccional de su provincia ¿Cuáles son los contenidos de la educación obligatoria propuestos en los diseños curriculares, que se relacionan con los problemas?
		\item Escriba un problema que se corresponda con el 6to, 7mo, 8vo, 9no año de escolaridad obligatoria
	\end{enumerate}
\end{itemize}

\begin{actividad}
	~
	\begin{enumerate}[a.]
		\item Con la tabla de los 100 primeros números negativos.
		\item ¿Se podrán enunciar las mismas tareas que las efectuadas en Actividad 1? Si la respuesta es negativa: ¿Qué cambia?
		\item ¿Qué propiedades están implícitas en las consignas anteriores? Identificarlas y escribirlas en forma coloquial y simbólica
		\item ¿En qué se diferencia utilizar la tabla o la recta numérica para el estudio de los números enteros?
		\item ¿Qué potencial se observa en la recta numérica para destacar las propiedades recién vistas?
	\end{enumerate}
\end{actividad}

\subsubsection{Clase asincrónica}

\begin{itemize}
	\item \textbf{Tiempo:} 3 hs
\end{itemize}

\begin{actividad}
	~
	\begin{enumerate}[a.]
		\item Elaborar un relato de fortalezas y debilidades el segundo encuentro sincrónico
		\item Analizar los siguientes videos e identificar las propiedades de los números fraccionarios que enseña la docente.
		
		\url{https://youtu.be/MPAuLf8C8IE?si=8n6DSfR4qtZ3ZjAn}
		
		\url{https://youtube.com/watch?v=uopbujGp5X8&feature=shared}
		
		\item A partir del análisis de tareas en los libros de texto formular cinco tareas, en los años que se desempeña a partir de las cuales se deduzcan algunas de las propiedades de los números enteros.
		
		A continuación, se presentarán las propiedades las propiedades del sistema de numeración y de los conjuntos numéricos de naturales, enteros y racionales.
	\end{enumerate}
\end{actividad}

\subsubsection{Clase sincrónica 3}

\begin{itemize}
	\item \textbf{Tiempo:} 1 $\nicefrac12$
	\item \textbf{Contenidos:} Números racionales y la recta numérica. Completitud en $\mathbb{R}$. Evaluación.
	\item \textbf{Recursos:} Tabla numérica de los 100 primeros números naturales. Recta numérica. Tabla numérica de los 100 primeros números enteros negativos. Recta numérica con números del -10 al 10.
\end{itemize}

\begin{actividad}
	~
	\begin{enumerate}
		\item Qué recurso utilizarías o sería más adecuado para deducir las propiedades de los números racionales?
		\item ¿Cuáles son esas propiedades? Justificar.
		\item ¿Se podrán enunciar las mismas tareas que las efectuadas en la Actividad 2? Si la respuesta es negativa: ¿Qué cambia?
	\end{enumerate}
\end{actividad}

\subsubsection{Evaluación}

\begin{enumerate}
	\item A partir del análisis de tareas en los libros de texto formular cinco tareas, a partir de las cuales se deduzcan algunas de las propiedades de los números RACIONALES.
	\item Elaborar un relato de fortalezas y debilidades del taller. Enviar hasta el 10 de agosto.
\end{enumerate}

\subsection{Bibliografía}

\nocite{*}
\printbibliography[keyword={05}]