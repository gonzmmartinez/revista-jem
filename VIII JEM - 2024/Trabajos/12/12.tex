%-------------------------------------------------------------------------
% INFORMACIÓN DEL ARTÍCULO
\thispagestyle{portadapage}
\setcounter{subsection}{0}
\setcounter{subsubsection}{0}
\setcounter{actividad}{0}
\setcounter{actividad_previa}{0}
\setcounter{actividad_entre}{0}
\renewcommand{\articulotipo}{Comunicación breve}
\renewcommand{\articulotitulo}{Enseñanza de funciones exponenciales y logarítmicas mediada por tecnologías digitales: relato y conclusiones de una experiencia áulica}
\renewcommand{\articulotitulocorto}{Enseñanza de funciones exponenciales y logarítmicas mediada por tecnologías digitales}
\section{\articulotitulo}
\desctotoc{Del Valle Vides, C. N.}

\noindent\rule{\linewidth}{2pt}

\vspace{0.25cm}

\begin{flushright}
	\addautor[cinthia.vides@exa.unsa.edu.ar]{Cinthia Noelia Del Valle Vides}{Universidad Nacional de Salta}
\end{flushright}

\vspace{0.5cm}

\begin{center}
	\begin{minipage}{0.75\linewidth} \small
		\textsc{Resumen}. ~
		En esta comunicación, se describe una experiencia áulica desarrollada durante la Pasantía Profesional de la Especialización en Educación Mediada por Tecnología Digital de la Universidad Nacional del Comahue. La pasantía se llevó a cabo en la Universidad Nacional de Salta en la asignatura ""Introducción a la Matemática"". Dicho proyecto abordó la enseñanza de funciones exponenciales y logarítmicas mediada por tecnologías digitales, incluyendo Moodle, Geogebra, Chat GPT, Photomath y Whatsapp. Además se justificará la propuesta desde diferentes autores de educación mediada por tecnología digital.
		
		Los resultados de la experiencia mostraron una mejor comprensión de los conceptos matemáticos, mayor participación activa y desarrollo de competencias digitales. Este enfoque demostró cómo la mediación tecnológica puede superar barreras en la enseñanza de la matemática y optimizar el proceso educativo.
	\end{minipage}
\end{center}
%-------------------------------------------------------------------------

\subsection{Introducción}

En la presente comunicación expondré la descripción de una experiencia áulica en el marco de la realización de la Pasantía Profesional correspondiente a la carrera Especialización en Educación Mediada por Tecnología Digital de la Universidad Nacional del Comahue (Carrera con modalidad virtual). Esta Pasantía profesional se desarrolló en la Universidad Nacional de Salta- Facultad de Ciencias Exactas dentro de la asignatura Introducción a la matemática.

La realización de dicha pasantía consistió en la ejecución de un Proyecto Integrador de Saberes, que surgió como un intento de resolver el siguiente problema planteado al inicio del mismo:

\paragraph*{Problema:} “La enseñanza tradicional de la matemática en el nivel universitario no contribuye a superar las dificultades que se presentan en el aprendizaje de las funciones exponenciales y logarítmicas y no se pone en práctica la utilización de herramientas digitales como una alternativa que puede contribuir a superar dichas dificultades”

Ante esta problemática se pensó en una propuesta didáctica que incorporó algunos componentes del ABP (Aprendizaje Basado en Proyectos) asistido por TIC de acuerdo con lo planteado por \textcite{marti2010}, con el fin de mejorar la habilidad para resolver problemas y desarrollar tareas complejas, mejorar la capacidad de trabajar en equipo, aumentar el conocimiento y habilidad en el uso de las TIC en un ambiente de proyectos, incrementar las capacidades de análisis y síntesis, etc.

Sin embargo, esta propuesta no constituyó netamente un aprendizaje basado en proyectos (ABP) puesto que no fue del todo constructivista, sino que la base de la enseñanza de los conceptos matemáticos propiamente dichos siguió un modelo de enseñanza basado en clases teóricas y luego prácticas.

Los aportes del ABP se vieron reflejados en las actividades que se propusieron para lograr lo mencionado anteriormente. No obstante, este proyecto integró saberes que permitieron que los estudiantes pongan en relación los elementos teóricos y prácticos de más de una disciplina, en este caso la matemática y competencias digitales en relación al uso de Tics, teniendo en cuenta lo que plantean \textcite{torres2019}.

Además, como lo exponen \textcite{galvis2013}, las plataformas educativas pueden ser utilizadas como soporte para el desarrollo de Entornos Virtuales de Enseñanza-Aprendizaje permitiendo la inclusión de materiales multimediales diversos, el acceso a múltiples fuentes de información y la comunicación e interacción sincrónicas y asincrónicas. Estas plataformas permiten asimismo potenciar intencionalidades pedagógicas que recuperen el rol protagónico y activo de los estudiantes en los procesos de aprendizaje. Como base del EVEA de este proyecto se propuso el uso de la plataforma Moodle y un canal de comunicación con los estudiantes por medio de un grupo de Whatsapp.

\textcite{area2012} por su parte mencionan que la alfabetización digital debe representar un proceso de desarrollo de una identidad como sujeto en el territorio digital, que se caracterice por la apropiación significativa de las competencias intelectuales, sociales y éticas necesarias para interactuar con la información y para recrearla de un modo crítico y emancipador. Teniendo en cuenta estas ideas se pensó en una propuesta en la que el alumno utilice algunas herramientas como el software Geogebra para graficar funciones, Chat GPT para resolver problemas y elaborar conclusiones, la calculadora digital Photomath para comparar resultados y detectar errores en el procedimiento de resolución de ecuaciones exponenciales y logarítmicas. De esta manera, aquí no solo está de fondo lo referente a dicha alfabetización digital, sino que también se encuentran elementos de la multialfabetización como por ejemplo la alfabetización social. Además, el uso de estas herramientas les permitió desarrollar la capacidad de criticar y evaluar la información que se obtuvo.

En este aspecto, autores de Educación Matemática como por ejemplo \textcite{deguzman2004} ya aseguraba que hay que poner el acento en la comprensión de los procesos matemáticos más bien que en la ejecución de ciertas rutinas que en nuestra situación actual ocupan todavía gran parte de la energía de nuestros alumnos, con el consiguiente sentimiento de esterilidad del tiempo que en ello emplean y que lo verdaderamente importante vendrá a ser su preparación para el diálogo inteligente con las herramientas que ya existen, de las que ya disponen. Si bien el en su tiempo se refería a las calculadoras y los ordenadores que recién estaban apareciendo sus palabras hoy están más que vigentes considerando las herramientas digitales actuales que facilitan el trabajo en las aulas y fuera de ellas y seguramente seguirán estando vigentes para las herramientas que vendrán en el futuro.

Por otro lado \textcite{garcia2020} se enfocan específicamente en el uso del Geogebra para el aprendizaje de las funciones exponenciales mediante el uso de deslizadores que permiten identificar cómo varía la gráfica de una función exponencial de la forma $f(x) = k a^x + b$ al variar los parámetros $a$, $k$, y $b$. Es por ello que se dedicó un tiempo de clase especial a que los alumnos puedan explorar un recurso de Geogebra asociado a un código QR para establecer conclusiones sobre la imagen, asíntotas, intervalos de crecimiento y/o decrecimiento de la función exponencial, intersecciones con los ejes, etc. de acuerdo a la variación de dichos parámetros teniendo en cuenta que sin el uso de este recurso sería muy difícil que los alumnos pudieran establecer estas conclusiones de manera autónoma sin que el docente tuviera que dictarle las respuestas.

\subsection{Contenidos}

El contenido principal de esta comunicación será el aprendizaje de las Funciones Exponenciales y Logarítmicas mediado por tecnologías digitales a partir de una experiencia áulica llevada a cabo en una asignatura de Matemáticas previas al Cálculo. El Proyecto Integrador de Saberes que dió origen a la experiencia áulica integró los contenidos matemáticos de Logaritmo y propiedades, Funciones Exponenciales y Logarítmicas y Aplicaciones de la Función Exponencial con cuestiones teóricas y prácticas respecto a la mediación de tecnologías digitales en la enseñanza.

\subsection{Objetivos}

\begin{itemize}
	\item Fortalecer los espacios de aprendizaje de las funciones exponenciales y logarítmicas a través de la mediación digital.
	\item Compartir una experiencia exitosa de integración de tecnologías digitales en la enseñanza de la matemática.
	\item Mostrar cómo las herramientas digitales pueden superar barreras epistemológicas en la enseñanza de funciones exponenciales y logarítmicas.
	\item Reflexionar sobre cómo la mediación tecnológica puede mejorar la organización de las clases y optimizar tiempos en la enseñanza de la	matemática.
	\item Inspirar a otros educadores a incorporar tecnologías digitales en sus prácticas de enseñanza para mejorar el aprendizaje de los estudiantes.
\end{itemize}

\subsection{Desarrollo de la comunicación}

Para el desarrollo de la comunicación se utilizarán diapositivas, las cuales tendrán palabras y conceptos claves para desarrollar los siguientes punto de dicha comunicación:

\begin{enumerate}[I.]
	\item \textbf{Contexto y Justificación}: Se describirá el contexto en el que se realizó la Pasantía Profesional, es decir la institución, asignatura, características del grupo clase, trabajo previo con tecnologías digitales, etc. Además se justificará desde diferentes autores de matemática, educación matemática y educación mediada por tecnología digital la elaboración de la propuesta didáctica del proyecto y particularmente la mediación tecnológica en la enseñanza de los contenidos matemáticos mencionados.
	\item \textbf{Actividades y Herramientas Digitales Utilizadas}: Se realizará una descripción de las actividades desarrolladas y de las herramientas digitales y recursos digitales utilizados. Entre dichas herramientas se destacan las siguientes:
	\begin{itemize}
		\item \textit{Plataforma Moodle}: Utilizada para compartir materiales didácticos y crear actividades evaluativas. Dentro de los recursos propios de moodle se utilizaron “Libro”, “Cuestionario” “Tarea”, “Carpeta”, etc. En particular se destaca el recurso Libro que recopiló las principales producciones de los estudiantes, algunas orientaciones para el desarrollo de actividades propuestas y recursos elaborados para favorecer la comprensión de los alumnos. De esta manera, aquellos estudiantes que por diversos factores no pudieron asistir a clases tuvieron al alcance una herramienta que les permitiera plantear y resolver los principales ejercicios del TP en conjunto con las diapositivas de clases prácticas.
		\item \textit{Presentación con diapositivas}: Se utilizaron diapositivas que contenían los principales conceptos teóricos y los ejercicios prácticos a desarrollar en el pizarrón (que fueron completados en las mismas diapositivas con una pizarra electrónica).
		\item \textit{Geogebra}: Proporcionó recursos interactivos para la visualización y manipulación de funciones exponenciales, permitiéndoles a los estudiantes observar como varían los parámetros de una función exponencial de la forma $f(x) = k a^x + b$. Además se aprovecharon sus facilidades para la verificación de resultados en el análisis de funciones exponenciales y logarítmicas y para visualizar cómo determinadas funciones exponenciales pueden modelar ciertas situaciones problemáticas.
		\item \textit{Chat GPT}: Herramienta utilizada para resolver problemas del tipo “Obtener una función de tipo exponencial que cumpla las siguientes condiciones”, analizando de manera crítica si las respuestas que esta IA les proporcionaba eran correctas con sus correspondientes justificaciones.
		\item \textit{Photomath}: Aplicación utilizada para la verificación de resultados y procedimientos en la resolución de ecuaciones exponenciales y logarítmicas.
		\item \textit{Whatsapp}: Las interacciones en el grupo de Whatsapp permitieron mantener una comunicación constante fluida y dinámica con los estudiantes, creando también un espacio de trabajo colaborativo en el que se compartieron ejercicios, dudas, correcciones, etc.
		\item \textit{Video Educativo}: Creación de videos específicos que abordaban conceptos clave de las funciones exponenciales y logarítmicas, permitiendo a los estudiantes revisar los temas a su propio ritmo.
		\item \textit{Zoom}: Para una clase virtual extra en la que se resolvieron ejercicios integradores de exámenes parciales de años anteriores.
	\end{itemize}
	\item \textbf{Resultados y conclusiones}: En este momento se expondrán las principales conclusiones y logros obtenidos a partir de la ejecución del Proyecto Integrador de Saberes propuesto. Entre ellos, se destaca que la integración de tecnologías digitales permitió:
	\begin{itemize}
		\item \textit{Mejor comprensión}: Los estudiantes mostraron una mejor comprensión de las funciones exponenciales y logarítmicas, como se evidenció en las actividades evaluativas, en los ejercicios compartidos y en los desarrollos que se observaron en clase.
		\item \textit{Participación Activa}: Aumento en la participación de los estudiantes en discusiones y actividades en línea.
		\item \textit{Desarrollo de Competencias Digitales}: Los estudiantes adquirieron habilidades digitales útiles para resolver problemas matemáticos y para su desempeño académico y profesional en general.
	\end{itemize}
\end{enumerate}

\subsection{Bibliografía}

\nocite{*}
\printbibliography[keyword={12}]