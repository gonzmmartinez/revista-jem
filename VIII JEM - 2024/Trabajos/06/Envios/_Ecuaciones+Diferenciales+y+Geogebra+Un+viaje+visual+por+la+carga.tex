%% LyX 2.3.6.1 created this file.  For more info, see http://www.lyx.org/.
%% Do not edit unless you really know what you are doing.
\documentclass[spanish,english]{article}
\usepackage[T1]{fontenc}
\usepackage[latin9]{inputenc}
\usepackage{amsmath}
\usepackage{graphicx}
\usepackage{babel}
\addto\shorthandsspanish{\spanishdeactivate{~<>}}

\begin{document}
\title{Ecuaciones Diferenciales y Geogebra: Un viaje visual por la carga
y descarga de un capacitor en circuito RC}
\maketitle

\section*{Resumen}

Esta comunicación se basará en una aplicacin de Ecuaciones Diferenciales
aplicado a Circuitos Eléctricos. Para este caso será con el circuito
R-C (Resistor - Capacitor) cuyo modelo es $R*q'+\frac{1}{C}*q=E(t)$
(la entrada simbolizada como $E(t)$ que representará como varáa la
diferencia de potencial en función del tiempo). Si $E(t)\neq0$ representa
la carga de un capacitor C, si fuera $E(t)=0$ representará la descarga
del capacitor. La solución de la EDO con condición inicial será $q(t)$
(será la carga en función del tiempo q(t) (expresada en Coulombs)).
Se mostrará la resolución de la ecuación mediante el uso de GeoGebra
con valores particulares $R=200\varOmega,$$C=100\mu F,$$E(t)=20V$
y luego con la herramienta \textquotedblleft deslizador\textquotedblright{}
en GeoGebra se hará variar los parámetros R y C para visualizar cómo
va cambiando la función de salida, y hacer algunas interpretaciones
físicas de lo que sucederá, con respecto a la cantidad de carga, constante
de tiempo, curvas y asíntotas, etc. También se aplicará la función
derivada a la misma función $q(t)$ para obtener la intensidad de
corriente eléctrica $i(t)$ expresada en Ampere. Es una buena aplicación
del software para utilizar en contexto de aprendizaje de Física General. 

\section*{Introducción}

La propuesta tendrá una impronta tecnológica-didáctica. Debería estar
incluida esta comunicación ya que el GeoGebra es uno de los software
más utilizados en los cursos de Análisis Matemático, Física General,
etc. Al trabajar ecuaciones diferenciales, es muy importante entender
el método de resolución para poder utilizarlo pero también sería interesante
que los alumnos puedan aprender en algún curso de ecuaciones diferenciales
o de circuitos eléctricos, estas herramientas para poder modelizar
de acuerdo a sus intereses y a sus conveniencias en el contacto con
las funciones exponenciales que serían la salida de la ecuación diferencial.
El circuito R-C (Resistor - Capacitor) cuyo modelo es $R*q'+\frac{1}{C}*q=E(t)$
será tratada mediante ese software. 

\section*{Requisitos previos}

Está destinado a docentes y estudiantes del nivel superior que trabajan
en cursos de Análisis Matemático y Física General.

\section*{Desarrollo}

Los temas que se tratará en la comunicación será la presentación de
la modelización del circuito R-C mediante el uso de ecuaciones diferenciales
de primer orden. Luego se ejemplificará con valores partículares $R=200\varOmega,C=100\mu F,E(t)=20V$
para simular la carga de un capacitor $200*q'+\frac{1}{10^{-4}}*q=20$
con condición inicial $q(0)=0$ y luego $E(t)=0V$ para la descarga
del capacitor, $200*q'+\frac{1}{10^{-4}}*q=0$ con condición inicial
$q(0)=0.002$. Eso se colocará en el software GeoGebra en la Vista
Cálculo Simbólico (CAS) $ResuelveEDO[200y'+10000y=20,(0,0)]$. Cuya
solución será: $y(t)=\frac{1}{500}*e^{-50t}.$ Luego se colocará $ResuelveEDO[R*y'+\frac{1}{C}*y=20,(0,0)]$
se aplicará con la herramienta ``deslizador'' para $R$ y $C$,
por ende se colocará un incremento de $100\varOmega$, con lo cual
se aumentará o disminuirá la resistencia y también la del capacitor,
y por último se hablará de la constante de tiempo $\tau=RC$. Se hará
la interpretación física, de que si la constante de tiempo es relativamente
grande es porque la resistencia es muy grande. El circuito cargará
con más rapidez si se utiliza una resistencia más pequeña. Y también
si a mayor capacitancia mayor carga tendrá el capacitor. Y si hay
mayor resistencia habrá un retardo en la carga. 
\begin{figure}

\caption{\protect\includegraphics[scale=0.6]{\string"Circuito RC carga capacitor\string".png}}

\end{figure}

\begin{figure}

\caption{\protect\includegraphics[scale=0.6]{\string"Circuito RC descarga capacitor\string".png}}

\end{figure}


\section*{Bibliografía}

\selectlanguage{spanish}%
- Kreyszig E. (2011), \textquotedblleft Advanced Engineering Mathematics\textquotedblright{}
-- Ed. Wiley -- Boston -- Estados Unidos

- Young, Hugh; Freedman, Roger (2009), \textquotedblleft Física universitaria
volumen 1\textquotedblright{} -- Ed. Pearson -- Naucalpán de Juarez
-- México.

- Zill, Denis; Cullen, Michael (2008), <<Matemáticas avanzadas para
ingeniería, vol. 1: ecuaciones diferenciales>> - Ed. McGraw-Hill
Interamericana - México D.F

- Giancoli, Douglas (2009), <<Física para ciencias e ingeniería con
física moderna>> - 4ta Edición - Pearson Educación - México D.F

- Luna, Maritza; Requjo, Elton; Villogas, Edwin, ``Resolución de
problemas de ecuaciones diferenciales utilizando geometría dinámica''
- Grupo de Investigación TecVEM-IREM PUCP, Perú . \selectlanguage{english}%

\end{document}
