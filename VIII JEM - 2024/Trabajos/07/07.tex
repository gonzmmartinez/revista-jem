%-------------------------------------------------------------------------
% INFORMACIÓN DEL ARTÍCULO
\thispagestyle{portadapage}
\setcounter{subsection}{0}
\setcounter{subsubsection}{0}
\setcounter{actividad}{0}
\setcounter{actividad_previa}{0}
\setcounter{actividad_entre}{0}
\renewcommand{\articulotipo}{Comunicación breve}
\renewcommand{\articulotitulo}{Investigación sobre las estrategias de enseñanza implementadas en un profesorado de matemática durante la pandemia por COVID-19}
\phantomsection
\stepcounter{section}
\addcontentsline{toc}{section}{\protect\numberline{\thesection} \articulotitulo}
\desctotoc{Montenegro, F.; Cravero, M.; Fernández, C.; Salazar, M. G. R.}

\begin{center}
	\setstretch{1.5}
	{\Huge \scshape 
		\articulotitulo
	}
\end{center}

\noindent\rule{\linewidth}{2pt}

\vspace{0.25cm}

\begin{flushright}
	{\Large \scshape
		Fabiana Montenegro
	}\\
	{\large \itshape
		Escuela Normal Superior N° 32 “General José de San Martín”
	}\\
	{\ttfamily \small
		montenegrofg@gmail.com
	}\\ \vspace{1em}
	{\Large \scshape
		Mariela Cravero
	}\\
	{\large \itshape
		Escuela Normal Superior N° 32 “General José de San Martín”
	}\\ \vspace{1em}
	{\Large \scshape
		Carlos Fernández
	}\\
	{\large \itshape
		Escuela Normal Superior N° 32 “General José de San Martín”
	}\\ \vspace{1em}
	{\Large \scshape
		María Gracia Raquel Salazar
	}\\
	{\large \itshape
		Escuela Normal Superior N° 32 “General José de San Martín”
	}\\
\end{flushright}

\vspace{0.5cm}

\begin{center}
	\begin{minipage}{0.75\linewidth} \small
		\textsc{Resumen}. ~
		Con el objeto de promover la función de investigación y suscitar la producción de conocimiento educativo y pedagógico en el nivel Superior y en los Institutos Superiores de Formación Docente (ISFD) el Instituto Nacional de Formación Docente (INFoD) en el año 2021 convocó a las y los docentes y estudiantes de carreras de Formación Docente de Institutos Superiores de gestión estatal a presentar proyectos de investigación. El eje priorizado, en dicha convocatoria, fue la educación en Argentina en el contexto de la emergencia sanitaria generada por el Coronavirus SARS-CoV-2. En el marco de dicha convocatoria los autores de esta ponencia presentamos un proyecto cuyo objetivo fue indagar y analizar, desde lo didáctico y lo curricular, las estrategias de enseñanza implementadas por las y los docentes de los espacios curriculares de la formación específica y de la formación en la práctica profesional del Profesorado de Educación Secundaria en Matemática del ISFD N°32 de Santa Fe, durante la pandemia por COVID-19. En este trabajo  presentamos los resultados de una encuesta que formó parte de la investigación desarrollada en el 2022.
	\end{minipage}
\end{center}
%-------------------------------------------------------------------------

\subsection{Introducción}

Esta ponencia describe algunos resultados derivados de una investigación referida a las estrategias de enseñanza llevadas a cabo en un profesorado de matemática para la educación secundaria durante los años de la pandemia. Consideramos que esta comunicación en las JEM favorece el intercambio de análisis de experiencias de un hecho reciente que obligó a todos los niveles del sistema educativo de 190 países del mundo a desarrollar tareas educativas en entornos virtuales.

\subsection{Requisitos previos}

Profesores y profesoras de Matemática y estudiantes de Profesorado de Matemática.

\subsection{Desarrollo}

Como consecuencia del Aislamiento Social Preventivo y Obligatorio (ASPO) el cuerpo docente se vio en la necesidad de adaptarse y brindar respuestas a las necesidades educativas de los y las estudiantes. Tal como otros educadores, llamamos a este escenario Educación Remota de Emergencia (ERE).

En este contexto, para desarrollar las propuestas educativas gran parte del colectivo docente recurrió al uso de TIC. Según \textcite{coll2011}, las propuestas educativas que incorporan las TIC con un cierto nivel de elaboración deberían incluir tres tipos de ingredientes: los recursos y software que profesores y alumnos utilizan para enseñar y aprender; un diseño instruccional elaborado y explícito con objetivos, contenidos y actividades de enseñanza, aprendizaje y evaluación y un conjunto de normas y sugerencias sobre cómo utilizar dichas herramientas. En síntesis, las propuestas educativas apoyadas en las TIC incluyen dos aspectos interdependientes: tecnológicos como pedagógicos, que se integran en lo que \textcite{coll2011} denominó un diseño tecnopedagógico.

Cabe aclarar que al hablar de TIC haremos referencia tanto a los dispositivos, aplicaciones, software informáticos como a las acciones que éstos habilitan para la enseñanza y el aprendizaje de la matemática. En cuanto a los usos que pueden darse a dichos recursos adoptamos la clasificación que se presenta en \textcite{bravo2016} diferenciando 2 dimensiones: los vinculados a la matemática (para producir matemática, para recibir información matemática, para comunicar información matemática) y los vinculados con la comunicación.

En relación al concepto de estrategias de enseñanza (EE), recuperamos la conceptualización de \textcite{anijovich2010} que define las EE como el conjunto de decisiones que asume el docente para orientar la enseñanza de un contenido disciplinar teniendo en cuenta qué quiere que los estudiantes comprendan, por qué y para qué.

Bajo este marco de referencia, consideramos indagar y analizar las EE implementadas por los y las docentes para desarrollar su propuesta educativa.

Considerando que la complejidad del objeto de estudio de esta investigación requería la utilización e integración de diferentes abordajes metodológicos, planteamos una investigación exploratoria y descriptiva y optamos por el enfoque metodológico conocido como método mixto, que complementa los métodos cualitativos y cuantitativos.

Para dar respuesta a nuestro problema de investigación, en primer lugar, se profundizó el marco teórico. En segundo lugar, se llevó a cabo una encuesta online a través de formulario de Google a las y los profesores de manera voluntaria y anónima. Seguidamente, se realizaron entrevistas en profundidad a algunos de las y los docentes encuestados. Culminamos analizando materiales didácticos que nos compartieron algunos docentes de manera voluntaria y que se desarrollaron durante el ASPO.

Por cuestiones de espacio a continuación presentaremos el análisis de una pregunta que formó parte de la encuesta. Decidimos presentar, en este trabajo, los resultados parciales de la encuesta puesto que constituyó nuestro primer instrumento de recolección de datos, porque nos permitió aprovechar conceptos y gráficos variados de la estadística descriptiva y porque de ella surgieron hipótesis que buscamos ratificar o rectificar en las entrevistas.

En una de las preguntas se pedía vincular diferentes recursos tecnológicos con los usos que se les dieron para la enseñanza de la matemática. Las y los docentes encuestados debieron elegir entre los siguientes recursos: aula virtual del INFoD, videoconferencias, mensajería interna, aplicativos, presentación, buscadores y redes sociales. Y los posibles usos dados: para debatir, compartir información, buscar información, emplear en la ejercitación, producir materiales, aprender definiciones, propiedades, procedimientos, comunicarse con las/os estudiantes.

Luego de analizar los resultados de las encuestas y empleando gráficos estadísticos para su interpretación puede concluirse que, el debate se desarrolló durante la ERE en mayor medida a través del Aula virtual del INFoD (60\%) y de videoconferencias (70\%). Todos los recursos tecnológicos considerados fueron empleados en el debate con y entre los estudiantes. Los recursos mayormente empleados para compartir resolución de ejercicios fueron el Aula Virtual del INFoD (60\%), videoconferencias (55\%) y presentaciones (50\%). También fueron importantes los aplicativos (GeoGebra, Excel, etc.) (35\%). Para la búsqueda de información no se emplearon las redes sociales como se esperaba; sino que, fue mayormente a través de buscadores de la web (Google, Bing, etc.) (55\%) y del Aula Virtual del INFoD (50\%).

Los aplicativos fueron el recurso más utilizado en el empleo de ejercitación (60\%). Las presentaciones (30\%) y el Aula Virtual del INFoD (35\%), suman aproximadamente el mismo porcentaje que las aplicaciones. Se puede asumir que la producción de materiales se canalizó por el Aula Virtual del INFoD (50\%) y por presentaciones (50\%).

El aprendizaje de definiciones, propiedades y procedimientos se desarrolló a través de tareas empleando mayoritariamente el Aula Virtual del INFoD como recurso (85\%). Además, se estimó que también se desarrolló a través de las videoconferencias (60\%). Resulta llamativo que aplicaciones --como GeoGebra-- no presenten mayor porcentaje debido a la posibilidad que ofrecen para visualizar, explorar y conjeturar. Las posibles comunicaciones docente-estudiante se establecieron principalmente a través del Aula Virtual del INFoD (65\%), en menor medida por videoconferencias (50\%) y en tercer lugar por el uso de mensajes a través del teléfono celular (30\%). Podemos distinguir que el primer y tercer grupo se refieren a comunicaciones asincrónicas mientras que las del segundo grupo son sincrónicas. De acuerdo a los resultados obtenidos, se puede aseverar que el uso de las comunicaciones asincrónicas es aproximadamente el doble que las sincrónicas, coincidiendo con las directrices establecidas por el INFoD.

Es posible, entonces, concluir que en los espacios curriculares de las y los profesores que respondieron a la encuesta aparecen las cuatro dimensiones en cuanto a los uso de los recursos informáticos que se consideran en \textcite{bravo2016}. El hecho de que los recursos más empleados fueron el Aula Virtual del INFoD y las videoconferencias puede explicarse por diversos motivos. En cuanto a la primera muchos de las y los docentes ya tenían parte de sus cátedras virtualizadas ya que nuestro IFD tiene acceso al uso de la misma desde el año 2014 y por disposición institucional fue el único medio por el que las y los profesores debíamos comunicarnos con las y los estudiantes durante el 2020. Y en relación a las videoconferencias suponemos que fue el recurso que permitió mediar en el vínculo pedagógico y contener, no sólo académicamente, a las y los estudiantes.

\subsection{Bibliografía}

\nocite{*}
\printbibliography[keyword={07}]