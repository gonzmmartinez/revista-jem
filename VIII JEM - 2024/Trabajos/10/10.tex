%-------------------------------------------------------------------------
% INFORMACIÓN DEL ARTÍCULO
\thispagestyle{portadapage}
\setcounter{subsection}{0}
\setcounter{subsubsection}{0}
\setcounter{actividad}{0}
\setcounter{actividad_previa}{0}
\setcounter{actividad_entre}{0}
\renewcommand{\articulotipo}{Comunicación breve}
\renewcommand{\articulotitulo}{Trabajo conjunto: primario, secundario, terciario y universitario para una Olimpiada Matemática Argentina en Salta}
\phantomsection
\stepcounter{section}
\addcontentsline{toc}{section}{\protect\numberline{\thesection} \articulotitulo}
\desctotoc{Flores  Rocha, V.}

\begin{center}
	\setstretch{1.5}
	{\Huge \scshape 
		\articulotitulo
	}
\end{center}

\noindent\rule{\linewidth}{2pt}

\vspace{0.25cm}

\begin{flushright}
	{\Large \scshape
		Verónica Flores Rocha
	}\\
	{\large \itshape
		-
	}\\
	{\ttfamily \small
		vfloresrocha@gmail.com
	}\\
\end{flushright}

\vspace{0.5cm}

\begin{center}
	\begin{minipage}{0.75\linewidth} \small
		\textsc{Resumen}. ~
		Luego de recabar datos de varios años, sin incluir el año de pandemia y el pos pandémico, decidimos que necesitamos incorporar nuevas estrategias para optimizar competencias y espacios para resolución de problemas en los diferentes niveles de participantes. Las pensamos, abordamos y se concretaron dos acciones conjuntas: una el convenio con el Instituto Superior del Profesorado de Salta, contando con la participación activa de diez futuros docentes y por otro lado con el dictado de Talleres de Geometría y Combinatoria con la Universidad Nacional de Salta. Contamos con el apoyo del Ministerio de Educación de la Provincia de Salta y en relación a aulas para entrenamiento de escuelas públicas con  la EET 3138“ Alberto Einstein” . Los diferentes niveles educativos y espacios diversos posibilitaron que en el 2023 se apoye a los estudiantes que sienten el interés y desean aprender aún más sobre Matemática.
	\end{minipage}
\end{center}
%-------------------------------------------------------------------------

\subsection{Desarrollo}

La Olimpiada Matemática Argentina presenta, en la actualidad, seis certámenes diferentes: \textbf{Olimpiada Internacional Canguro}, \textbf{Olimpiada Matemática Ñandú} (orientado a primaria), \textbf{Olimpiada Matemática OMA} (orientado a secundaria), \textbf{Certamen de Literatura y Matemática} (este certamen está destinado a estudiantes de primaria y secundaria que además disfruten del placer de escribir), \textbf{Mateclubes} (primaria y secundaria donde los estudiantes participan en grupos de dos o tres estudiantes del mismo grado o curso y no necesariamente del mismo colegio), y los \textbf{Torneos de Geometría} (primaria y secundaria con certámenes independientes). Este último tuvo muy buena acogida por estudiantes de nivel secundario en Salta y nos permitió trabajar con contenidos que se encuentran en los NAPs de la provincia de una manera integrada.

A través del convenio con el Instituto Superior del Profesorado logramos ampliar la posibilidad de brindar problemas para resolución con un enfoque diferente a más de cien estudiantes brindando un espacio de aprendizaje y enseñanza de la matemática de diferentes temáticas.

Dentro de los talleres seleccionados para su realización en el marco de la Universidad Nacional de Salta, con docentes destacados en ambas materias, tuvimos un abordaje de la enseñanza y aprendizaje de la matemática desde una diferentes mirada que fue muy bien recibida por los diferentes estudiantes del medio.

Esta integración de miradas y estrategias diversas logra que un estudiante que participa de la Olimpiada Matemática Argentina, tenga una mirada abarcativa, diferente y sobre todo con marcada diferencia en cuanto al resto de los estudiantes.

Estamos seguros que los conceptos trabajados convergerán en una mejora en relación al desempeño dentro de las evaluaciones de calidad de la Provincia de Salta para este año.

Los estudiantes participantes de talleres llegan no solo de Salta Capital sino de diferentes puntos de la provincia unificando el concepto de que el aprendizaje y enseñanza de la matemática une más allá de los límites geográficos.

También para los docentes planteamos perfeccionamientos específicos con docentes que son especialistas en diferentes temáticas pertenecientes a la división educativa de la Olimpiada Matemática Argentina

Con todo esto es un placer invitar a los docentes a ser parte de esta gran propuesta en pos de un logro conjunto de mejora en la enseñanza-aprendizaje de Matemática en la provincia de Salta.