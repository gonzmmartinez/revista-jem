%-------------------------------------------------------------------------
% INFORMACIÓN DEL ARTÍCULO
\thispagestyle{portadapage}
\setcounter{subsection}{0}
\setcounter{subsubsection}{0}
\setcounter{actividad}{0}
\setcounter{actividad_previa}{0}
\setcounter{actividad_entre}{0}
\renewcommand{\articulotipo}{Comunicación breve}
\renewcommand{\articulotitulo}{Explorando nuevos recursos para la enseñanza de la matemática en la educación superior}
\renewcommand{\articulotitulocorto}{Explorando nuevos recursos para la enseñanza de la matemática en la educación superior}
\section{\articulotitulo}
\desctotoc{Gallardo Lopez, V.; Oliva, E.}

\noindent\rule{\linewidth}{2pt}

\vspace{0.25cm}

\begin{flushright}
	\addautor[vanesagallardol@gmail.com]{Vanesa Gallardo Lopez}{-}
	\vspace{1em}
	\addautor[]{Elisa Oliva}{-}
\end{flushright}

\vspace{0.5cm}

\begin{center}
	\begin{minipage}{0.75\linewidth} \small
		\textsc{Resumen}. ~
		El uso de escenarios lúdicos en la enseñanza de la matemática, en el nivel superior, puede transformar el aprendizaje, resultando altamente beneficioso. La matemática recreativa tiene un enfoque didáctico, mejorando el proceso de aprendizaje, se puede ver desde dos perspectivas: educativa y de entretenimiento. El uso de los recursos lúdicos, influye positivamente en la actitud de los estudiantes, fomentando un autoconcepto matemático positivo, aumentando la motivación en el proceso de aprendizaje, y también promueve un pensamiento innovador generando interés en la asignatura. En un aspecto recreativo, proporciona diversión satisfaciendo esta necesidad humana, saliendo en parte, de la forma de enseñanza más tradicional, ofreciendo problemas intrigantes o juegos con fines específicos. La matemática recreativa ha sido precursora de importantes desarrollos matemáticos, como en la influencia en la Teoría de Grafos por el problema de los siete puentes de Königsberg y los trabajos de Von Neumann en la Teoría de Juegos y Conducta Económica.
	\end{minipage}\\
	
	\vspace{0.5em}
	
	\begin{minipage}{0.75\linewidth} \small
		\textsc{Palabras clave} --- Espacios Lúdicos, Matemática Recreativa, Gamificación, Educación Superior
	\end{minipage}
\end{center}
%-------------------------------------------------------------------------

\subsection{Introducción}

El estudio sobre matemática recreativa, inicia como parte de un proyecto de investigación llevado adelante desde la Universidad en la cual trabajamos. El mismo se denomina: “\textbf{Recursos para la enseñanza de la Matemática: Generación de Espacios Lúdicos y Generalización de Patrones, con apoyo de Tecnologías}” Para esta comunicación, se trabajara en particular sobre matemática recreativa, conectada con gamificación pudiendo hacer uso, o no, de tecnologías.

Se considera que trabajar en un proyecto que involucre escenarios lúdicos es crucial para la comunidad educativa de matemáticas en el nivel superior. En general, esta impuesta la idea de que es una ciencia dura, genera cierto temor e incertidumbre en los estudiantes, y no es simple llevarla adelante en carreras que utilicen la matemática como recurso para la resolución de situaciones específicas de sus campos de estudio (Astronomía, Geofísica, Informática, Física, Geología, entre otros), es decir, no eligieron en sí la matemática para estudiarla (como sería el caso de futuros profesores o licenciados en matemática).

Al trabajar con escenarios lúdicos, incorporando elementos de la matemática recreativa, no solo aumenta la motivación y el interés de los estudiantes, sino que también mejora su comprensión y aprecio por la materia. Según el matemático \textcite{gardner2006}, autor destacado en el campo de la matemática recreativa, "la diversión en matemáticas puede ser la llave que abra la puerta del entendimiento profundo". Gardner promovió la idea de que los puzzles y los juegos pueden hacer que los conceptos matemáticos sean accesibles y atractivos.

\textcite{smullyan1980}, conocido por sus ingeniosos acertijos y problemas lógicos, sostuvo que "los juegos de lógica no solo entretienen, sino que también fortalecen el pensamiento crítico y la capacidad de resolución de problemas". Esta perspectiva es vital en la formación de estudiantes que utilicen las matemáticas en el nivel superior, como herramienta para la resolución de situaciones aplicadas, donde la capacidad de pensar críticamente y resolver problemas complejos es esencial.

Bertrand Russell, filosofo y matemático, subrayo la importancia de una aproximación mas relajada y contemplativa a las matemáticas, lo cual es destacable en la enseñanza en nivel superior, pensando esta ciencia como mas accesible y menos intimidante para los estudiantes. En su obra ``The Study of Mathematics'', \textcite{russell1919} argumenta que "las matemáticas, cuando se entienden adecuadamente, poseen no solo la verdad, sino la suprema belleza".

Si hacemos hincapié sobre la didáctica de las matemáticas, el uso de juegos y actividades recreativas en la enseñanza de las matemáticas puede ayudar a los estudiantes a ver la relevancia y aplicación de los conceptos matemáticos en la vida diaria, \textcite{guzman2000}. A su vez, Guzman destaca que este tipo de actividades promueven una actitud positiva hacia la matemática.

Otro aspecto que podemos destacar sobre la incorporación de elementos lúdicos en el aula, es que fomenta un ambiente de aprendizaje colaborativo y dinámico, donde los estudiantes, de una forma mas relajada, se sienten mas motivados a participar y explorar, \textcite{luque2015}. Luque Bartolo añade también, que estos métodos son especialmente efectivos en el nivel superior, donde la complejidad de los temas puede ser mitigada a través de un enfoque mas interactivo y divertido.

Incorporar escenarios lúdicos en el currículo no solo promueve un autoconcepto matemático positivo en los estudiantes, como sugiere John Allen Paulos, autor de ``Innumeracy'', quien argumenta que "la matemática recreativa puede reducir la ansiedad matemática al presentar los problemas en contextos mas amigables y menos intimidantes". Este enfoque es particularmente relevante en el nivel superior, donde muchos estudiantes pueden sentirse abrumados por la abstracción y la dificultad de los conceptos avanzados.

El uso de escenarios lúdicos, muestra ciertas capacidades para desmitificar las matemáticas, haciendo que los estudiantes se sientan mas cómodos y confiados en su aprendizaje. Promueven a su vez, la creatividad, habilidad muy valiosa no solo en las matemáticas, sino en cualquier campo académico y profesional.

La matemática recreativa y la gamificacion comparten el objetivo de hacer que el aprendizaje de las matemáticas sea mas atractivo y efectivo. Ambos enfoques utilizan elementos lúdicos para motivar a los estudiantes, promover el aprendizaje activo y desarrollar habilidades críticas, creando una experiencia de aprendizaje mas positiva y envolvente. Carina Lion presenta la gamificacion como una estrategia didáctica, a través de desafíos que nos interpelan. Permite llevar a cabo experiencias que se entraman con lo emocional, lo cual aumenta la motivación, el interés en aprender y fortalece la autoestima. Se rescata un punto muy positivo en este aspecto, y es que genera una competencia sana donde el fracaso no tiene el mismo impacto que en una evaluación, aquí hay mas vidas, aquí el error, es constructivo.

Se destaca la importancia de algunas conexiones entre la matemática recreativa (MR) y la gamificacion (GAM):
\begin{itemize}
	\item Aumentan la motivación de los estudiantes. La MR usa juegos, acertijos y problemas captando el interés y la curiosidad de los estudiantes; mientras que la GAM aplica el diseño de juegos que pueden incluir puntos, niveles, recompensas,	desafíos que incentiven a los estudiantes.
	\item Fomentan un aprendizaje activo. La MR lo hace desde la exploración y la resolución de problemas, mientras que la GAM incluye actividades donde la practica y la experimentación esta dada dentro de un marco de juego.
	\item Minimizan el temor hacia las matemáticas. La MR permite presentar el contenido de un modo menos formal, mas accesible, permitiendo reducir la ansiedad que suele presentarse ante las matemáticas. A través de la GAM los errores son vistos como oportunidades para aprender, cometer un error jugando no tendrá el mismo costo que cometerlo en evaluaciones. Esto hace que el error sea constructivo, en los juegos tengo mas vidas para seguir jugando y esto hace que se puedan abordar situaciones con una actitud mas positiva y confiada.
	\item Aplicabilidad. La MR permite mostrar aplicaciones de la matemática en situaciones cotidianas, permitiendo que el estudiante visualice la utilidad del estudio de la matemática. Por su parte, a través de la GAM conectan el contenido académico con escenarios del mundo real, con actividades de aprendizaje significativas
\end{itemize}

\subsection{Contenido}

Durante el desarrollo de esta comunicación, se espera trabajar principalmente sobre la relevancia de implementar en nuestras aulas, espacios con escenarios lúdicos, haciendo uso también de gamificacion, con especial hincapié en la aplicabilidad en la enseñanza en el nivel superior, o secundario por supuesto. Esto se debe, a que en cierto modo, esta mas impuesto, que los espacios lúdicos se dan especialmente en nivel primario. En este espacio, se desarrollaran aspectos que contemplen los beneficios que pueden otorgar las actividades lúdicas, haciendo uso, o no, de tecnología. También, se mostraran algunas experiencias llevadas a cabo con estudiantes de nivel superior con el fin de comentar los resultados que fueron obtenidos; lo cual nos permitió a su vez evaluar la pertinencia de este trabajo, y su aplicabilidad en situaciones específicas.

\subsection{Objetivos}

Dando a conocer lo que se ha investigado con este proyecto y los inicios de una puesta en practica inicial, compartiendo algunas experiencias; se espera:
\begin{itemize}
	\item Despertar el interés del auditorio sobre el uso de estos escenarios, generando motivación y entusiasmo para implementar actividades de matemática recreativa en sus aulas; permitiendo captar el interés de los estudiantes.
	\item Fomentar el uso de juegos, acertijos, problemas que muestren un escenario de aprendizaje mas atractivo y accesible para el estudiante.
	\item Propiciar el uso de actividades que no solo aumenten los conocimientos matemáticos, sino que también ofrezcan un enfoque practico y divertido.
	\item Valorar la aplicabilidad de actividades lúdicas para mejorar la comprensión de conceptos matemáticos; facilitando el entendimiento del contenido.
	\item Reconocer que no hay momentos específicos para hacer uso de estos escenarios, tampoco así con una única finalidad, la exploración del docente y su experiencia, sera determinante para ir tomando decisiones; ya sea para mostrar el inicio de algún tema, para evaluar el nivel de apropiación del contenido que se esta desarrollando, para darle un cierre a algún contenido, o tan solo jugar poniendo en practica lo aprendido.
	\item Identificar la utilidad de estos escenarios para desarrollar habilidades de pensamiento crítico y resolución de problemas.
	\item Percibir la relevancia de generar estos espacios, para promover un ambiente de aprendizaje colaborativo, facilitando la interacción entre los estudiantes.
\end{itemize}

\subsection{Desarrollo de la comunicación}

Se espera exponer esta comunicación a través de una Presentación con diapositivas, elaboradas con algún recurso como PowerPoint, Canva, entre otros. La exposición comenzara comentando el marco sobre el cual esta dado este trabajo, dentro del mencionado proyecto, a través del cual se origina este trabajo de investigación. Posteriormente, se destacara todo el contenido de interés mencionado en la comunicación referido especialmente a Matemática Recreativa y algunas conexiones con la Gamificacion; teniendo en cuenta que pueden ser empleados con los mismos fines en algunos casos. Luego, se presentaran algunas experiencias llevadas a cabo con estudiantes de nivel superior, para mostrar los resultados obtenidos con estos desarrollos, en primeras experiencias de aplicabilidad de estos enfoques, que sin dudas no son totalmente nuevos, pero si presentan grandes desafíos para comenzar a investigar sobre los mismos para llevarlos al aula.

\subsection{Bibliografía}

\nocite{*}
\printbibliography[keyword={10}]