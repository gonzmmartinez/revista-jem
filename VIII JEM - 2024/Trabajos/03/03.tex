%-------------------------------------------------------------------------
% INFORMACIÓN DEL ARTÍCULO
\thispagestyle{portadapage}
\setcounter{subsection}{0}
\setcounter{subsubsection}{0}
\setcounter{actividad}{0}
\setcounter{actividad_previa}{0}
\setcounter{actividad_entre}{0}
\renewcommand{\articulotipo}{Taller}
\renewcommand{\articulotitulo}{Uso de recursos tecnológicos para optimizar la enseñanza de la geometría}
\renewcommand{\articulotitulocorto}{Uso de recursos tecnológicos para optimizar la enseñanza de la geometría}
\section{\articulotitulo}
\desctotoc{Bifano, F. J; Carranza, P. F.}

\noindent\rule{\linewidth}{2pt}

\vspace{0.25cm}

\begin{flushright}
	\addautor[patry.26.38@gmail.com]{Patricia Angélica Ruiz}{Universidad Nacional de Salta}
	\vspace{1em}
	\addautor[]{Antonio Noé Sángari}{Universidad Nacional de Salta}
\end{flushright}

\vspace{0.5cm}

\begin{center}
	\begin{minipage}{0.75\linewidth} \small
		\textsc{Resumen}. ~
		Este curso-taller está diseñado para capacitar a docentes de matemáticas de secundaria en el uso de recursos tecnológicos para optimizar la enseñanza de la geometría. El principal objetivo es desarrollar habilidades que permitan integrar herramientas tecnológicas en el aula, favoreciendo así la comprensión espacial y geométrica de los estudiantes. Además, se busca aplicar estos conocimientos adquiridos en la práctica educativa. Los participantes tendrán acceso a materiales teóricos, actividades prácticas y herramientas de evaluación para asegurar un aprendizaje integral.
	\end{minipage}
\end{center}
%-------------------------------------------------------------------------

\subsection{Introducción}

La enseñanza de la geometría ha experimentado una notable evolución en los últimos años, gracias a los avances tecnológicos. La integración de la tecnología en el aula se ha revelado como una herramienta eficaz para mejorar el aprendizaje y la comprensión de los conceptos geométricos. Una estrategia didáctica que puede ser empleada es el uso de software y aplicaciones interactivas. Con este fin, se ha diseñado el curso-taller "Uso de recursos tecnológicos para optimizar la enseñanza de la geometría", dirigido a docentes de matemáticas en secundaria. El objetivo principal es proporcionarles las herramientas y estrategias necesarias para integrar recursos tecnológicos en su práctica docente, mejorando así la comprensión y el rendimiento académico de sus estudiantes en esta área específica del conocimiento matemático.

\subsection{Contenidos}

\subsubsection{Módulo 1: Introducción a los recursos tecnológicos en la enseñanza de la geometría}

\begin{itemize}
	\item Importancia de la tecnología en la educación. \textcite{thomas2010}.
	\item Visión general de las herramientas tecnológicas disponibles.
	\item Estrategias didácticas para integrar tecnología en la enseñanza de la geometría.
\end{itemize}

\subsubsection{Herramientas de Software para la Enseñanza de la Geometría}

\begin{itemize}
	\item Software de geometría dinámica. \textcite{hohenwarter2015}
	\item Plataformas interactivas y recursos en línea.
	\item Ejercicios prácticos y actividades interactivas. (Construcción de triángulos)
\end{itemize}

\subsubsection{Módulo 3: Implementación Práctica en el Aula}

\begin{itemize}
	\item Planificación de clases utilizando recursos tecnológicos.
	\item Ejemplos de lecciones interactivas.
	\item Evaluación del impacto de la tecnología en el aprendizaje de los estudiantes. \textcite{kay1991}
	\item Actividades prácticas y simulaciones.
\end{itemize}

\subsection{Requisitos previos}

Docentes de matemáticas de nivel secundario y estudiantes avanzados de profesorados de matemática, interesados en integrar recursos tecnológicos en la enseñanza de la geometría.

\subsection{Objetivos}

\begin{enumerate}
	\item Capacitar a los docentes en el uso de recursos tecnológicos para la enseñanza de la geometría.
	\item Mejorar la comprensión espacial y geométrica de los estudiantes mediante el uso de tecnología.
	\item Proveer estrategias didácticas para integrar tecnología en el aula.
	\item Evaluar el impacto de la tecnología en el rendimiento y comprensión de los estudiantes.
\end{enumerate}

\subsection{Actividades}

\subsubsection{Actividades previas}

En un curso de la plataforma Moodle de la Facultad de Ciencias Exactas se adjuntarán tutoriales y guías sobre el uso de herramientas tecnológicas para la enseñanza de la geometría, con un énfasis especial en \textcite{geogebra2024}. Los materiales incluirán:
\begin{itemize}
	\item \textbf{Tutoriales Interactivos}: Documentos detallados sobre la instalación y uso básico de GeoGebra, incluyendo ejemplos de construcción geométrica y manipulación de figuras.
	\item \textbf{Guías Avanzadas}: Materiales sobre funciones más avanzadas de GeoGebra, como la creación de hojas de trabajo interactivas y simulaciones dinámicas.
	\item \textbf{Videos Demostrativos}: Videos que muestran paso a paso cómo utilizar GeoGebra en un entorno de enseñanza, presentando casos de estudio y mejores prácticas.
	\item \textbf{Ejemplos de Lecciones Interactivas}: Modelos de lecciones que integran GeoGebra con la plataforma Moodle, mostrando cómo los alumnos pueden interactuar con el contenido.
	\item \textbf{Programa y Cronograma de Actividades}: Documento detallado con el plan de trabajo del curso, incluyendo fechas y temas de cada módulo.
	\item \textbf{Cuestionario de Autoevaluación}: Un cuestionario en Moodle para que los participantes puedan evaluar su conocimiento previo sobre el uso de herramientas tecnológicas en la enseñanza de la
	geometría.
\end{itemize}

\subsubsection{Primeras hora y media sincrónicas}

Comenzaremos con una breve presentación del taller, incluyendo los objetivos, el cronograma y las expectativas. Posteriormente, se realizará:

\begin{itemize}
	\item \textbf{Introducción a los Recursos Tecnológicos}: Explicación de los principales recursos tecnológicos disponibles para la enseñanza de la geometría, destacando especialmente el uso de GeoGebra.
	\item \textbf{Importancia de la Tecnología en la Enseñanza de la Geometría}: Discusión sobre cómo la tecnología puede mejorar la comprensión y la enseñanza de conceptos geométricos.
	\item \textbf{Participación de los Cursantes}: Solicitud a los participantes para que compartan sus experiencias previas con la tecnología en el aula y sus expectativas respecto al taller.
\end{itemize}

\subsubsection{Primeras tres horas entre clases}

Estas horas serán dedicadas a la preparación de las próximas clases sincrónicas donde se tratará el módulo 2. Las actividades incluirán:

\begin{itemize}
	\item \textbf{Preparación de Materiales}: Creación de documentos y videos instructivos sobre el uso de GeoGebra para hacer presentaciones y su integración con otros recursos, como la plataforma Moodle.
	\item \textbf{Desarrollo de Contenidos}: Elaboración de contenidos que los cursantes deberán revisar y practicar antes de la próxima sesión sincrónica.
	\item \textbf{Foros de Discusión}: Apertura de foros en Moodle para que los participantes puedan plantear dudas y discutir sobre los materiales proporcionados.
\end{itemize}

\subsubsection{Segundas hora y media sincrónicas}

Abordaremos el módulo 2 de manera similar a las primeras horas sincrónicas. Las actividades incluirán:

\begin{itemize}
	\item \textbf{Enfoque en el Uso de Software de Geometría Dinámica}: Demostraciones y prácticas sobre el uso avanzado de GeoGebra y otras herramientas tecnológicas.
	\item \textbf{Actividad Práctica}: Solicitud a los participantes para que creen una hoja de trabajo en GeoGebra, la suban al repositorio de Moodle, y extraigan el código para incrustarla en la plataforma.
	\item \textbf{Discusión y Feedback}: Espacio para que los participantes compartan sus creaciones y reciban retroalimentación de sus compañeros y del instructor.
\end{itemize}

\subsubsection{Segundas tres horas entre clases}

Estas horas se dedicarán a la preparación de las próximas clases sincrónicas, donde se continuará con el módulo 3 e iniciará la implementación práctica en el aula. Las actividades incluirán:

\begin{itemize}
	\item \textbf{Preparación de Materiales para el Módulo 3}: Desarrollo de contenidos y ejemplos de materiales didácticos que incorporen GeoGebra en las clases.
	\item \textbf{Creación de un Ejemplo de Clase}: Diseño de una clase modelo que los participantes puedan usar como referencia, utilizando GeoGebra como herramienta principal.
	\item \textbf{Foros de Discusión}: Espacios en Moodle para discutir sobre los materiales y prepararse para la implementación práctica.
\end{itemize}

\subsubsection{Terceras hora y media sincrónicas}

Continuaremos tratando el módulo 3, con énfasis en la planificación de clases y evaluación del impacto de la tecnología en el aprendizaje de los estudiantes. Las actividades incluirán:

\begin{itemize}
	\item \textbf{Planificación de Clases con Tecnología}: Orientación sobre cómo planificar y estructurar clases que integren herramientas tecnológicas como GeoGebra.
	\item \textbf{Evaluación del Impacto Tecnológico}: Métodos para evaluar cómo la tecnología está afectando el aprendizaje de los estudiantes y cómo se puede mejorar su uso.
	\item \textbf{Resolución de Dudas}: Espacio final para evacuar las dudas de los participantes con respecto a la evaluación final del curso y cualquier otra inquietud que puedan tener.
\end{itemize}

\subsubsection{Evaluación final}

\begin{itemize}
	\item Elaboración de un mini-proyecto final aplicando los conocimientos adquiridos.
	\item Cuestionario final que aborde los principales temas del taller.
	\item Reflexión escrita sobre la experiencia del taller y el aprendizaje obtenido.
\end{itemize}

\subsection{Bibliografía}

\nocite{*}
\printbibliography[keyword={03}]