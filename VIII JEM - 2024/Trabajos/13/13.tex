%-------------------------------------------------------------------------
% INFORMACIÓN DEL ARTÍCULO
\thispagestyle{portadapage}
\setcounter{subsection}{0}
\setcounter{subsubsection}{0}
\setcounter{actividad}{0}
\setcounter{actividad_previa}{0}
\setcounter{actividad_entre}{0}
\renewcommand{\articulotipo}{Comunicación breve}
\renewcommand{\articulotitulo}{La cota de Lamé}
\renewcommand{\articulotitulocorto}{La cota de Lamé}
\section{\articulotitulo}
\desctotoc{Zivec, R.}

\noindent\rule{\linewidth}{2pt}

\vspace{0.25cm}

\begin{flushright}
	\addautor[ricardo.zivec@unsta.edu.ar]{Ricardo Zivec}{Facultad de Ingeniería - Universidad del Norte Santo Tomas de Aquino}
\end{flushright}

\vspace{0.5cm}

\begin{center}
	\begin{minipage}{0.75\linewidth} \small
		\textsc{Resumen}. ~
		El teorema de la cota de Lamé, es un resultado en teoría de números que proporciona una cota superior para el número de pasos necesarios en el algoritmo de Euclides para encontrar el (MCD) de dos números naturales.
	\end{minipage}\\
	
	\vspace{0.5em}
	
	\begin{minipage}{0.75\linewidth} \small
	\textsc{Palabras clave} --- Comprensión, Análisis, Preparación pedagógica
	\end{minipage}
\end{center}
%-------------------------------------------------------------------------

\subsection{Introducción}

El teorema de la cota de Lamé, es un resultado en teoría de números que proporciona una cota superior para el número de pasos necesarios en el algoritmo de Euclides para encontrar el (MCD) de dos números naturales.

En la comunicación se dará una descripción general, de la importancia de la enseñanza a futuros docentes en matemáticas, la cuál es crucial por 4 razones: Profundización en la teoría de números, eficiencia algorítmica, preparación para la enseñanza, pensamiento crítico y analítico.

\subsection{Objetivos}

Este trabajo tiene como objetivos;
\begin{enumerate}
	\item 1. Comprensión Teórica: Lograr que los estudiantes comprendan el enunciado y la demostración del teorema de Lame.
	\item Aplicación Práctica: Desarrollar habilidades para aplicar el algoritmo de Euclides en la determinación del MCD y entender su eficiencia.
	\item Análisis Crítico: Fomentar la capacidad de analizar la complejidad y la importancia de la eficiencia en matemáticas.
	\item Preparación didáctica: Preparar a los estudiantes para enseñar estos conceptos a nivel escolar, explicándoles de manera clara y accesible.
\end{enumerate}

\subsection{Contenidos}
\begin{itemize}
	\item Introducción al algoritmo de Euclides.
	\item Enunciado del Teorema de Lamé: Demostración y ejemplos prácticos
	\item Análisis de eficiencia.
	\item Métodos de enseñanza.
\end{itemize}

En definitiva, la enseñanza del Teorema de Lamé a futuros profesores de matemáticas no solo enriquece su conocimiento teórico y práctico, sino que también mejora su capacidad para transmitir estos conceptos, fomentando una educación matemática más sólida y profunda.

\subsection{Bibliografía}

\nocite{*}
\printbibliography[keyword={13}]