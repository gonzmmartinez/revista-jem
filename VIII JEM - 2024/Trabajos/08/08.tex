%-------------------------------------------------------------------------
% INFORMACIÓN DEL ARTÍCULO
\thispagestyle{portadapage}
\setcounter{subsection}{0}
\setcounter{subsubsection}{0}
\setcounter{actividad}{0}
\setcounter{actividad_previa}{0}
\setcounter{actividad_entre}{0}
\renewcommand{\articulotipo}{Comunicación breve}
\renewcommand{\articulotitulo}{Estrategias para la resolución de conflictos en el nivel secundario a través de la Teoría de Juegos}
\renewcommand{\articulotitulocorto}{Estrategias para la resolución de conflictos en el nivel secundario a través de la Teoría de Juegos}
\section{\articulotitulo}
\desctotoc{Zalazar, F. B.; Gimenez, A.}

\noindent\rule{\linewidth}{2pt}

\vspace{0.25cm}

\begin{flushright}
	\addautor[lic.flaviazalazar@gmail.com]{Flavia Beatriz Zalazar}{-}
	\vspace{1em}
	\addautor[alifangi@gmail.com]{Alicia Fanny Gimenez}{Universidad Nacional de San Juan}
\end{flushright}

\vspace{0.5cm}

\begin{center}
	\begin{minipage}{0.75\linewidth} \small
		\textsc{Resumen}. ~
		 Las instituciones educativas deben ser espacios que fomenten el desarrollo de las habilidades sociales de sus estudiantes. Tienen que proveer un entorno seguro en donde se garanticen los valores como el respeto, la cooperación y la solidaridad junto con prácticas como la solución pacífica de conflictos, que se tornan más relevantes. Uno de los objetivos de nuestro grupo de investigación es mostrar que, con la incorporación de conceptos básicos de la Teoría de Juegos en los contenidos curriculares, se propiciarían cambios, en la manera de socializar, en la forma de enfrentar un conflicto (ya sea social, laboral o familiar) y en el desarrollo de los estudiantes como ciudadanos.
	\end{minipage}\\
	
	\vspace{0.5em}
	
	\begin{minipage}{0.75\linewidth} \small
	\textsc{Palabras clave} --- Teoría de juegos, Resolución de conflictos, Cooperación
	\end{minipage}
\end{center}
%-------------------------------------------------------------------------

\subsection{Introducción}

Jacques Delors (1996) propone los cuatro pilares en los que debe basarse la educación a lo largo de toda la vida: aprender a conocer, aprender a hacer, aprender a convivir y aprender a ser. Aprender a conocer implica desarrollar la capacidad para apropiarse del conocimiento de manera responsable y consciente. Aprender a hacer es desarrollar habilidades para aplicar el conocimiento en la solución de problemas de la vida diaria. \textbf{Aprender a convivir es la capacidad para comprender a otra persona, manejar conflictos y promover valores para la paz}. Aprender a ser se basa en el respeto por la personalidad de cada individuo y la libertad para expresar sus emociones, sentimientos y valores. Bajo esta premisa, y teniendo en cuenta que los conflictos surgen por la oposición de los intereses de las partes involucradas, consideramos que es preciso enseñar, tanto a los adolescentes como a sus formadores, a pensar estratégicamente, ya que así podrían tomar decisiones que solucionen tales conflictos mediante habilidades de negociación y cooperación. Es importante aprender a vivir juntos desarrollando la comprensión del otro y la percepción de las formas de interdependencia —realizar proyectos comunes y prepararse para tratar los conflictos— respetando los valores de pluralismo, la comprensión mutua y la paz.

La Teoría de Juegos es una rama de la Matemática que diseña modelos para analizar situaciones de conflicto y cooperación, entre dos o más agentes (personas, empresas, países, etc.), que toman decisiones con el fin de maximizar sus beneficios y predecir cuál será el resultado cierto o posible. Mostramos que con sencillas herramientas de la Teoría de Juegos, los alumnos podrán encontrar otras formas de resolución de conflictos basadas en la cooperación.

\subsection{Contenidos}

\begin{itemize}
	\item Introducción a la Teoría de Juegos.
	\item Pensamiento estratégico y toma de decisiones.
	\item Nociones básicas de la Teoría de Juegos.
	\item Modelos: Halcón-Paloma. El Dilema del Prisionero. Tragedia de los comunes.
	\item Situaciones que involucran los contenidos de la Teoría de Juegos para el nivel medio.
\end{itemize}

\subsection{Objetivos}

\begin{itemize}
	\item Abordar los conceptos básicos de Teoría de Juegos y las técnicas empleadas en esta teoría para el desarrollo del pensamiento estratégico y la toma de decisiones.
	\item Descubrir los cambios en el desarrollo de la competencia Matemática y la resolución de problemas, en los estudiantes a partir de la implementación del juego cooperativo como estrategia de enseñanza.
	\item Promover y describir la relación que existe entre la implementación del juego cooperativo como estrategia de enseñanza y el fortalecimiento de la competencia matemática “resolución de conflictos” en los estudiantes.
\end{itemize}

\subsection{Desarrollo de la comunicación}

Se comenzará señalando la importancia del desarrollo del pensamiento estratégico y la toma de decisiones. Esto dará lugar a una breve introducción sobre conceptos básicos de la Teoría de Juegos. A continuación, se desarrollarán algunos modelos de esta teoría como “el dilema del prisionero”, “el juego del halcón-paloma” y “la tragedia de los comunes” y cómo se pueden aplicar en la educación secundaria.

Esta exposición se desarrollará mediante una presentación de diapositivas y uso de material multimedia como videos cortos, ejemplificando algunos de los modelos antes mencionados. Se mostrarán aplicaciones de gran relevancia ya realizadas.

\subsection{Bibliografía}

\nocite{*}
\printbibliography[keyword={08}]