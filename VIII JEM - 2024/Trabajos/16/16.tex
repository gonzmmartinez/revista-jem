%-------------------------------------------------------------------------
% INFORMACIÓN DEL ARTÍCULO
\thispagestyle{portadapage}
\setcounter{subsection}{0}
\setcounter{subsubsection}{0}
\setcounter{actividad}{0}
\setcounter{actividad_previa}{0}
\setcounter{actividad_entre}{0}
\renewcommand{\articulotipo}{Comunicación breve}
\renewcommand{\articulotitulo}{Matemática y brotes epidémicos en localidades de Salta: Acciones desde el ámbito educativo}
\renewcommand{\articulotitulocorto}{Matemática y brotes epidémicos en localidades de Salta: Acciones desde el ámbito educativo}
\section{\articulotitulo}
\desctotoc{Humacata, I. C.; Guzmán, V. R.}

\noindent\rule{\linewidth}{2pt}

\vspace{0.25cm}

\begin{flushright}
	\addautor[pablofernando300494@gmail.com]{Pablo Fernando	Quintana}{-}
	\vspace{1em}
	\addautor[]{Emanuel Osedo}{-}
\end{flushright}

\vspace{0.5cm}

\begin{center}
	\begin{minipage}{0.75\linewidth} \small
		\textsc{Resumen}. ~
		La modelización de epidemias se ha convertido en un tema de actualidad con la pandemia de 2020 causada por un coronavirus. Nociones técnicas como el parámetro $R_0$, número reproductivo básico, han aparecido en el discurso de los responsables políticos más importantes de todos los países afectados en el mundo.
		
		El concepto de $R_0$ elemental conlleva la idea básica de una progresión geométrica, a una exponencial, del número de casos. Se trata de una especie de promedio, que, por lo general, no resulta un número entero.
		
		En este trabajo se presentan aplicaciones de la matemática en epidemias, se analizan algunos brotes epidémicos y estimaciones de parámetros de enfermedades transmitidas por vectores en localidades de Salta. Se discuten algunas experiencias para la incorporación de la temática en el ámbito educativo. Se comentan resultados del impacto de acciones en la formación de recursos humanos y su divulgación en distintos niveles educativos.
	\end{minipage}
\end{center}
%-------------------------------------------------------------------------

\subsection{Introducción}

El 11 de marzo de 2020, el Director General de la Organización Mundial de la Salud (OMS), Dr. Ghebreyesus, anunció que la nueva enfermedad por el coronavirus 2019 (COVID-19) puede caracterizarse como una pandemia. La caracterización de pandemia significa que la epidemia se ha extendido por varios países, continentes o todo el mundo, y que afecta a un gran número de personas. Los habitantes del mundo en ese lapso pudieron experimentar el peligro de lo que significa una pandemia, cuando se trata de enfermedades para las cuales no existen tratamientos, remedios o vacunas para enfrentar este tipo de amenaza a toda la humanidad.

En 1920 el demógrafo Alfred Lotka propuso la idea seminal del número de reproducción, como una medida de la tasa de reproducción de una población determinada. En los años 50, del siglo XX, el epidemiólogo George MacDonald sugirió usarlo para describir el potencial de transmisión de la malaria. Propuso que si $R_0$ es menor que 1, entonces la enfermedad va a desaparecer en una población. Las autoridades sanitarias realizan esfuerzos ante un brote epidémico, para intentar reducir el valor de $R_0$ para que sea inferior a 1.

Sin embargo las estimaciones se basan en escenarios teóricos dependientes de los diferentes modelos que pudieran surgir para el análisis de propagación de una enfermedad.

En éste sentido la formación de recursos humanos y la capacitación que ameritan modelos más complejos precisan de un fortalecimiento de acciones por parte de las instituciones educativas. Surge nuevamente la importancia de la interdisciplinariedad que juega la Matemática como soporte de las diferentes ciencias que hacen uso de las diversas herramientas que proporcionan los métodos de la Matemática Aplicada.

\subsection{Contenidos}

En ésta sección presentamos los contenidos que constituyen esta comunicación.

\subsubsection{Brotes epidémicos del Covid en Campo Quijano, Dengue en Cafayate}

El COVID 19 es una enfermedad respiratoria causada por el virus SARS-COV2. Este virus pertenece a la familia de los coronavirus que causan habitualmente enfermedades inofensivas para las personas como los resfriados, aunque existen los que pueden causar infecciones más graves como el llamado SARS (síndrome agudo respiratorio). Este virus causa una forma grave de neumonía, el cual provoca brotes epidémicos desde el año 2003 (\url{https://www.paho.org/es/temas/coronavirus}). En el caso de campo Quijano hubo un registro de datos de los primeros 40 días, posteriormente el sistema de testeo quedó colapsado ya que los casos eran cada vez mayores y los tests no eran de realización exclusiva del hospital sino que podrían hacerse en laboratorios privados y los resultados no eran informados.

En lo que respecta al Dengue, durante el periodo de vigilancia 2023-2024, según el Ministro de Salud Pública de Salta, en la provincia se notificaron 25.237 positivos. El aumento fue de 9.467 casos respecto al periodo anterior, en el que hubo 15.770, lo que significa un incremento del 60\% de los casos. En particular, en Cafayate, se registraron 399 casos. Llama la atención la extensión hacia departamentos de Salta que antes tenían baja incidencia y solía concentrarse en el norte provincial, en los departamentos San Martín, Orán y Anta.

\subsubsection{Matemática aplicada, modelización matemática de epidemias. Modelos clásicos. El modelo exponencial. El modelo SIR. El número de reproducción básico}

El modelo exponencial es conocido, prácticamente por alumnos de los niveles secundario, terciario y universitario. Por lo general, el sentido en las actividades de enseñanza-aprendizaje siempre es desde las expresiones funcionales a sus gráficas. El cambio de paradigma que se propone, siempre que sea posible es, ir de los datos a la obtención de las expresiones funcionales.

El modelo SIR se considera el más simple de los modelos para describir una epidemia, tiene en cuenta una población homogénea y está compuesto por tres ecuaciones diferenciales ordinarias en compartimientos que describen la dinámica de los contagios. Sus siglas se refieren a $S$ (susceptibles), $I$ (infectados) y $R$ (recuperados o removidos). En él se consideran parámetros como la tasa de transmisión y tasa de recuperación las cuales permiten calcular el número de reproducción básico $R_0$.

A partir de estos modelos es posible llevar a cabo el análisis de listas de datos en función del tiempo, por lo general días o semanas epidemiológicas, para lograr probables curvas o distribución del número de casos o incidencias o alguna transformación con el objeto de suavizar las descripciones encontrar que facilitan su análisis.

Con los datos es posible, recurrir a elementos de estadística descriptivas, estimaciones de parámetros fundamentales de posición y/o dispersión, análisis de series de tiempo, como así también parámetros específicos relacionados con epidemiología.

\subsubsection{Breve reseña de las acciones de investigación, extensión y enseñanza-aprendizaje para el fortalecimiento de la línea de trabajo matemática epidemiología y en la formación de recursos humanos}

Resultados de estas actividades constituyen parte del trabajo de Seminario del profesorado en Matemática, para la obtención del título en cuestión, que se está finalizando, surgen en el marco de las diferentes actividades fueron llevadas dentro de proyectos de investigación del Consejo de Investigación de la Universidad Nacional de Salta. También el ámbito de la Facultad de Ciencias Exactas, en el Departamento de Matemática y anteriormente como Extensión con el Ministerio de Educación de la Provincia de Salta, en su programa de capacitación docente en el interior de la provincia de Salta, Programa de Alfabetización Científica de la Unidad Técnica Provincial del Ministerio de Ciencia y Tecnología de la Provincia de Salta.

En la Facultad de Ciencias Exactas, se diseñaron asignaturas optativas para alumnos del profesorado en matemática y la licenciatura en matemática. Modelos matemáticos aplicados a la Biología y Modelos matemáticos aplicados a la Biología: Aplicaciones con situaciones didácticas. Cursos de posgrado: Series de tiempo aplicadas al Análisis de Epidemias I. Introducción a la Epidemiología Matemática de Enfermedades Infecciosas. Estas acciones dieron como resultado 3 tesis de licenciatura en matemática y 1 seminario de profesor de matemática, en la actualidad están en curso 1 tésis de maestría y 1 Seminario de profesorado en matemática. También es importante mencionar que la participación de alumnos adscritos y profesionales ha permitido que algunos estén cursando estudios de posgrado o finalizaron los mismos en la línea de epidemiología matemática.

\subsection{Objetivos}

\begin{itemize}
	\item Visualizar la dinámica temporal del modelo Exponencial y SIR a partir de datos del COVID y Dengue para localidades de Salta, Campo Quijano y Cafayate.
	\item Interpretar los parámetros estimados y describir los escenarios encontrados.
	\item Comentar situaciones diseñadas de enseñanza-aprendizaje, que forman parte del Seminario de finalización del profesorado de matemática.
	\item Reseñar actividades diagramadas en el ámbito educativo desde el Departamento de Matemática, Facultad de Ciencias Exactas y del Consejo de Investigaciones de la Universidad Nacional de Salta, para la capacitación y formación de recursos humanos en la línea de Matemática Aplicada a Epidemias.
\end{itemize}

\subsection{Desarrollo de la comunicación}

La presentación de la comunicación se llevará a cabo a través de diapositivas donde se desarrollarán detalles de las actividades mencionadas en la sección Contenidos, realizadas con el software Power Point. En las mismas también se incluirán imágenes, gráficos como apoyo para la presentación. Los gráficos fueron obtenidos con los softwares GeoGebra, Excel. Se analizarán los parámetros estimados obtenidos en base a los modelos mencionados en Contenidos.

\subsection{Bibliografía}

\nocite{*}
\printbibliography[keyword={16}]