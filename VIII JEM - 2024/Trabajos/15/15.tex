%-------------------------------------------------------------------------
% INFORMACIÓN DEL ARTÍCULO
\thispagestyle{portadapage}
\setcounter{subsection}{0}
\setcounter{subsubsection}{0}
\setcounter{actividad}{0}
\setcounter{actividad_previa}{0}
\setcounter{actividad_entre}{0}
\renewcommand{\articulotipo}{Comunicación breve}
\renewcommand{\articulotitulo}{Retos en la aplicación del aprendizaje basado en problemas en la asignatura Bioestadística y Diseño Experimental}
\renewcommand{\articulotitulocorto}{Retos en la aplicación del aprendizaje basado en problemas en la asignatura Bioestadística y Diseño Experimental}
\section{\articulotitulo}
\desctotoc{Humacata, I. C.; Guzmán, V. R.}

\noindent\rule{\linewidth}{2pt}

\vspace{0.25cm}

\begin{flushright}
	\addautor[ivonehumacata@fca.unju.edu.ar]{Ivone Carolina Humacata}{Facultad de Ciencias Agrarias, Universidad Nacional de Jujuy}
	\vspace{1em}
	\addautor[]{Vilma Roxana Guzmán}{Facultad de Ciencias Agrarias, Universidad Nacional de Jujuy}
\end{flushright}

\vspace{0.5cm}

\begin{center}
	\begin{minipage}{0.75\linewidth} \small
		\textsc{Resumen}. ~
		El Aprendizaje Basado en Problemas (ABP) es una estrategia didáctica innovadora en la enseñanza de Bioestadística y Diseño experimental. Sin embargo, su implementación enfrenta algunos desafíos como la dedicación de más tiempo por parte de estudiantes y docentes, resistencia inicial a la propuesta, necesidad de mayor coordinación grupal y dificultades en el aprendizaje de software estadístico. Para mejorar la aplicación del ABP, se proponen las siguientes estrategias: optimizar la gestión del tiempo, ofrecer cursos complementarios, crear comunidades de aprendizaje docente, conformar grupos reducidos y ofrecer cursos de formación docente y considerar al ABP como una metodología complementaria al método tradicional.
	\end{minipage}\\
	
	\vspace{0.5em}
	
	\begin{minipage}{0.75\linewidth} \small
	\textsc{Palabras clave} --- ABP, estadística, agronomía, estrategias didácticas, educación superior
	\end{minipage}
\end{center}

\vspace{1em}

\begin{center}
	\begin{minipage}{0.75\linewidth} \small
		\textsc{Abstract}. ~
		Problem-Based Learning (PBL) is an innovative teaching strategy in the teaching of Biostatistics and Experimental Design. However, its implementation faces some challenges such as the dedication of more time by students and teachers, initial resistance to the proposal, the need for greater group coordination and difficulties in learning statistical software. To improve the application of PBL, the following strategies are proposed: optimise time management, offer complementary courses, create teacher learning communities, form small groups and offer teacher training courses, and consider PBL as a complementary methodology to the traditional method.
	\end{minipage}\\
	
	\vspace{0.5em}
	
	\begin{minipage}{0.75\linewidth} \small
	\textsc{Keywords} --- BL, statistics, agronomy, teaching strategies, higher education
	\end{minipage}
\end{center}
%-------------------------------------------------------------------------

\subsection{Introducción}

La enseñanza de la estadística en el nivel superior enfrenta desafíos debido a la naturaleza abstracta y conceptual de los conceptos estadísticos. Una de las dificultades identificadas es la tendencia de los estudiantes a memorizar fórmulas y procedimientos sin comprender su significado y aplicabilidad. Esto puede conducir a un aprendizaje superficial y limitado, donde los estudiantes pueden tener dificultades para transferir los conocimientos estadísticos a situaciones reales \textcite{batanero2001}.

En este contexto, el Aprendizaje Basado en Problemas (ABP) surge como una estrategia didáctica, que permite a los estudiantes aplicar sus conocimientos previos, desarrollar competencias específicas y contextualizar su aprendizaje (Kilpatrick, 1918; \textcite{fernandez2013}). En línea con \textcite{vargas2020} “el ABP es una experiencia de aprendizaje que involucra al estudiante en un proyecto complejo y significativo, el cual permite el desarrollo integral de sus capacidades, habilidades, actitudes y valores.” (\textcite[p.~170]{vargas2020}).

La integración del ABP en la enseñanza de Bioestadística y Diseño experimental, en la educación superior, se presenta como una oportunidad para repensar las prácticas docentes y los objetivos de aprendizaje (Vargas et al., 2020). Sin embargo, es necesario identificar los retos y dificultades que enfrentan los docentes y estudiantes al implementar esta estrategia, con el fin de mejorar su aplicación y contribuir al desarrollo de competencias estadísticas y de resolución de problemas en los estudiantes.

El objetivo del presente trabajo fue identificar los retos que enfrenta la aplicación del ABP en la asignatura de Bioestadística y Diseño Experimental, en el nivel de educación superior.

\subsection{Contenidos}

El presente trabajo es de tipo cualitativo transversal. Para la identificación de los desafíos y retos en la implementación del ABP se recurrió a las entrevistas a docentes y estudiantes. Se registraron las entrevistas por medio de grabaciones, posteriormente se transcribieron para su análisis.

Los desafíos identificados en la implementación del ABP en la enseñanza de Bioestadística y Diseño experimental, coinciden con los encontrados en la literatura revisada (\textcite{gonzalezhernando2016, paredescurin2016}):
\begin{itemize}
	\item Requiere más tiempo por parte de los estudiantes, ya que asumen un rol activo y participativo en el proceso de enseñanza-aprendizaje.
	\item Demanda más tiempo del docente en cuanto a tutorías para los grupos, ya que debe diseñar los problemas, preparar material, organizar los grupos y programar el trabajo, supervisarlos y evaluarlos.
	\item Al principio, los estudiantes pueden mostrar rechazo o resistencia a la forma de aprender, al estar acostumbrados a métodos tradicionales.
	\item Es necesaria una mayor coordinación de grupo, en cuanto a las tareas y en el horario para reunirse.
	\item El aprendizaje que propone el ABP puede resultar complejo para los estudiantes, al requerir nuevas habilidades.
	\item Algunos temas conceptuales de la asignatura cobran más relevancia que otros, dependiendo del problema planteado a resolver.
	\item Surgen problemas interpersonales en el grupo, debido a la falta de compromiso o participación en la realización de las actividades asignadas.
	\item Dificultades en el aprendizaje y aplicación del software estadístico R, especialmente en el uso de paquetes, librerías y funciones
\end{itemize}

\subsubsection{Estrategias para mejorar la implementación del ABP}

A partir de los desafíos identificados en la implementación del ABP se proponen las siguientes estrategias para mejorar su aplicación:
\begin{enumerate}
	\item Optimizar la gestión del tiempo de clases, mediante la creación de videos tutoriales que permitan a los estudiantes acceder a material de apoyo de manera asincrónica.
	\item Ofrecer cursos curriculares complementarios para el uso de software estadístico.
	\item Crear comunidades de aprendizaje docente que fomenten espacios de reflexión y discusión sobre las prácticas de enseñanza-aprendizaje basadas en el ABP.
	\item Conformar grupos de trabajo reducidos (hasta 3 integrantes) y permitir que los estudiantes elijan libremente a sus compañeros.
	\item Ofrecer cursos de formación docente para los docentes, con el fin de profundizar en la implementación de estrategias didácticas como el ABP en Bioestadística y diseño experimental.
	\item Considerar el ABP como una metodología docente complementaria al método tradicional.
\end{enumerate}

\subsection{Objetivos}

\subsubsection{General}

Identificar los retos que enfrenta la aplicación del Aprendizaje Basado en Problemas (ABP) en la asignatura de Bioestadística y Diseño Experimental en el nivel de educación superior.

\subsubsection{Específicos}

\begin{itemize}
	\item Identificar las principales dificultades y desafíos que enfrentan los docentes y los estudiantes al implementar el ABP en la enseñanza de Bioestadística y Diseño Experimental.
	\item Proponer estrategias para mejorar la aplicación del ABP en la enseñanza de Bioestadística y Diseño Experimental.
\end{itemize}

\subsection{Desarrollo de la comunicación}

\subsubsection{Introducción (2 minutos)}

\begin{itemize}
	\item Presentación del tema y objetivos de la comunicación breve.
	\item Contextualización de la importancia del tema en el campo de estudio.
\end{itemize}

\subsubsection{Desarrollo (10 minutos)}

\begin{itemize}
	\item Definición y características del Aprendizaje Basado en Problemas (ABP).
	\item Explicación del ABP como estrategia didáctica.
	\item Ventajas y beneficios del ABP en la enseñanza de Bioestadística y Diseño Experimental.
	\item Identificación de los principales desafíos para los docentes y los estudiantes.
	\item Sugerencias y recomendaciones para superar las dificultades identificadas.
\end{itemize}

\subsubsection{Conclusión (1 minuto)}

\begin{itemize}
	\item Resumen de los principales puntos abordados.
	\item Reafirmación de la importancia del tema y sus implicaciones en la enseñanza de Bioestadística y Diseño Experimental.
\end{itemize}

\subsubsection{Material de apoyo}

\begin{itemize}
	\item Uso de diapositivas para resaltar los puntos clave.
\end{itemize}

\subsection{Bibliografía}

\nocite{*}
\printbibliography[keyword={15}]