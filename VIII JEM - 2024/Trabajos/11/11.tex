%-------------------------------------------------------------------------
% INFORMACIÓN DEL ARTÍCULO
\thispagestyle{portadapage}
\setcounter{subsection}{0}
\setcounter{subsubsection}{0}
\setcounter{actividad}{0}
\setcounter{actividad_previa}{0}
\setcounter{actividad_entre}{0}
\renewcommand{\articulotipo}{Comunicación breve}
\renewcommand{\articulotitulo}{La modelización en matemática desde el enfoque antropológico didáctico en la Facultad Regional Orán (UNSa)}
\renewcommand{\articulotitulocorto}{La modelización en matemática desde el enfoque antropológico didáctico en la Facultad Regional Orán (UNSa)}
\section{\articulotitulo}
\desctotoc{Villagra, C. E.; Chorolque, E.}

\noindent\rule{\linewidth}{2pt}

\vspace{0.25cm}

\begin{flushright}
	\addautor[villagracelia@gmail.com]{Celia Elizabeth Villagra}{Universidad Nacional de Salta Facultad Regional Orán}
	\vspace{1em}
	\addautor[]{Edith Chorolque}{Universidad Nacional de Salta}
\end{flushright}

\vspace{0.5cm}

\begin{center}
	\begin{minipage}{0.75\linewidth} \small
		\textsc{Resumen}. ~
		El porcentaje promedio de estudiantes que regularizaron la asignatura matemática de primer año en el año 2022, en la Facultad Regional de Orán (FRO), fue de 38\%; situación que evidencia la dificultad que tienen los alumnos para comprender la matemática. Por consiguiente, es común que se cuestionen acerca de la relevancia de lo que están aprendiendo, ya que, para muchos de ellos, el contenido carece de sentido. Es por ello que nos preguntamos cómo podremos aumentar el interés de los estudiantes hacia la matemática y mejorar el desempeño académico en la misma.  Es así que estamos llevando a cabo un proyecto de investigación en el que nos proponemos analizar la incidencia de la inclusión de la modelización en matemática como metodología de enseñanza, con el enfoque de la Teoría Antropológica Didáctica, en temas contextualizados en fenómenos propios de disciplinas troncales de algunas carreras de la Facultad Regional de Orán.
	\end{minipage}\\
	
	\vspace{0.5em}
	
	\begin{minipage}{0.75\linewidth} \small
		\textsc{Palabras clave} --- Modelización, Matemática, TAD, Universidad, Enseñanza
	\end{minipage}
\end{center}
%-------------------------------------------------------------------------

\subsection{Introducción}

Es necesario que los estudiantes universitarios le den sentido a los conceptos matemáticos que aprenden porque necesitan comprender cómo los conceptos matemáticos se aplican en situaciones cotidianas y particularmente en diferentes disciplinas. Al respecto \textcite{brousseau2007} afirma que si los estudiantes son capaces de construir su propio significado y comprensión de los conceptos matemáticos podrán transferir ese conocimiento a nuevas situaciones y de esta manera también desarrollarán el pensamiento crítico y el razonamiento lógico.

Según \textcite{chevallard1999-11}, la construcción de sentido en matemáticas implica que los estudiantes no solo adquieran conocimientos matemáticos, sino que también desarrollen la capacidad de utilizarlos de manera significativa en situaciones concretas.

La modelización matemática es una metodología de enseñanza que incluye estrategias didácticas que promueven la construcción de sentido. \textcite{florensa2020} expresan que distintas comunidades de investigadores y profesionales destacan la importancia que puede desempeñar la modelización matemática en la enseñanza y aprendizaje de matemáticas, permitiendo alternativas a una enseñanza de la matemática útil y funcional para el estudio de problemas reales. En concordancia con esta idea, \textcite{brito2011} considera que este proceso genera en los estudiantes habilidades para la solución de posibles problemas prácticos, lo que permitirá la articulación efectiva entre la teoría y la práctica profesional, la cual en la actualidad se ve como disociada.

Es la TAD la que posibilita el proceso de modelización, pero, como expresan \textcite{sanchez2017}, introduciendo una praxeología matemática de saberes que se corresponden con disciplinas ajenas a la matemática y su enseñanza. \textcite{garcia2019} afirma que la TAD estudia la difusión de conocimientos y su adquisición en la sociedad y es por eso que su campo de estudio no se limita al ámbito de las matemáticas.

Como lo mencionamos precedentemente, el principal objetivo del trabajo es analizar si la modelización en el aula de matemática con el enfoque de la TAD de fenómenos que ponen en relevancia la utilidad de la matemática en disciplinas troncales de su carrera, incide favorablemente en el aprendizaje permitiendo que mejore el rendimiento académico de los estudiantes en esta ciencia. Es decir que se espera que la construcción de sentido de los saberes matemáticos permita una mayor conexión entre la formación y el desarrollo profesional del estudiantado de las carreras que participaran en el estudio. A nivel general se espera, que se asuma que la TAD brinda herramientas que promueven un aprendizaje que propende a una visión real de la aplicación de los conocimientos en la práctica y del trabajo que deben realizar los futuros egresados como profesionales. La investigación se realiza en el marco de la Convocatoria de CIUNSA, tiene una duración de dos años (2024-2025), sigue una lógica mixta, pero con más rasgos cualitativos, se encuentra concluyendo la etapa de diagnóstico.

\subsection{Contenidos}

Planteo del Problema. Objetivos de la investigación. Importancia y relevancia del problema investigado. Marco teórico: Construcción del sentido, Modelización en Matemática, Teoría Antropológica Didáctica. Antecedentes. Metodología. Resultados esperados. Resultados parciales.

\subsection{Objetivos}

\begin{itemize}
	\item Generar un espacio para reflexionar sobre la necesidad de que los estudiantes universitarios valoren la matemática en su aspecto instrumental, particularmente en la profesión que ejercerán.
	\item Analizar aspectos de la modelización matemática como metodología de la enseñanza que podría promover aprendizajes con sentido.
	\item Reconocer la Teoría Antropológica Didáctica como un enfoque que promueve una educación más contextualizada, inclusiva y significativa.
\end{itemize}

\subsection{Desarrollo de la comunicación}

Mediante la presentación de un Power Point se desarrollarán los temas explicitados en contenidos.

\subsubsection{Planteo del problema}

Se describirá la problemática que originó el proyecto de investigación, haciendo referencia a aspectos relacionados con el aprendizaje y con
la enseñanza en matemática de las diferentes carreras de la FRO.

\subsubsection{Objetivos}

Se explicará el objetivo general.

\begin{itemize}
	\item Analizar si la utilización de la modelización en matemática integrada al enfoque de la teoría antropológica de lo didáctico impacta positivamente sobre el desempeño de los estudiantes en las cátedras de matemática, cuando los fenómenos a los que alude se vinculan a las principales disciplinas de las carreras de la Facultad de Orán y consideran el contexto sociocultural de los estudiantes.
\end{itemize}

Y los objetivos específicos

\begin{itemize}
	\item Realizar un diagnóstico situacional acerca de las tradiciones de enseñanza en matemática y el aprendizaje de los estudiantes en materias del área de matemática de la Facultad Regional de Orán (FRO).
	\item Profundizar el estado del arte respecto al proceso de modelización en matemática en el nivel universitario.
	\item Comprender los elementos básicos que caracterizan la teoría antropológica de lo didáctico (TAD).
	\item Determinar y caracterizar los fenómenos de informática, de ciencias naturales y de electrónica que pueden ser modelizados por los estudiantes de la FRO.
	\item Planificar clases de matemática promoviendo el proceso de modelización matemática de fenómenos de informática, de ciencias naturales y de electrónica.
	\item Implementar las secuencias didácticas planificadas utilizando el enfoque antropológico didáctico.
	\item Analizar la viabilidad y la adecuación de las experiencias para ser transferidas a los distintos niveles de educación.
\end{itemize}

\subsubsection{Importancia y relevancia del problema investigado}

La práctica de enseñanza de la matemática en las universidades generalmente sigue un enfoque académico y teórico que busca proporcionar a los estudiantes una comprensión profunda y rigurosa de los conceptos y principios matemáticos y nuestra universidad no está alejada de esa tradición. Es necesario que los estudiantes universitarios le den sentido a los conceptos matemáticos que aprenden porque necesitan comprender cómo los conceptos matemáticos se aplican en situaciones cotidianas y particularmente en diferentes disciplinas. Al respecto \textcite{brousseau2007} afirma que si los estudiantes son capaces de construir su propio significado y comprensión de los conceptos matemáticos podrán transferir ese conocimiento a nuevas situaciones y de esta manera también desarrollarán el pensamiento crítico y el razonamiento lógico. Según \textcite{lesh1992} "dar sentido a los conceptos matemáticos involucra el desarrollo de habilidades de razonamiento matemático, el análisis de problemas y la capacidad de justificar y comunicar ideas matemáticas".

Según \textcite{chevallard1999-11}, la construcción de sentido en matemáticas implica que los estudiantes no solo adquieran conocimientos matemáticos, sino que también desarrollen la capacidad de utilizarlos de manera significativa en situaciones concretas. Para ello, propone el concepto de ``transferencia didáctica'', que consiste en la capacidad de aplicar los conocimientos matemáticos adquiridos en una variedad de contextos y situaciones. \textcite{chevallard2007} también sostiene que la construcción de sentido en matemáticas no se limita a la adquisición de conceptos y procedimientos, sino que también implica la comprensión de las implicaciones y significados asociados a ellos. Para ello, es necesario tener en cuenta los aspectos culturales, históricos y sociales de las matemáticas, así como las interacciones entre los estudiantes, los profesores y los contenidos matemáticos. Además, destaca la importancia de considerar la diversidad de los estudiantes y sus diferentes modos de construir el sentido en matemáticas. Reconoce que cada estudiante tiene una manera única de entender y relacionarse con los conceptos matemáticos, por lo que es necesario promover en el aula un ambiente de discusión y reflexión que permita a los estudiantes construir su propio sentido en base a sus experiencias y conocimientos previos.

La modelización matemática es una metodología de enseñanza que incluye estrategias didácticas que promueven la construcción de sentido. Es la TAD la que posibilita el proceso de modelización pero, como expresan \textcite{sanchez2017}, introduciendo una praxeología matemática de saberes que se corresponden con disciplinas ajenas a la matemática y su enseñanza. \textcite{garcia2019} afirma que la TAD estudia la difusión de conocimientos y su adquisición en la sociedad y es por eso que su campo de estudio no se limita al ámbito de las matemáticas. Un ejemplo de esta expansión hacia otros dominios de la didáctica son los trabajos de \textcite{florensa2018} y de \textcite{bartolome2019} en la formación de ingenieros. Se espera que la construcción de sentido de los saberes matemáticos permita una mayor conexión entre la formación y el desarrollo profesional del estudiantado de las carreras que participaran en el estudio. A nivel general se espera que se asuma que la TAD brinda herramientas que promueven un aprendizaje que propende una visión real de la aplicación de los conocimientos en la práctica y del trabajo que deben realizar los futuros egresados como profesionales.

\subsubsection{Marco teórico}

Se hará hincapié en tres teorías fundamentales:
\begin{itemize}
	\item Construcción del sentido de la matemática. \textcite{chevallard1999-11}
	\item Modelización en matemática. \textcite{florensa2020}, \textcite{brito2011}
	\item Teoría Antropológica didáctica. \textcite{chevallard1999-11}, \textcite{garcia2019}, \textcite{bartolome2019}
\end{itemize}

\subsubsection{Antecedentes}

Se hará referencia a los pocos antecedentes sobre modelización con TAD en el nivel universitario.

\begin{itemize}
	\item “Enseñanza del Álgebra Lineal en carreras de ingeniería un análisis del proceso de la modelización matemática en el marco de la Teoría Antropológica de lo Didáctico”, Álvarez-Macea y Costas (2019). UNLP.
	\item “Modelización matemática en la formación de ingenieros. La importancia del contexto”. Meideble y Ortiz (2003-2007). UCNA.
	\item “Vista de Enseñanza de las matemáticas en ingeniería Modelación matemática y matemática contextual”. Bravo, Castañeda, Hernández (2016). UDEC.
\end{itemize}

\subsubsection{Metodología}

Se explicará la metodología de investigación que se lleva a cabo.

\begin{itemize}
	\item Metodología mixta: cualitativo y cuantitativo.
	\item Objeto de estudio: Práctica docente
	\item Paradigma cualitativo: Investigación Acción
	\item Primera Fase: Diagnóstico
	\item Segunda Fase: Planificación
	\item Tercera Fase: Acción y Evaluación
	\item Cuarta Fase: Reflexión
\end{itemize}

\subsubsection{Resultados esperados}

e espera la construcción de los objetos matemáticos con sentido a través de la modelización con temas de matemática contextualizados en fenómenos propios de disciplinas troncales de las carreras Licenciatura en Análisis de Sistemas, Tecnicatura en Informática de Gestión e Ingeniería en Recursos Naturales y Medio Ambiente de la Facultad Regional de Orán y de esta manera lograr mayor interés por la matemática en el alumnado y por ende mejorar el desempeño en el área.

\subsubsection{Resultados parciales}

Se compartirán algunos resultados parciales de la primera fase del proyecto.

\subsection{Bibliografía}

\nocite{*}
\printbibliography[keyword={11}]